% % % % % % % % % % % % % % % % % % % % % 
% numerical-intro.tex - Ian Huston
% $Id: perts.tex,v 1.9 2009/10/05 18:19:54 ith Exp $
% % % % % % % % % % % % % % % % % % % % % 
% Redefine CVSRevision for this section
\renewcommand{\CVSrevision}{\version$Id: perts.tex,v 1.9 2009/10/05 18:19:54 ith Exp $}
% % % % % % % % % % % % % % % % % % % % % % % % % % % % % % % % 
% =========================================================== %
% % % % % % % % % % % % % % % % % % % % % % % % % % % % % % % %
\chapter{Cosmological Perturbations}
\label{ch:perts}
% % % % % % % % % % % % % % % % % % % % % % % % % % % % % % % % 
% =========================================================== %
% % % % % % % % % % % % % % % % % % % % % % % % % % % % % % % %

% % % % % % % % % % % % % % % % % % % % % % % % % % % % % % % % 
% =========================================================== %
% % % % % % % % % % % % % % % % % % % % % % % % % % % % % % % %
\section{Introduction}
\label{sec:intro-numerical}
% % % % % % % % % % % % % % % % % % % % % % % % % % % % % % % % 
% =========================================================== %
% % % % % % % % % % % % % % % % % % % % % % % % % % % % % % % %

Cosmological perturbation theory is an essential tool for the analysis
of cosmological models, in particular as the amount of observational
data continues to increase. With the recent launch of the
{\sc{planck}} satellite, the {\sc{wmap}} mission reaching its eighth
year, and a host of other new experiments, we will have access to more
information about the early universe than ever before
\cite{planck,Komatsu:2008hk}.

To distinguish between theoretical models 
it is necessary to go beyond the standard statistical analyses that
have been so successful in the recent past. As a result much interest
has been focused on non-gaussianity as a new tool to help classify and
test models of the early universe. Perturbation theory beyond first
order will be required to make the best possible use of 
the data.  In this paper we outline an important step in the
understanding of perturbation theory beyond first order, demonstrating
that second order perturbations are readily amenable to numerical
calculation, even on small and intermediate scales inside the horizon.


Inflationary model building has for the past few years focused on
meeting the requirements of first order perturbation theory, namely
that the power spectra of scalar and tensor perturbations should match
that observed in the Cosmic Microwave Background (CMB).  Inflationary
models are classified and tested based on their predictions for the
power spectrum of curvature perturbations, the spectral index of these
perturbations and the ratio of tensor to scalar perturbations.  As the
potential for moving beyond first order perturbations has been
explored, these three observable quantities have been joined by a
measure of the departure from gaussianity exhibited by the
perturbations, the non-gaussianity parameter $\fnl$. This parameter is
not yet well constrained by observational data in comparison with the
other quantities but can already be used to rule out models with
particularly strong non-gaussian signatures.


There are two main approaches to studying higher order effects and
non-gaussianity.  
One approach uses nonlinear theory and a gradient expansion in various
guises, either explicitly, \eg
Refs.~\cite{Salopek:1990jq,Rigopoulos:2005xx} or in the form of the
$\Delta N$ formalism, \eg
Refs.~\cite{Starobinsky:1982ee,
Starobinsky:1986fxa, Sasaki:1995aw, Sasaki:1998ug,
Lyth:2004gb,Lyth:2005fi,Langlois:2006vv}
By virtue of having to employ a gradient expansion this approach is so
far only usable on scales much larger than the particle horizon.  The
other approach uses cosmological perturbation theory following Bardeen
\cite{Bardeen:1980kt} and extending it to second order,
e.g.~Refs.~\cite{Tomita:1967,Mukhanov:1996ak,Bruni:1996im,
  Acquaviva:2002ud,Nakamura:2003wk,Noh:2004bc,
  Bernardeau:2002jy,Maldacena:2002vr,
  Finelli:2003bp,Bartolo:2004if,Enqvist:2004bk,Lyth:2005du,Seery:2005gb,
  Malik:2003mv, Barnaby:2006cq}
(for an extensive list of references and a recent review on these
issues see Ref.~\cite{Malik:2008im}).
%
This approach works on all scales, but can be more complex than in
particular the $\Delta N$ formalism. Both these approaches give the
same results on large scales \cite{Malik:2005cy}. We
will follow the Bardeen approach in this paper.

As the first order perturbations of the inflaton field are taken in
the standard treatment to be purely gaussian it is in general
necessary to go to second order in order to understand and estimate
the non-gaussian contribution of any inflationary model (for a recent
review see Ref.~\cite{Malik:2008im}). Deriving the equations of motion is
not trivial at second order and only recently was the Klein-Gordon
equation for scalar fields derived in closed form, taking into account
metric backreaction \cite{Malik:2006ir}. This allows for the first time
a direct computation of the second order perturbation in full, in
contrast with previous attempts which have focused only on certain
terms in the expression, for example \Rref{Finelli:2006wk}.


In this paper we 
solve numerically the second order Klein-Gordon
equation in closed form in Fourier space and show that this procedure
is readily applicable to the study of non-gaussianity and other higher
order effects.
%
As this is, to our knowledge, the first numerical solution to the full
second order evolution equation we will outline the numerical steps
taken in the system we have developed, examine the current constraints
on the calculation and describe the next steps required in
detail. This calculation uses the slow roll version of the second
order equation, but solves the full non-slow roll equations for the
background and first order.
% 
The models that we test in this paper are single field models with a canonical
action. Significant second order corrections
are expected only when a non-canonical action or multiple fields are
used, or slow roll is violated. Numerical simulations will be particularly
useful in analysing models with these characteristics.
We will discuss in
Section~\ref{sec:disc} planned future work to extend our current numerical
system to deal with these extensions beyond the standard single field slow roll
inflation.


In Section \ref{sec:perts-num} we will give a brief outline of
perturbation theory and describe the second order perturbation
equations that will be numerically calculated. Chapter
\ref{ch:numericalsystem} describes the numerical implementation of the
calculation, including the initial conditions used and the
computational requirements. We present the results of this calculation
in Section \ref{sec:results} including a comparison of the second
order perturbation calculated for the $\msqphisq$ and
$\lambdaphifour$ potentials. We will discuss these results
and the next stages of this work in Section \ref{sec:disc-num}. 



% 
% 
% 
% 
% % % % % % % % % % % % % % % % % % % % % % % % % % % % % % % % 
% =========================================================== %
% % % % % % % % % % % % % % % % % % % % % % % % % % % % % % % %
\section{Perturbations}
\label{sec:perts-num}
% % % % % % % % % % % % % % % % % % % % % % % % % % % % % % % % 
% =========================================================== %
% % % % % % % % % % % % % % % % % % % % % % % % % % % % % % % %


In this section we will briefly review the derivation of first and
second order perturbations in the uniform curvature gauge and describe
the slow roll approximation that we will use in this paper. There are
many reviews on the subject of cosmological perturbation theory, and
here we will follow Ref.~\cite{Malik:2008im}.  The full closed
Klein-Gordon equation for second order perturbations was recently
given by one of the authors and we will outline the derivation in
Ref.~\cite{Malik:2006ir} below.

%%%%%%%%%%%%%%%%%%%%%%%%%%%%%%%%%%%%%
\subsection{First and Second Order}
\label{sec:fosoperts-num}
%%%%%%%%%%%%%%%%%%%%%%%%%%%%%%%%%%%%%


In this paper we will consider perturbations of a single scalar field
and will work throughout in the uniform curvature or flat gauge. Our
goal is to describe scalar perturbations up to second order and the
first step to achieve this is to examine the metric tensor:
%
\begin{eqnarray}
\label{eq:metric1-num}
%
g_{00}&=&-a^2\left(1+2\phi_1+\phi_2\right) \,, \\
%
g_{0i}&=&a^2\left(B_1+\frac{1}{2}B_2\right)_{,i}\,, \\
%
g_{ij}&=&a^2\left[\left(1-2\psi_1-\psi_2\right)\delta_{ij}
+2E_{1,ij}+E_{2,ij}\right]\,,
\end{eqnarray}
%
where $a=a(\eta)$ is the scale factor, $\eta$ conformal time,
$\delta_{ij}$ is the flat background metric, $\phi_1$ and $\phi_2$ the
lapse functions, and $\psi_1$ and $\psi_2$ the curvature perturbations
at first and second order; $B_1$ and $B_2$ and $E_1$ and $E_2$ are
scalar perturbations describing the shear.
Spatial 3-hypersurfaces are flat in our chosen gauge and so
%  
\begin{equation}
 \label{eq:gauge-num}
\wt\psi_1=\wt\psi_2=\wt E_1=\wt E_2=0 \,,
\end{equation}
% 
where the tilde denotes quantities in flat gauge.


The Sasaki-Mukhanov variable, i.e.\ the field perturbation on uniform curvature
hypersurfaces \cite{Sasaki:1986hm,Mukhanov:1988jd}, evaluated at first order is
given by
%  
\begin{equation}
\label{eq:Q1I-num}
\wt{\dvp1}=\dvp1+\frac{\vp_{0}'}{\H}\psi_1\,, 
\end{equation}
% 
where $\vp_0$ is the background value of the field and the perturbations of
$\varphi$ are defined as
%  
\begin{equation}
 \varphi(x^\mu) = \vp_0(\eta) + \dvp1(\eta, x^i) + \frac{1}{2}\dvp2(\eta, x^i)
\,.
\end{equation}
% 
At second order the Sasaki-Mukhanov variable becomes more complicated
\cite{Malik:2005cy,Malik:2003mv}:
% 
\begin{align}
\label{eq:Q2I-num}
\wt{\dvp2} &= \dvp2+\frac{\vp_{0}'}{\H}\psi_2
+\left(\frac{\psi_1}{\H}\right)^2\left[
2\H\vp_{0}'+\vp_{0}''-\frac{\H'}{\H}\vp_{0}'\right] \nonumber\\
& \quad +2\frac{\vp_{0}'}{\H^2}\psi_1'\psi_1+\frac{2}{\H}\psi_1\dvp1'
-2\delta\vp_{1,k}E_{1,}^{~k}
+{\cal{X}}\left(\psi,E\right)\,,
\end{align}
% 
where ${\cal{X}}\left(\psi,E\right)$ contains terms quadratic in
gradients of the metric perturbations $\psi_1$ and $E_1$. From now on we will
drop the tildes and talk only about variables in the flat gauge.
The potential of the scalar field is also split
% 
\begin{equation}
 U(\varphi) = \U + \delta U_{1} + \frac{1}{2}\delta U _{2}\,,\quad
 \delta U_{1} = \Uphi \dvp1 \,,\quad
 \delta U_{2} = \Upp \dvp1^2 + \Uphi\dvp2 \,,
\end{equation}
% 
where $\Uphi = \pd{U}{\varphi}$.
The Klein-Gordon equation describes the evolution of the scalar field. For the
background field we have
%
\begin{equation}
\label{eq:KGback-num}
\vp_{0}''+2\H\vp_{0}'+a^2 \Uphi = 0\,,   
\end{equation}
%
where $\H\equiv\frac{a'}{a}$ is related to the Hubble parameter $H$ by $\H=aH$.
% 
The first order equation is
%
\begin{equation}
\label{eq:KGflatsingle-num}
\dvp1''+2\H\dvp1'+2a^2 \Uphi \phi_1
-\nabla^2\dvp1-\vp_{0}'\nabla^2 B_1
-\vp_{0}'\phi'_1 + a^2 \Upp \dvp1
=0\,,
\end{equation}
%
and the second order
%
\begin{align}
\label{eq:KG2flatsingle-num}
\dvp2'' &+ 2\H\dvp2'-\nabla^2\dvp2+a^2 \Upp \dvp2
+ a^2 \Uppp (\dvp1)^2 +2a^2 \Uphi \phi_2
-\vp_{0}'\left(\nabla^2 B_2+\phi_2'\right)\nonumber\\
%
&+ 4\vp_{0}' B_{1,k}\phi_{1,}^{~k}
+2\left(2\H\vp_{0}'+a^2 \Uphi\right) B_{1,k}B_{1,}^{~k}
+4\phi_1\left(a^2 \Upp \dvp1-\nabla^2\dvp1\right) \nonumber\\
&+ 4\vp_{0}'\phi_1\phi_1'
-2\dvp1'\left(\nabla^2 B_1+\phi_1'\right)-4{\dvp1'}_{,k}B_{1,}^{~k} \nonumber \\
&= 0\,,
\end{align}
% 
where as mentioned before all the variables are now in the flat gauge.


The Einstein field equations are also required at first and second order. We
will not reproduce them here but instead refer the interested reader to Section
II B of \Rref{Malik:2006ir}.
Using the perturbed Einstein equations, the Klein-Gordon equations above can be
written in closed form at both first and second orders. These equations will
form the basis of the numerical scheme described in Chapter \ref{ch:numericalsystem}.


The dynamics of the scalar field becomes clearer in Fourier space but terms in
the second order equation of the form $\left(\dvp1(x)\right)^2$ require the use
of the convolution theorem (see for example \Rref{Vretblad:2005}).
Following Refs.~\cite{Malik:2006ir} and \cite{book:liddle} we will write
$\dvp{}(k^i)$
for the Fourier component of $\dvp{}(x)$ such that
% 
\begin{equation}
 \dvp{}(\eta, x^i) = \frac{1}{(2 \pi)^3} \int \d^3k \dvp{}(\kvi) \exp (i k_i
x^i)
\,,
\end{equation}
% 
where $\kvi$ is the comoving wavenumber.


In Fourier space the closed form of the first order Klein-Gordon equation
transforms into
%
\begin{multline}
\label{eq:fokg}
 \dvp1(\kvi)'' + 2\H \dvp1(\kvi)' + k^2\dvp1(\kvi) \\
+ a^2 \left[\Upp +
\frac{8\pi G}{\H}\left(2\vp_{0}' \Uphi + (\vp_{0}')^2\frac{8\pi G
}{\H}\U\right)\right]\dvp1(\kvi) = 0 \,.
\end{multline}
%
As mentioned above the second order equation requires more careful
consideration with terms quadratic in the first order perturbation, which
 require convolutions of the form
%
\begin{equation}
 f(x)g(x) \longrightarrow \frac{1}{(2 \pi)^3} \int \d^3q \d^3p\, \delta^3(\kvi
-\pvi -\qvi) f(\pvi)
g(q^i) \,.
\end{equation}
%
For convenience we will group the terms with gradients of $\dvp1(x)$
together and denote them by $F$. 
The full closed form second order Klein-Gordon
equation in Fourier Space is
%  
\begin{align}
\label{eq:SOKG-real-num}
\dvp2''(\kvi) &+ 2\H \dvp2'(\kvi) + k^2 \dvp2(\kvi) \nonumber \\
%
&+ a^2\left[\Upp + \frac{8\pi G}{\H}\left(2\vp_{0}'\Uphi
+ (\vp_0')^2\frac{8\pi G}{\H}\U \right) \right]\dvp2(\kvi) \nonumber \\
% 
&+ \frac{1}{(2\pi)^3}\int \d^3q \d^3p\, \delta^3(\kvi -\pvi -\qvi) \biggl\{
\biggr.\nonumber\\
&\quad \frac{16\pi G}{\H} \left[ Q \dvp1'(\pvi) \dvp1(\qvi) + \vp_{0}' a^2\Upp
\dvp1(\pvi)\dvp1(\qvi) \right]  \nonumber \\
%
&+ \left(\frac{8\pi G}{\H}\right)^2\vp_{0}'\left[2a^2\Uphi\vp_{0}'
\dvp1(\pvi)\dvp1(\qvi) + \vp_{0}'Q\dvp1(\pvi)\dvp1(\qvi) \right]
\nonumber \\
%
&- 2\left(\frac{4\pi G}{\H}\right)^2\frac{\vp_{0}' Q}{\H} \left[Q\dvp1(\pvi)
\dvp1(\qvi) +
\vp_{0}' \dvp1(\pvi) \dvp1'(\qvi)\right] \nonumber \\
%
\biggl. &+ \frac{4\pi G}{\H} \vp_{0}' \dvp1'(\pvi) \dvp1'(\qvi) 
 + a^2\left[\Uppp + \frac{8\pi G}{\H}\vp_{0}' \Upp\right] \dvp1(\pvi)
\dvp1(\qvi) \biggr\} \nonumber \\
 %
&+ F(\dvp1(\kvi), \dvp1'(\kvi)) = 0\,.
\end{align}
%
Here we use $Q=a^2 (8\pi G \U \vp_0'/\H + \Uphi)$ for convenience.
The $F$ term contains gradients of $\dvp1$ in real space and therefore
the convolution integrals include additional factors of $k$ and
$q$. It is given by
%
\begin{eqnarray}
 \label{eq:Fdvk1-fourier-num}
&&%\hspace{-5mm}
F\left(\dvp1(\kvi),\dvp1'(\kvi)\right)
= \frac{1}{(2\pi)^3}\int \d^3p \d^3q\delta^3(\kvi-\pvi-\qvi) 
\Bigg\{ 
2\left(\frac{8\pi G}{\H}\right)\frac{p_kq^k}{q^2}
\delta\vp_{1}'(\pvi)\left(
Q\dvp1(\qvi)+\vp_{0}'\dvp1'(\qvi)\right)
\nonumber\\
&&
+p^2\frac{16\pi G}{\H}\dvp1(\pvi)\vp_{0}'\dvp1(\qvi)
+\left(\frac{4\pi G}{\H}\right)^2
\frac{\vp_{0}'}{\H}\Bigg[
\left(p_lq^l-\frac{p^iq_jk^jk_i}{k^2}\right) 
\vp_{0}'\delta\vp_{1}(\pvi)\vp_{0}'\delta\vp_{1}(\qvi)
\Bigg]\nonumber\\
&&
+2\frac{Q}{\H}\left(\frac{4\pi G}{\H}\right)^2 
\frac{p_lq^lp_mq^m+p^2q^2}{k^2q^2}
\Bigg[\vp_{0}'\delta\vp_{1}(\pvi)
\left(Q\dvp1(\qvi)+\vp_{0}'\dvp1'(\qvi)\right)
\Bigg]
%
\nonumber\\
%
&&
+\frac{4\pi G}{\H}
\Bigg[
4Q\frac{q^2+p_lq^l}{k^2}\left(
\dvp1'(\pvi)\dvp1(\qvi)\right)
-\vp_{0}'p_lq^l \delta\vp_{1}(\pvi)\delta\vp_{1}(\qvi)
\Bigg]
\nonumber\\
%
&&
+\left(\frac{4\pi G}{\H}\right)^2
\frac{\vp_{0}'}{\H}\Bigg[
\frac{p_lq^lp_mq^m}{p^2q^2}
\left( Q\dvp1(\pvi)+\vp_{0}'\dvp1'(\pvi)\right)
\left(Q\dvp1(\qvi)+\vp_{0}'\dvp1'(\qvi)\right)
\Bigg]\nonumber\\
%
&&
+\frac{\vp_{0}'}{\H}
\Bigg[
8\pi G\left(\frac{p_lq^l+p^2}{k^2}q^2\dvp1(\pvi)\dvp1(\qvi)
-\frac{q^2+p_lq^l}{k^2}\dvp1'(\pvi)\dvp1'(\qvi)
\right)
\nonumber\\
%
%
&&\qquad\qquad\qquad
+\left(\frac{4\pi G}{\H}\right)^2
\frac{k^jk_i}{k^2}\Bigg(
2\frac{p^ip_j}{p^2}
\left(Q\dvp1(\pvi)+\vp_{0}'\dvp1'(\pvi)\right)
Q\dvp1(\qvi)
\Bigg)\Bigg]
\Bigg\}\,.
\end{eqnarray}


%%%%%%%%%%%%%%%%%%%%%%%%%%%%%%%%%%%%
\subsection{Slow Roll approximation}
\label{sec:slowroll}
%%%%%%%%%%%%%%%%%%%%%%%%%%%%%%%%%%%%


In order to establish the viability of a numerical calculation of the
Klein-Gordon equation we have confined ourselves in this paper to studying the
evolution in the slow roll approximation. In our case this involves taking
%
\begin{equation}
 \vp_{0}'' = \H \vp_{0}' \simeq 0\,,\quad
\frac{\left(\vp_{0}'\right)^2}{2a^2}\ll \U\,,
\end{equation}
%
such that $Q=0$ and $\H^2 = (8\pi G/3) a^2 \U$. The slow roll parameter
$\epsilon_H$ as defined in Refs.~\cite{Malik:2006ir} and \cite{Seery:2005gb} 
(which is the square-root of the usual $\epsilon$) is given by
%
\begin{equation}
 \varepsilon_H = \sqrt{4\pi G} \frac{\vp_{0}'}{\H}\,.
\end{equation}
%
With this approximation the second order equation (\ref{eq:SOKG-real-num})
simplifies dramatically, and with the $F$ term included is
%
\begin{eqnarray}
 \label{eq:KG2-fourier-sr-num}
&&\dvp2''(\kvi)+2\H\dvp2'(\kvi)+k^2\dvp2(\kvi)
+\left(a^2
\Upp-{24 \pi G}(\vp_{0}')^2\right)
\dvp2(\kvi) \\
%
&&+\int \d^3p\ \d^3q\ \delta^3(\kvi-\pvi-\qvi) \Bigg\{
a^2\left(\Uppp
+ \frac{8\pi G}{\H}\vp_{0}' \Upp\right)
 \dvp1(\pvi)\dvp1(\qvi)
+\frac{16\pi G}{\H}a^2
\vp_{0}'\Upp\dvp1(\pvi)\dvp1(\qvi)\Bigg\}
\nonumber \\
%
&&+ \frac{8\pi G}{\H}
\int \d^3p\ \d^3q\ \delta^3(\kvi-\pvi-\qvi)  \Bigg\{
%
\frac{8\pi G}{\H}\frac{p_l q^l}{q^2}\vp_{0}'\dvp1'(\pvi)
\dvp1'(\qvi)
+2p^2\vp_{0}' \dvp1(\pvi) \dvp1(\qvi)\nonumber\\
%
&&\qquad\qquad\qquad\qquad
+\vp_{0}'
\Bigg(
\left(\frac{p_lq^l+p^2}{k^2}q^2-\frac{p_lq^l}{2}\right)
\dvp1(\pvi)\dvp1(\qvi)
+\left(\frac{1}{2}-\frac{q^2+p_lq^l}{k^2}\right)
\dvp1'(\pvi)\dvp1'(\qvi)\Bigg)
\Bigg\}=0 \,.\nonumber
\end{eqnarray}
%
The numerical simulation in this paper will solve the slow roll
version of the second order above, \eq{eq:KG2-fourier-sr-num}, the first
order equation (\ref{eq:fokg}) and the background equation
(\ref{eq:KGback-num}). In the next section we set up the correct form of
these equations for the numerical simulation and discuss the
implementation and some tests of the accuracy of the method.
% 
% 
% 
% 
% % % % % % % % % % % % % % % % % % % % % % % % % % % % % % % % 
% =========================================================== %
% % % % % % % % % % % % % % % % % % % % % % % % % % % % % % % %
\section{Observable quantities}
\label{sec:observable-perts}
% % % % % % % % % % % % % % % % % % % % % % % % % % % % % % % % 
% =========================================================== %
% % % % % % % % % % % % % % % % % % % % % % % % % % % % % % % %

% 
% 
% 
% 
% % % % % % % % % % % % % % % % % % % % % % % % % % % % % % % % 
% =========================================================== %
% % % % % % % % % % % % % % % % % % % % % % % % % % % % % % % %
\section{Discussion}
\label{sec:disc-perts}
% % % % % % % % % % % % % % % % % % % % % % % % % % % % % % % % 
% =========================================================== %
% % % % % % % % % % % % % % % % % % % % % % % % % % % % % % % %