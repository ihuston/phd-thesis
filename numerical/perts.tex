% % % % % % % % % % % % % % % % % % % % % 
% numerical-intro.tex - Ian Huston
% $Id: perts.tex,v 1.27 2009/11/07 13:21:34 ith Exp $
% % % % % % % % % % % % % % % % % % % % % 
% Redefine CVSRevision for this section
\renewcommand{\CVSrevision}{\version$Id: perts.tex,v 1.27 2009/11/07 13:21:34 ith Exp $}
% % % % % % % % % % % % % % % % % % % % % % % % % % % % % % % % 
% =========================================================== %
% % % % % % % % % % % % % % % % % % % % % % % % % % % % % % % %
\chapter{Cosmological Perturbations}
\label{ch:perts}
% % % % % % % % % % % % % % % % % % % % % % % % % % % % % % % % 
% =========================================================== %
% % % % % % % % % % % % % % % % % % % % % % % % % % % % % % % %

% % % % % % % % % % % % % % % % % % % % % % % % % % % % % % % % 
% =========================================================== %
% % % % % % % % % % % % % % % % % % % % % % % % % % % % % % % %
\section{Introduction}
\label{sec:intro-numerical}
% % % % % % % % % % % % % % % % % % % % % % % % % % % % % % % % 
% =========================================================== %
% % % % % % % % % % % % % % % % % % % % % % % % % % % % % % % %

Cosmological perturbation theory is an essential tool for the analysis
of cosmological models. It will become more so as the quantity and quality
of observational
data continues to improve. With the recent launch of the
Planck satellite, the WMAP mission reaching its eighth
year, and a host of other new experiments, we will have access to more
information about the early universe than ever before
\cite{planck,Komatsu:2008hk}.

To distinguish between theoretical models, 
it is necessary to go beyond the standard statistical analyses that
have been so successful in recent years. As a result, much interest
has been focused on non-Gaussianity as a new tool to classify and
test models of the early universe. Perturbation theory beyond first
order will be required to make the best possible use of 
the data. In Chapter~\ref{ch:introduction} cosmological perturbations at first order
were introduced.  In Part~\ref{part:numerical} of this work we outline an important
step in the understanding of perturbation theory beyond first order, demonstrating
that second order perturbations are readily amenable to numerical
calculation, even on small and intermediate scales inside the horizon.


Inflationary model building has for the past few years focused on
meeting the requirements of first order perturbation theory, namely
that the power spectra of scalar and tensor perturbations, as defined in
Eqs.~\eqref{eq:Prdefn-intro} and \eqref{eq:Ptdefn-intro}, should match
those observed in the CMB.  Inflationary
models are classified and tested according to their predictions for the scalar power
spectrum, scalar spectral index and tensor-scalar ratio. 
An important observable that arises at second order is the non-Gaussianity
parameter $\fnl$. As described in Section~\ref{sec:fnl-intro}, this parameter is
not yet well constrained by observational data in comparison with the
other quantities. In Part~\ref{part:dbi}, however, it was shown that $\fnl$ can
already be used to rule out models with particularly strong non-Gaussian signatures.


There are two main approaches to studying higher order effects and
non-Gaussianity.  
The first uses nonlinear theory and a gradient expansion in various
forms, either explicitly, \eg
Refs.~\cite{Salopek:1990jq,Rigopoulos:2005xx}, or through the
$\Delta N$ formalism, \eg
Refs.~\cite{Starobinsky:1982ee,
Starobinsky:1986fxa, Sasaki:1995aw, Sasaki:1998ug,
Lyth:2004gb,Lyth:2005fi,Langlois:2006vv}.

However, a gradient expansion approach is restricted and can only be applied on
scales much larger than the particle horizon.  
The
second approach uses cosmological perturbation theory developed by Bardeen
\cite{Bardeen:1980kt} and extends it to second order,
\eg~Refs.~\cite{Tomita:1967,Mukhanov:1996ak,Bruni:1996im,
  Acquaviva:2002ud,Nakamura:2003wk,Noh:2004bc,
  Bernardeau:2002jy,Maldacena:2002vr,
  Finelli:2003bp,Bartolo:2004if,Enqvist:2004bk,Lyth:2005du,Seery:2005gb,
  Malik:2003mv, Barnaby:2006cq}\footnotemark.
\footnotetext{For an extensive list of references and a recent review on these
issues see Ref.~\cite{Malik:2008im}.}
%
This approach works on all scales, but can be more complex in comparison to the
$\Delta N$ formalism. The two methods lead to identical results on large scales
\cite{Malik:2005cy}. We
will follow the Bardeen approach here.


In Section~\ref{sec:fnl-intro} the first order perturbations of the inflaton field
were taken to be purely Gaussian. It is therefore 
necessary to go to second order if we are to understand and estimate
the non-Gaussian contribution of any inflationary model (for a recent
review see Ref.~\cite{Malik:2008im}). Deriving the equations of motion is
not trivial at second order and only recently was the Klein-Gordon
equation for scalar fields derived in closed form, taking into account the
metric backreaction \cite{Malik:2006ir}. This allows for the first time
a direct and complete computation of the second order perturbation, in
contrast with previous attempts which have focused only on certain
terms in the expression, for example \Rref{Finelli:2006wk}.


% Over the course of Chapters~\ref{ch:perts}, \ref{ch:numericalsystem} and
% \ref{ch:results} the second order Klein-Gordon
% equation in closed form is solved numerically in Fourier space and show that this
% procedure
% is readily applicable to the study of non-Gaussianity and other higher
% order effects.
In this chapter the equations of motion for first and second order field
perturbations are described. These form the basis of the numerical calculation
undertaken in Chapters~\ref{ch:numericalsystem} and \ref{ch:results}.
% As this is, to the author's knowledge, the first numerical solution to the full
% second order evolution equation 
Chapter~\ref{ch:numericalsystem} describes the numerical implementation of the
calculation, including the initial conditions used and the
computational requirements. We outline the numerical steps
taken in the system and examine the current constraints
on the calculation. The calculation is based on the slow roll version of the second
order equation, but solves the full non-slow roll equations for the
background and first order systems.
% 
We present the results of the calculation
in Chapter~\ref{ch:results}, including a comparison of the second
order scalar field perturbation calculated for specific inflationary potentials. 

%
% 
The models tested in this calculation are single field models with a canonical
action. Significant second order corrections
are expected only in models with a non-canonical action or multiple fields, or
when slow roll is violated. Numerical simulations will be particularly
useful in analysing models with these characteristics.
Section~\ref{sec:next-res} discusses planned future work to extend the current
numerical system to deal with these extensions beyond the standard single field, slow
roll inflation.


In Section~\ref{sec:perts-num} of this chapter the Klein-Gordon equations
for first and second order perturbations are introduced. These will be the central
governing equations of the numerical calculation. They are
initially written in
terms of the metric perturbations and
then described in closed form, \iec terms of the field perturbations
alone. In
Section~\ref{sec:slowroll}, the second order equation is written in a slow roll
approximation.
% 
Section~\ref{sec:observable-perts} describes the observable quantities that can be
calculated from second order field perturbations. In Section~\ref{sec:disc-perts}, we
discuss the results of this chapter.




% 
% 
% 
% 
% % % % % % % % % % % % % % % % % % % % % % % % % % % % % % % % 
% =========================================================== %
% % % % % % % % % % % % % % % % % % % % % % % % % % % % % % % %
\section{Perturbation Equations}
\label{sec:perts-num}
% % % % % % % % % % % % % % % % % % % % % % % % % % % % % % % % 
% =========================================================== %
% % % % % % % % % % % % % % % % % % % % % % % % % % % % % % % %


In this section we will briefly review the derivation of the first and
second order perturbation equations in the uniform curvature gauge and describe
the slow roll approximation that will be used. There are
many reviews on the subject of cosmological perturbation theory, and
here we will follow Ref.~\cite{Malik:2008im}.  The full closed
Klein-Gordon equation for second order perturbations was recently
derived in Ref.~\cite{Malik:2006ir} and the results of that paper will be
outlined below.

%%%%%%%%%%%%%%%%%%%%%%%%%%%%%%%%%%%%%
\subsection{Second Order Perturbations}
\label{sec:fosoperts-num}
%%%%%%%%%%%%%%%%%%%%%%%%%%%%%%%%%%%%%
In Section~\ref{sec:perts-intro} cosmological perturbations of a single scalar field
were introduced at first order. 
% In this section we will consider perturbations up to
% second order, again working in the uniform curvature or flat
% gauge. 
The methods adopted in that section can be extended at second
order to
find gauge invariant quantities. Recall that scalar quantities such as the inflaton
field, $\varphi$, can be split into an homogeneous background, $\vp_0$, and
inhomogeneous perturbations. Up to second order $\vp$ becomes
%  
\begin{equation}
 \varphi(\eta, x^\mu) = \vp_0(\eta) + \dvp1(\eta, x^i) + \frac{1}{2}\dvp2(\eta, x^i)
\,.
\end{equation}

The metric tensor $g_{\mu\nu}$ must also be perturbed up to second order. In
\eq{eq:pertmetric-intro} the vector and tensor perturbations were included at first
order. Here we consider only the scalar metric perturbations \cite{Malik:2008im}:
%
\begin{align}
\label{eq:metric1-num}
%
g_{00}&= -a^2\left(1+2\phi_1+\phi_2\right) \,, \nonumber\\
%
g_{0i}&= a^2\left(B_1+\frac{1}{2}B_2\right)_{,i}\,, \nonumber\\
%
g_{ij}&= a^2\left[\left(1-2\psi_1-\psi_2\right)\delta_{ij}
+2E_{1,ij}+E_{2,ij}\right]\,,
\end{align}
%
where $\delta_{ij}$ is the flat background metric, $\phi_1$ and $\phi_2$ are the
lapse functions at first and second order, $\psi_1$ and $\psi_2$ are the curvature
perturbations, and $B_1$, $B_2$, $E_1$ and $E_2$ are the
scalar perturbations describing the shear.
% 
In addition to the first order transformation vector \eqref{eq:xidefn-intro}, there
is
now a second order tranformation vector
% 
\begin{equation}
 \label{eq:xi2defn-perts}
\xi_2^\mu = (\alpha_2, \beta_{2,}^{~~i}) \,,
\end{equation}
% 
where the spatial vector part of the transformation has been ignored. 

As before, we
can write down how a second order scalar quantity (such as $\dvp2$) will be
transformed \cite{Malik:2005cy}:
% 
\begin{equation}
\label{eq:dvp2transform-perts}
 \wt{\dvp2} = \dvp2 + \vp_0'\alpha_2 + \alpha_1\left(\vp_0'' \alpha_1 + \vp_0'
\alpha_1' + 2\dvp1'\right) + \left(2\dvp1 + \vp_0'\alpha_1\right)_{,i}
\beta_{1,}^{~~i} \,,
\end{equation}
% 
where a tilde ($\wt{~}$) denotes a transformed quantity. 
The metric curvature perturbation transformation at first order is straightforward,
$\wt{\psi_1} = \psi_1 -\H \alpha_1$, but at second order it becomes more complicated
\cite{Malik:2008im}:
% 
\begin{equation}
 \label{eq:psi2transform-perts}
\wt{\psi_2} = \psi_2 -\H\alpha_2 -\frac{1}{4}\mathcal{X}^k_{~k} +
 \frac{1}{4}\nabla^{-2} \mathcal{X}^{ij}_{~~ij}\,.
\end{equation}
% 
The inverse
\begin{align}
 \label{Xijdef}
\mathcal{X}_{ij} \equiv 
&2\Big[\left(\H^2+\frac{a''}{a}\right)\alpha_1^2
+\H\left(\alpha_1\alpha_1'+\alpha_{1,k}\xi_{1}^{~k}
\right)\Big] \delta_{ij}\nonumber\\
%
&
+4\Big[\alpha_1\left(C_{1ij}'+2\H C_{1ij}\right)
+C_{1ij,k}\xi_{1}^{~k}+C_{1ik}\xi_{1~~,j}^{~k}
+C_{1kj}\xi_{1~~,i}^{~k}\Big] \nonumber\\
% 
&+2\left(B_{1i}\alpha_{1,j}+B_{1j}\alpha_{1,i}\right)
+4\H\alpha_1\left( \xi_{1i,j}+\xi_{1j,i}\right)
-2\alpha_{1,i}\alpha_{1,j}+2\xi_{1k,i}\xi_{1~~,j}^{~k} \nonumber \\
% 
&+\alpha_1\left( \xi_{1i,j}'+\xi_{1j,i}' \right)
+\left(\xi_{1i,jk}+\xi_{1j,ik}\right)\xi_{1}^{~k}
+\xi_{1i,k}\xi_{1~~,j}^{~k}+\xi_{1j,k}\xi_{1~~,i}^{~k} \nonumber \\
% 
&+\xi_{1i}'\alpha_{1,j}+\xi_{1j}'\alpha_{1,i}
\,.
\end{align}

% 
We will work in the uniform curvature gauge where spatial 3-hypersurfaces are flat.
This implies that
%  
\begin{equation}
 \label{eq:gauge-num}
\wt\psi_1=\wt\psi_2=\wt E_1=\wt E_2=0 \,.
\end{equation}
% 
These relations can be used at first and then at second order to define gauge
invariant variables. It follows from Section~\ref{sec:perts-intro} that the first
order
transformation variables in the flat gauge satisfy $\alpha_1 = \psi_1/\H$ and
$\beta_1
= -E_1$. At second order, for the transformation of scalar quantities, as in
\eq{eq:dvp2transform-perts}, we require only $\alpha_2$. This is found by using
\eq{eq:psi2transform-perts} to have the form
% 
\begin{equation}
 \label{eq:alpha2-perts}
\alpha_2=\frac{\psi_2}{\H}+\frac{1}{4\H}\left[
\nabla^{-2}\X^{ij}_{~~,ij}-\X^k_{ k}\right]\,,
\end{equation}
% 
where the first order gauge variables have been substituted into $\X_{ij}$.

The Sasaki-Mukhanov variable, \iec the field perturbation on uniform curvature
hypersurfaces \cite{Sasaki:1986hm,Mukhanov:1988jd}, was given at first order in
\eq{eq:flatdvp1-intro} as
%  
\begin{equation}
\label{eq:Q1I-num}
\wt{\dvp1}=\dvp1+\frac{\vp_{0}'}{\H}\psi_1\,.
\end{equation}
% 
% 
At second order the Sasaki-Mukhanov variable becomes more complicated
\cite{Malik:2005cy,Malik:2003mv}:
% 
\begin{align}
\label{eq:Q2I-num}
% 
\wt{\dvp2} = &\dvp2
+\frac{\vp_0'}{\H}\psi_2
+\frac{\vp_0'}{4\H}\left(
\nabla^{-2}\X^{ij}_{~~,ij}-\X^k_{k}\right)
\\
&+\frac{\psi_1}{\H^2}\Big[\vp_0'' {\psi_1}
+\vp_0'\left(\psi_1'-\frac{\H'}{\H}\psi_1\right)+2\H\delta\vp_1'\Big]
+\left(2\dvp1+\frac{\vp_0'}{\H}\psi_1\right)_{,k}\xi_{1\mathrm{flat}}^k \,,
\end{align}
% 
where $\xi_{1\mathrm{flat}} = -(E_{1,i} +F_{1i})$. From now on we will
drop the tildes and only refer to variables calculated in the flat gauge.


The potential of the scalar field can also be separated into homogeneous and
perturbed sectors:
% 
\begin{align}
 V(\varphi) &= \U + \delta V_{1} + \frac{1}{2}\delta V _{2}\,,\quad \\
 \delta V_{1} &= \Uphi \dvp1 \,,\quad \\
 \delta V_{2} &= \Upp \dvp1^2 + \Uphi\dvp2 \,.
\end{align}
% 


Finally, the Klein-Gordon equations describe the evolution of the scalar field and
are
found by requiring the perturbed energy-momentum tensor $T_{\mu\nu}$ to obey the
energy conservation equation $\nabla_\mu T^{\mu\nu}=0$ (see for example
\Rref{Malik:2005cy}). For the
background field, $\vp_0$, the Klein-Gordon equation is 
%
\begin{equation}
\label{eq:KGback-num}
\vp_{0}''+2\H\vp_{0}'+a^2 \Uphi = 0\,.   
\end{equation}
%
% 
% 
The first order equation is
%
\begin{equation}
\label{eq:KGflatsingle-num}
\dvp1''+2\H\dvp1'+2a^2 \Uphi \phi_1
-\nabla^2\dvp1-\vp_{0}'\nabla^2 B_1
-\vp_{0}'\phi'_1 + a^2 \Upp \dvp1
=0\,,
\end{equation}
%
and the second order version is given by
%
\begin{align}
\label{eq:KG2flatsingle-num}
\dvp2'' &+ 2\H\dvp2'-\nabla^2\dvp2+a^2 \Upp \dvp2
+ a^2 \Uppp (\dvp1)^2 +2a^2 \Uphi \phi_2
-\vp_{0}'\left(\nabla^2 B_2+\phi_2'\right)\nonumber\\
%
&+ 4\vp_{0}' B_{1,k}\phi_{1,}^{~k}
+2\left(2\H\vp_{0}'+a^2 \Uphi\right) B_{1,k}B_{1,}^{~k}
+4\phi_1\left(a^2 \Upp \dvp1-\nabla^2\dvp1\right) \nonumber\\
&+ 4\vp_{0}'\phi_1\phi_1'
-2\dvp1'\left(\nabla^2 B_1+\phi_1'\right)-4{\dvp1'}_{,k}B_{1,}^{~k} \nonumber \\
&= 0\,,
\end{align}
% 
where all the variables are now in the flat gauge.


In order to write the Klein-Gordon equations in closed form, the Einstein field
equations \eqref{eq:einstein-intro} are also required at first and second order.
These are not reproduced here, but are presented for example in Section~II~B of
\Rref{Malik:2006ir}. 

% % % % % % % % % % % % % % % % 
\subsection{Fourier Transform}
\label{sec:fourier-perts}
% % % % % % % % % % % % % % % % 
% 
In general, the dynamics of the scalar field becomes clearer in Fourier space.
However, terms at second order of the form
$\left(\dvp1(x)\right)^2$ require the use
of the convolution theorem (see for example \Rref{Vretblad:2005}).
Following Refs.~\cite{book:liddle} and \cite{Malik:2006ir} we will write
$\dvp{}(k^i)$
for the Fourier component of $\dvp{}(x)$ such that
% 
\begin{equation}
 \dvp{}(\eta, x^i) = \frac{1}{(2 \pi)^3} \int \d^3k \dvp{}(\kvi) \exp (i k_i
x^i)
\,,
\end{equation}
% 
where $\kvi$ is the comoving wavenumber.
In Fourier space, the closed form of the first order Klein-Gordon equation then
transforms into
%
\begin{multline}
\label{eq:fokg}
 \dvp1(\kvi)'' + 2\H \dvp1(\kvi)' + k^2\dvp1(\kvi) \\
+ a^2 \left[\Upp +
\frac{8\pi G}{\H}\left(2\vp_{0}' \Uphi + (\vp_{0}')^2\frac{8\pi G
}{\H}\U\right)\right]\dvp1(\kvi) = 0 \,.
\end{multline}
%


The second order equation requires a careful
consideration of terms that are quadratic in the first order perturbation. In
particular, we
require convolutions of the form
%
\begin{equation}
 f(x)g(x) \longrightarrow \frac{1}{(2 \pi)^3} \int \d^3q \d^3p\, \delta^3(\kvi
-\pvi -\qvi) f(\pvi)
g(q^i) \,.
\end{equation}
%
For convenience we will group together those terms with gradients of $\dvp1(x)$
and denote them by $F$. 
The full closed form, second order Klein-Gordon
equation in Fourier Space is then given by \cite{Malik:2006ir}
%  
\begin{align}
\label{eq:SOKG-real-num}
\dvp2''(\kvi) &+ 2\H \dvp2'(\kvi) + k^2 \dvp2(\kvi) \nonumber \\
%
&+ a^2\left[\Upp + \frac{8\pi G}{\H}\left(2\vp_{0}'\Uphi
+ (\vp_0')^2\frac{8\pi G}{\H}\U \right) \right]\dvp2(\kvi) \nonumber \\
% 
&+ \frac{1}{(2\pi)^3}\int \d^3q \d^3p\, \delta^3(\kvi -\pvi -\qvi) \biggl\{
\biggr.\nonumber\\
&\quad \frac{16\pi G}{\H} \left[ Q \dvp1'(\pvi) \dvp1(\qvi) + \vp_{0}' a^2\Upp
\dvp1(\pvi)\dvp1(\qvi) \right]  \nonumber \\
%
&+ \left(\frac{8\pi G}{\H}\right)^2\vp_{0}'\left[2a^2\Uphi\vp_{0}'
\dvp1(\pvi)\dvp1(\qvi) + \vp_{0}'Q\dvp1(\pvi)\dvp1(\qvi) \right]
\nonumber \\
%
&- 2\left(\frac{4\pi G}{\H}\right)^2\frac{\vp_{0}' Q}{\H} \left[Q\dvp1(\pvi)
\dvp1(\qvi) +
\vp_{0}' \dvp1(\pvi) \dvp1'(\qvi)\right] \nonumber \\
%
\biggl. &+ \frac{4\pi G}{\H} \vp_{0}' \dvp1'(\pvi) \dvp1'(\qvi) 
 + a^2\left[\Uppp + \frac{8\pi G}{\H}\vp_{0}' \Upp\right] \dvp1(\pvi)
\dvp1(\qvi) \biggr\} \nonumber \\
 %
&+ F(\dvp1(\kvi), \dvp1'(\kvi)) = 0\,,
\end{align}
%
where we have defined the parameter $Q=a^2 (8\pi G \U \vp_0'/\H + \Uphi)$ for
convenience.
The $F$ term contains gradients of $\dvp1$ in real space and therefore
the convolution integrals include additional factors of $k$ and
$q$. The form of $F$ is given by \cite{Malik:2006ir}
%
\begin{align}
 \label{eq:Fdvk1-fourier-num}
&F\left(\dvp1(\kvi),\dvp1'(\kvi)\right)
= \frac{1}{(2\pi)^3}\int \d^3p\, \d^3q\,\delta^3(\kvi-\pvi-\qvi) 
\Bigg\{ \nonumber \\
% 
&\quad 2\left(\frac{8\pi G}{\H}\right)\frac{p_kq^k}{q^2}
\delta\vp_{1}'(\pvi)\left(Q\dvp1(\qvi)+\vp_{0}'\dvp1'(\qvi)\right)
+p^2\frac{16\pi G}{\H}\dvp1(\pvi)\vp_{0}'\dvp1(\qvi) \nonumber \\
% 
&\quad 
+\left(\frac{4\pi G}{\H}\right)^2
\frac{\vp_{0}'}{\H}\Bigg[
\left(p_lq^l-\frac{p^iq_jk^jk_i}{k^2}\right) 
\vp_{0}'\delta\vp_{1}(\pvi)\vp_{0}'\delta\vp_{1}(\qvi)
\Bigg]\nonumber\\
% 
&\quad +2\frac{Q}{\H}\left(\frac{4\pi G}{\H}\right)^2 
\frac{p_lq^lp_mq^m+p^2q^2}{k^2q^2}
\Bigg[\vp_{0}'\delta\vp_{1}(\pvi)
\left(Q\dvp1(\qvi)+\vp_{0}'\dvp1'(\qvi)\right)
\Bigg]
%
\nonumber \displaybreak[0]\\
%
&\quad +\frac{4\pi G}{\H}
\Bigg[
4Q\frac{q^2+p_lq^l}{k^2}\left(
\dvp1'(\pvi)\dvp1(\qvi)\right)
-\vp_{0}'p_lq^l \delta\vp_{1}(\pvi)\delta\vp_{1}(\qvi)
\Bigg]
\nonumber\\
%
&\quad +\left(\frac{4\pi G}{\H}\right)^2
\frac{\vp_{0}'}{\H}\Bigg[
\frac{p_lq^lp_mq^m}{p^2q^2}
\left( Q\dvp1(\pvi)+\vp_{0}'\dvp1'(\pvi)\right)
\left(Q\dvp1(\qvi)+\vp_{0}'\dvp1'(\qvi)\right)
\Bigg]\nonumber\\
%
&\quad +\frac{\vp_{0}'}{\H}
\Bigg[
8\pi G\left(\frac{p_lq^l+p^2}{k^2}q^2\dvp1(\pvi)\dvp1(\qvi)
-\frac{q^2+p_lq^l}{k^2}\dvp1'(\pvi)\dvp1'(\qvi)
\right)
\nonumber\\
%
%
&\quad +\left(\frac{4\pi G}{\H}\right)^2
\frac{k^jk_i}{k^2}\Bigg(
2\frac{p^ip_j}{p^2}
\left(Q\dvp1(\pvi)+\vp_{0}'\dvp1'(\pvi)\right)
Q\dvp1(\qvi)
\Bigg)\Bigg]
\Bigg\}\,.
\end{align}


%%%%%%%%%%%%%%%%%%%%%%%%%%%%%%%%%%%%
\subsection{Slow Roll Approximation}
\label{sec:slowroll}
%%%%%%%%%%%%%%%%%%%%%%%%%%%%%%%%%%%%


In order to establish the viability of a numerical calculation of the evolution of
second order perturbations from the
Klein-Gordon equation, Chapters~\ref{ch:numericalsystem} and \ref{ch:results} will be limited to
the framework of the slow roll approximation. 
This involves taking
%
\begin{align}
 \vp_{0}'' + \H \vp_{0}' &\simeq 0\,,\quad \\
\frac{\left(\vp_{0}'\right)^2}{2a^2} &\ll \U\,,
\end{align}
%
such that $Q=0$ and $\H^2 = (8\pi G/3) a^2 \U$. 
In Chapter~\ref{ch:introduction} the slow-roll parameter $\varepsilon_H$ was
defined in \eq{eq:epsilonHdefn-intro}. 
In this chapter and the rest of Part~\ref{part:numerical}, a different slow 
roll parameter will be used, denoted by
$\bar{\varepsilon}_H$ and defined in Refs.~\cite{Malik:2006ir} and
\cite{Seery:2005gb}. 
This new parameter is the square-root of $\varepsilon_H$ and is given by
%
\begin{equation}
 \bar{\varepsilon}_H = \sqrt{4\pi G} \frac{\vp_{0}'}{\H} = \sqrt{\varepsilon_H}\,.
\end{equation}
%
The second slow roll parameter is still $\eta_H = \varepsilon_H -
\varepsilon_H'/2\H \varepsilon_H$. Following \Rref{Malik:2006ir} we
will implement the slow roll approximation by keeping terms up to and including
$\mathcal{O}(\bar{\varepsilon}^2_H)$ and terms which are
$\mathcal{O}(\bar{\varepsilon}_H \eta_H)$. 
Within this approximation the second order equation (\ref{eq:SOKG-real-num})
simplifies dramatically, and with the $F$ term reduces to
%
\begin{align}
 \label{eq:KG2-fourier-sr-num}
&\dvp2''(\kvi)+2\H\dvp2'(\kvi)+k^2\dvp2(\kvi)
+\left(a^2
\Upp-{24 \pi G}(\vp_{0}')^2\right)
\dvp2(\kvi) \nonumber\\
%
&+ \int \d^3p\ \d^3q\ \delta^3(\kvi-\pvi-\qvi) \Bigg\{
a^2\left(\Uppp
+ \frac{8\pi G}{\H}\vp_{0}' \Upp\right)
 \dvp1(\pvi)\dvp1(\qvi) \nonumber \\
%
&\qquad +\frac{16\pi G}{\H}a^2
\vp_{0}'\Upp\dvp1(\pvi)\dvp1(\qvi)\Bigg\}
\nonumber \\
%
&+ \frac{8\pi G}{\H}
\int \d^3p\ \d^3q\ \delta^3(\kvi-\pvi-\qvi)  \Bigg\{
%
\frac{8\pi G}{\H}\frac{p_l q^l}{q^2}\vp_{0}'\dvp1'(\pvi)
\dvp1'(\qvi) \nonumber\\
% 
&\qquad+ 2p^2\vp_{0}' \dvp1(\pvi) \dvp1(\qvi)
%
+\vp_{0}'
\Bigg(
\left(\frac{p_lq^l+p^2}{k^2}q^2-\frac{p_lq^l}{2}\right)
\dvp1(\pvi)\dvp1(\qvi) \nonumber \\
% 
&\qquad+\left(\frac{1}{2}-\frac{q^2+p_lq^l}{k^2}\right)
\dvp1'(\pvi)\dvp1'(\qvi)\Bigg)
\Bigg\} \nonumber \\
% 
&=0 \,.
\end{align}
%
The numerical simulation described in Chapter~\ref{ch:numericalsystem} will solve the
slow roll version of the second order equation given above,
\eq{eq:KG2-fourier-sr-num}, together with the complete first
order equation (\ref{eq:fokg}) and background equation
(\ref{eq:KGback-num}). 
% 
% 
% 
% 
% % % % % % % % % % % % % % % % % % % % % % % % % % % % % % % % 
% =========================================================== %
% % % % % % % % % % % % % % % % % % % % % % % % % % % % % % % %
\section{Observable Quantities}
\label{sec:observable-perts}
% % % % % % % % % % % % % % % % % % % % % % % % % % % % % % % % 
% =========================================================== %
% % % % % % % % % % % % % % % % % % % % % % % % % % % % % % % %

Cosmological perturbations at second order are becoming increasingly important now
that
statistical quantities beyond the power spectrum and spectral index are being
investigated. Observations, however, do not tell us anything about the inflaton
field directly. In this section the second order perturbations described above will
be related to observable quantities in order to demonstrate how a numerical
calculation could
be employed in the near future to gain further insight into the nature of the field
that drives inflation.
% 

The temperature fluctuations observed in the CMB can
be directly related to the curvature perturbation $\R$. In
Section~\ref{sec:perts-intro}, $\R$ was defined at first order in terms of $\dvp1$.
When the second order contribution is included the total comoving curvature
perturbation is defined as
% 
\begin{equation}
\label{eq:Rfulldefn-perts}
 \R = \R_1 + \frac{1}{2}\R_2\,.
\end{equation}
% 
The first order term is related to the inflaton perturbation in the flat gauge by
$\R_1
= \H\dvp1/\vp_0'$. The second order part includes terms quadratic in $\dvp1$ and so
in Fourier space requires convolutions. We are interested in the value of $\R$ after
horizon crossing for the calculation of $\Pr$ and a determination of the
non-gaussianity produced during inflation. This allows us to neglect gradient terms
in real space or terms proportional to $k$ in Fourier space.
In this limit the real space expression for $\R_2$ is \cite{Malik:2005cy}
% 
\begin{equation}
 \label{eq:R2real-perts}
\R_2(\eta, x^i) = \frac{\H}{\vp_0'}\dvp2 - 2\frac{\H}{\left(\vp_0'\right)^2}
\dvp1'\dvp1 + \frac{\dvp1^2}{\left(\vp_0'\right)^2}\left(\H\frac{\vp_0''}{\vp_0} 
 - \H' -2\H^2 \right)\,.
\end{equation}
% 
Using the background evolution equation \eqref{eq:KGback-num} and transforming to
Fourier space implies that \eq{eq:R2real-perts} can be written as
% 
\begin{align}
 \R_2(\eta, \kvi) &= \frac{\H}{\vp_0'}\dvp2(\eta, \kvi) 
  + \frac{1}{(2\pi)^3}\int \d^3q\, \d^3p\, \delta(\kvi-\qvi-\pvi) \Bigg\{
\nonumber\\
% 
 &\qquad -2\frac{\H}{\left(\vp_0'\right)^2} \dvp1(\eta, \pvi)\dvp1'(\eta, \qvi) \nonumber\\
%  
&\qquad -\frac{1}{\left(\vp_0'\right)^2} \left(
  2\H^2 \frac{\vp_0'}{\vp_0} + \frac{a^2 \H}{\vp_0}\Uphi + (8\pi G)a^2 \U
 \right)
\dvp1(\eta,\pvi) \dvp1(\eta, \qvi)\Bigg\}\,.
\end{align}
% 
Once the numerical calculation has been carried out at first and second order as
described in Chapter~\ref{ch:numericalsystem} this quantity can be evaluated after
horizon crossing.

In Chapter~\ref{ch:introduction} the non-gaussianity parameter $\fnlloc$ was defined
in terms of $\R$ in \eq{eq:fnllocdefn-intro}. Writing \eq{eq:fnllocdefn-intro} in
Fourier space using \eq{eq:Rfulldefn-perts} implies that
% 
\begin{align}
 \R(\kvi) &= \R_1(\kvi)
  + \frac{3}{5}\fnlloc \Bigg( \frac{1}{(2\pi)^3}\int \d q^3\, \R_1(\qvi)
\R_1(\kvi-\qvi) \\ \nonumber
% 
  &\quad- \left\langle \frac{1}{(2\pi)^3} \int \d q^3\, \R_1(\qvi)
\R_1(\kvi-\qvi) \right\rangle \Bigg) \,,
\end{align}
% 
where $\langle \rangle$ denotes the expectation value.
A good approximation of the local non-gaussianity produced is then given by
% 
\begin{align}
 \label{eq:fnlloc-perts}
\fnlloc &= \frac{5}{6} \R_2(\kvi) \Bigg[ \frac{1}{(2\pi)^3}\int \d q^3\, \R_1(\qvi)
\R_1(\kvi-\qvi) \\ \nonumber
% 
  &\quad- \left\langle \frac{1}{(2\pi)^3} \int \d q^3\, \R_1(\qvi)
\R_1(\kvi-\qvi) \right\rangle \Bigg]^{-1} \,.
\end{align}
% 
Calculating $\dvp2$ and $\R_2$ therefore provides direct insight into the behaviour
and
production of the non-gaussianity parameter $\fnlloc$. 

% 
% 
% 
% 
% % % % % % % % % % % % % % % % % % % % % % % % % % % % % % % % 
% =========================================================== %
% % % % % % % % % % % % % % % % % % % % % % % % % % % % % % % %
\section{Discussion}
\label{sec:disc-perts}
% % % % % % % % % % % % % % % % % % % % % % % % % % % % % % % % 
% =========================================================== %
% % % % % % % % % % % % % % % % % % % % % % % % % % % % % % % %

In this chapter, the equations of motion for a single scalar
field
up to second order in cosmological perturbations have been introduced. The second
order gauge
transformation has been discussed and the transformation components determined for
the uniform curvature gauge.
% 
In Chapter~\ref{ch:introduction} first order classical perturbations were quantised
in the Minkowski spacetime limit and normalised using the Wronskian condition in
\eq{eq:quantcondition-intro}. This constraint also fixes the quantisation for other
orders of the perturbation, including $\dvp2$ \cite{Seery:2008qj}. 

The perturbation equations are better understood in Fourier space, although the cost
of adopting this approach is the need to employ convolution integrals of the first
order
perturbations. When written in Fourier space, the second order Klein-Gordon equation
can
be described entirely in terms of the field perturbations and background quantities. 

\eq{eq:SOKG-real-num}, first derived in \Rref{Malik:2006ir}, is valid on all scales
inside and outside the horizon. When a particular slow roll approximation is made
this equation simplifies to that found in \eq{eq:KG2-fourier-sr-num}. This slow roll
version of the equation will be the central governing equation of the numerical
calculation described in the next chapter.

Finding numerical solutions of \eq{eq:KG2-fourier-sr-num} is the first step towards
solving the full equation \eqref{eq:SOKG-real-num} for a single field and
ultimately the multi-field equation given in \Rref{Malik:2006ir}. Understanding
cosmological perturbations beyond linear order is critical if higher order statisical
effects are to be accurately calculated. Section~\ref{sec:observable-perts}
outlined the connection between $\dvp2$ and observable quantities such as the
comoving curvature perturbation and the non-Gaussianity of the perturbations. Going
beyond single field slow roll models, non-linear effects become more important. In
Chapters~\ref{ch:numericalsystem} and \ref{ch:results} the first step is taken
towards calculating higher order perturbations for these models.
