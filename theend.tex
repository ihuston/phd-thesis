% % % % % % % % % % % % % % % % % % % % % % 
% theend.tex - Ian Huston
% $Id: theend.tex,v 1.5 2009/11/09 13:23:14 ith Exp $
% % % % % % % % % % % % % % % % % % % % % % 
% Redefine CVSRevision for this section
\renewcommand{\CVSrevision}%
{\version$Id: theend.tex,v 1.5 2009/11/09 13:23:14 ith Exp $}

% % % % % % % % % % % % % % % % % % % % % % % % % % % % % % % % 
% =========================================================== %
% % % % % % % % % % % % % % % % % % % % % % % % % % % % % % % %
\chapter{Final Conclusions}
\label{ch:conclusions}
% % % % % % % % % % % % % % % % % % % % % % % % % % % % % % % % 
% =========================================================== %
% % % % % % % % % % % % % % % % % % % % % % % % % % % % % % % %

As the number of cosmological models grows, the need to constrain them is also
increasing. At the same time, the quantity and quality of observational data are
advancing. There now exists the opportunity to go beyond linear statistical analyses
and confront the predictions of models with observational data from the non-linear
regime. In this thesis both analytic and numerical methods have been used to
constrain inflationary models.

The framework used was the Friedmann-Robertson-Walker universe, which was reviewed
in Chapter~\ref{ch:introduction}. During the accelerated expansion of the
inflationary period, quantum fluctuations seeded energy density variations which
gave rise to the diverse structure in the universe. First order perturbation theory
was introduced to describe these fluctuations. Using the Bunch-Davies vacuum initial
conditions, observational predictions for the inflationary scenario were made.

The main observable quantities are the power spectrum of scalar perturbations, its
spectral index, and the ratio of tensor to scalar perturbations. In addition to
these the degree of departure of the perturbations from purely Gaussian statistics
was considered in the local and equilateral cases.


\pagebreak

In Part~\ref{part:dbi} of this thesis, analytic methods were used to constrain
string theory inspired inflationary models. The Dirac-Born-Infeld scenario was
outlined in Chapter~\ref{ch:dbi-intro}. In this model, a D3-brane propagates in a
six-dimensional warped throat. The radial position of the brane from the tip of the
throat assumes the role of the inflaton field. The non-canonical nature of the DBI
action restricts the kinetic energy of this field no matter how steep the potential
which enables an inflationary period of sufficient duration to occur.
This model was widely regarded as a very promising implementation of an inflationary
model in a string-theory context[cite***].

In \Rref{bmpaper}, Baumann and McAllister used the Lyth bound \cite{lyth} to limit
the tensor-scalar ratio based on the conservative assumption that the brane could
not propogate further than the full length of the throat. In Chapter~\ref{ch:dbi},
we showed that this bound can be tightened by applying it over the
portion of the throat which the brane passes through during the directly observable
e-foldings of inflation. Restricting the field variation to be over these,
approximately, four e-foldings constrains $r$ to be less than $10^{-7}$ for standard
parameter values.

The most optimistic estimates of advances in experimental techniques and data
analysis, including foreground reduction, indicate that observations of $r>10^{-4}$
might be achievable [cite CMBPOL etc]. Therefore, this new bound in
\eq{eq:upperbound} immediately rules out the observation of a tensor mode signal
from this model in the foreseeable future.

In addition to this, we also derived a lower bound on $r$ in \eq{eq:lowerbound}.
This depends on observable quantities, namely the scalar spectral index and the
equilateral non-gaussianity. Saturating the WMAP5 observational limit on $\fnleq$
and taking the best fit value for $n_s$, we found that the most conservative lower
limit is $r>0.005$. This is clearly incompatible with the previously derived upper
bound. Therefore, for standard parameter values, the D3-brane DBI scenario is not
viable. 

In Section~\ref{sec:relaxing-dbi}, a phenomenological approach was taken to easing
the upper bound. By considering a DBI action with unspecified field functions,
$f_i$, we showed that the generalised lower and upper bounds can be consistent
if the product of $f_1$ and $f_2$ is large on observable scales.

The discovery of the inconsistent bounds for DBI inflation has had a noticeable
impact on the research community, spurring interest in finding models which evade
these bounds. Many such models have been proposed with varying degrees of success. In
Section~\ref{sec:others-dbi} these were categorised according to whether they
featured single or multiple fields and single or multiple branes. 


One such ...

