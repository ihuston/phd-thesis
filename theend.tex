% % % % % % % % % % % % % % % % % % % % % % 
% theend.tex - Ian Huston
% $Id: theend.tex,v 1.21 2009/11/29 14:46:33 ith Exp $
% % % % % % % % % % % % % % % % % % % % % % 
% Redefine CVSRevision for this section
\renewcommand{\CVSrevision}%
{\version$Id: theend.tex,v 1.21 2009/11/29 14:46:33 ith Exp $}

% % % % % % % % % % % % % % % % % % % % % % % % % % % % % % % % 
% =========================================================== %
% % % % % % % % % % % % % % % % % % % % % % % % % % % % % % % %
\chapter{Conclusion and Discussion}
\label{ch:conclusions}
% % % % % % % % % % % % % % % % % % % % % % % % % % % % % % % % 
% =========================================================== %
% % % % % % % % % % % % % % % % % % % % % % % % % % % % % % % %

As the number of viable cosmological models increases, the need to constrain
them becomes more
important. At the same time, the quantity and quality of observational data continue
to improve.
There now exists the opportunity to go beyond linear statistical analyses
and confront the predictions of models with observational data from the non-linear
regime. In this thesis both analytic and numerical methods have been developed to
constrain inflationary models.

The framework used in this thesis is the Friedmann-Robertson-Walker universe,
reviewed in Chapter~\ref{ch:introduction}. During the accelerated expansion of the
inflationary period, quantum fluctuations seeded energy density variations, which in turn
gave rise to the diverse structure of the present universe. First order
cosmological perturbation theory
is necessary to describe the evolution of these fluctuations. The main observable
quantities can be calculated at horizon crossing by using the Bunch-Davies vacuum
initial conditions. The departure of the perturbations from a purely Gaussian random
field is parametrised by $\fnl$, which is described in two limits, local and
equilateral. 
In addition to canonical actions, we also introduced
non-canonical models, for which the speed of sound of the perturbations plays a
crucial role. When the sound speed is small, the amplitude of $\fnleq$ for these
models is large. 



% DBI
In Part~\ref{part:dbi}, analytic methods were developed to constrain
string theory inspired non-canonical inflationary models. The Dirac-Born-Infeld
scenario was
outlined in Chapter~\ref{ch:dbi-intro}. In this model, a D3-brane propagates in a
six-dimensional warped throat. The radial position of the brane from the tip of the
throat assumes the role of the inflaton field. The non-canonical nature of the DBI
action restricts the kinetic energy of this field no matter how steep the potential.
This allows an inflationary period of sufficient duration to occur.
This model has been widely regarded as a very promising realisation of an
inflationary
model in a string-theory context.

In \Rref{bmpaper}, Baumann \& McAllister used the Lyth bound \cite{lyth} to limit
the tensor-scalar ratio. Their analysis was based on the conservative assumption that the brane
could
not propagate further than the full length of the throat. In Chapter~\ref{ch:dbi},
we showed that this bound can be tightened by applying it over the
portion of the throat through which the brane passes during the directly observable
stage of inflation. Restricting the field variation to be over these
approximately four e-foldings constrains $r$ to be less than $10^{-7}$ for standard
parameter values.

The most optimistic estimates of advances in experimental techniques and data
analysis, including foreground reduction, indicate that observations of $r>10^{-4}$
might be achievable in the future \cite{Baumann:2008aq,vpj}. Therefore, the new bound in
\eq{eq:upperbound} immediately rules out the observation of a tensor mode signal
from this model.

In addition to this, we also derived a lower bound on $r$ in \eq{eq:lowerbound}.
This depends on observable quantities, namely the scalar spectral index and the
equilateral non-Gaussianity. Saturating the WMAP5 observational limit on $\fnleq$
and taking the best fit value for $n_s$, we found that the most conservative lower
limit is $r>0.005$. This is clearly incompatible with the previously derived upper
bound. Therefore, for standard parameter values, the D3-brane DBI scenario is not
viable. Numerical simulations by Peiris \etal in \Rref{Peiris:2007gz} have
demonstrated the lower bound, in the relativistic limit, using the
Hamiltonian flow approach.


In Section~\ref{sec:relaxing-dbi}, a phenomenological approach was taken to easing
the upper bound on the tensor-scalar ratio. By considering a DBI-type action with unspecified field
functions,
$f_i$, we showed that the generalised lower and upper bounds can be consistent
if the product of $f_A$ and $f_B$ is sufficiently large on observable scales. This provides a guide
to the types of models which could evade the inconsistency of the bounds on $r$. For more general
models with a non-canonical action, a bound on $r$ which relates the geometry of the throat, the
number of e-foldings of observable inflation, and the derivatives of the action has been derived in
\eq{eq:LHbound}. This bound, although it does not in general relate to observational quantities can
be used when the details of a particular physical model are known.

The discovery of the incompatible bounds on $r$ for DBI inflation has had a noticeable
impact on the research community, spurring interest in finding models which evade
these bounds. Many such models have been proposed with varying degrees of success. In
Section~\ref{sec:others-dbi} these were categorised according to whether they
featured single or multiple fields, and single or multiple branes. Some of these models are still
constrained by the bounds on $r$ but not to the same extent as the standard DBI scenario. For
example the parameter space of the models with wrapped brane configurations is still extremely
limited by the observational values from WMAP5 \cite{Alabidi:2008ej}. For other models an analysis
in terms of the bounds derived in this thesis has yet to be undertaken. As the observational limits
on $\fnleq$ and $r$ continue to improve, an important step in ensuring the validity of DBI based
models is to check whether equivalent bounds to those derived here exist, and whether they can be
met for any significant proportion of the parameter space.

% Multi
General actions which relax the bounds on $r$, were derived in
Chapter~\ref{ch:multibrane}. 
We found a class of actions similar in form to that of the DBI model. 
However, instead of a square-root in the kinetic term, the index of the main term of these
actions depends on the constant of proportionality between $\Lambda$ and the sound
speed of inflaton fluctuations.
The upper bound on $r$ can be derived for these general actions and when the index
is below the critical value of $1/2$ (corresponding to the standard DBI scenario), this bound is
significantly relaxed.  
When new models are proposed in the future, our phenomenological derivation of this family of
actions will allow those models for which the bound on $r$ is less stringent to be easily
identified.

One such model is the single field, multi-coincident brane scenario of
Thomas \& Ward \cite{thomasward}. When $n$ D3-branes propagate in a throat, the
non-Abelian interactions between the branes result in major departures from the
single brane case. This is in contrast to the non-interacting branes model, in which
the total action is simply the sum of copies of the single brane action.

In the limit of a large number of branes being coincident, the effective action is
similar to $n$ times the single brane model, with the addition of a fuzzy potential
term. 
% 
Indeed this model is of the type considered phenomenologically in Section~\ref{sec:relaxing-dbi} and
the bounds derived in that section can be applied. 
% 
These bounds on $r$ can be somewhat relaxed when this potential is large,
but the model is still strongly constrained by current observations. For an $AdS_5$
throat, standard parameter choices limit the number of allowed branes to be
less than 150, at which point the assumption of arbitrarily large $n$ becomes
questionable.

More promising is the finite $n$ limit of the coincident brane model. The
non-Abelian nature of the interactions leads, in this case, to a recursive relation for
the $n$-brane action in terms of the $n=2$ one. In
Section~\ref{sec:finiten-multi}, we showed that the action for finite $n$ is one of
the class of bound-evading actions described above. This identification is possible
because the last term of the recursive sum dominates in the relativistic limit. This
approximation is valid at least when $n<10$ and the backreaction of the multiple
branes is kept well under control for this range of $n$.

Although the bounds on $r$ are eased
for this model, we showed that observations strongly constrain the possibility of
an observable tensor signal being generated. If an observable tensor-scalar ratio is
considered to be $r>10^{-4}$, then only the two or three brane cases are capable of
producing such a signal. This bound on $n$ depends on the WMAP5 limit on $\fnleq$.
If, as expected, the observational limits on $\fnleq$ tighten considerably in the
future, the possibility of an observable tensor signal from the multi-coincident
brane model could be ruled out.

On the other hand, the choice of $r>10^{-4}$ as the threshold of an observable signal is very
optimistic. If foreground removal techniques and the signal-to-noise ratios of future experiments
cannot reach this threshold, and instead reach $r>10^{-3}$, no number of branes will be able to
produce an observable tensor signal when combined with the current limits on the non-Gaussianity.
There will then be little possibility of a distinguishing
observational signature for these coincident brane models.

% Numerical
In Part~\ref{part:numerical} of this thesis, numerical methods were used to test inflationary
models up to second order in cosmological perturbation theory. 
% 
% 
The Klein-Gordon equation at second order was derived in \Rref{Malik:2006ir} for the
multi-field case. 
In Chapter~\ref{ch:perts}, second order gauge transformations were outlined
and the Klein-Gordon equation reproduced for a single scalar field model. In contrast
to the $\Delta \N$ approach, this equation is valid on all scales, both inside and
outside the horizon. 


In Fourier space, the second order Klein-Gordon equation \eqref{eq:SOKG-real-num} contains a
convolution term of the first order scalar field perturbation. For this first
demonstration of the numerical system, a slow roll approximation of the full
equation was used. 


Calculating the second order scalar field perturbations provides the possibility of
a unique insight into the generation and evolution of non-linear contributions to
the scalar curvature perturbation. 
% 
One advantage of using the inflaton field equations is that we can directly
investigate how the perturbations are generated.
If, instead, we integrated the evolution
equation for a derived observable quantity, there would be a degree of
separation from the physical origins of this process. Indeed, there is, as
yet,
no known evolution equation for the main observable quantity, the comoving
curvature perturbation, at second order. 
% 
Using cosmological perturbation theory also provides control over the calculation.
The domain of applicability of the perturbative expansion is well defined and the
resultant equations are certain to be valid in this domain.

% 
The main observable quantity is not however the second order scalar perturbation, but rather the
departure from Gaussianity in the CMB temperature map, parametrised by the amplitude of the
bispectrum of the perturbations. In Section~\ref{sec:observable-perts} we outlined how $\fnl$ could
be calculated from the numerically found $\dvp2$ both for the local type and more generally using
the bispectrum of the uniform density curvature perturbation. As the observational limits on $\fnl$
are tightened over the course of the remaining WMAP releases and future Planck data, the importance
of comparing the predictions for $\fnl$ of inflationary models with the observed values will only
increase. In this thesis we have not computed $\fnl$ for the models we have considered, but this is
an important future step that will be undertaken. 


The long term aim of this continuing project is
to analyse multi-field, non-slow roll models, in which non-Gaussian effects are
expected to play an important role.
% 
As a step towards this goal, we described the implementation of a numerical
calculation of the single field, slow roll, second order equation in
Chapter~\ref{ch:numericalsystem}. The construction and evaluation of the convolved
source term in \eq{eq:KG2-src-sr-aterms} proved to be the most numerically complex
step required. 

To allow a numerical calculation, an energy scale cutoff must be implemented. We used
a sharp cutoff at small wavenumbers below which the perturbations were taken to be
identically zero. Another cutoff at small scales (or equivalently large wavenumbers) was dictated
by practical considerations of calculation size and computation time. 

We defined, in \eq{eq:AtoD-num}, four $\theta$ dependent integrals
into which the convolution term can be decomposed. By comparison with an analytic
solution for a particular smooth choice of the first order perturbation, an estimate
was made of the relative error present in the integration of each term. The
number of discrete wavenumber values, their spacing, and the number of discrete
$\theta$ values were chosen to minimise the error in one of the integrals. From
these parameter values, three
different finite ranges of discrete values of the wavenumber were defined.
These all contain the WMAP pivot scale at $\kwmap=0.002\Mpc^{-1}$ and cover the
WMAP observed scales to varying degrees.
% 
Despite the $k$ ranges having being chosen to minimise the relative error in the
integral of only
one of the $\theta$ dependent terms, the integrals of the three other terms also
display small relative errors for these ranges. 
% 
The analytic solutions which have been found will form an
important part of the verification of any future modifications to the
numerical code. 



The execution of the code is in four stages, building on previous calculations of
first order perturbations in Refs.~\cite{Martin:2006rs, Ringeval:2007am,
Salopek:1988qh}. To begin, the background equations are solved and the end time of
inflation is fixed. The initial conditions for the first order perturbation can then
be set and solutions found for the evolution equations.
Despite the large volume of calculations required at each
time step, the easily parallelisable nature of the source term calculations allows
the run time of the third stage to be reduced significantly. 
The final stage of the calculation uses the source term results to solve the second
order perturbation equations.
% 

The initial conditions for the second order perturbations are taken to be $\dvp2=0$ and
$\dN{\dvp2}=0$ as described in Section~\ref{sec:initconds-num}. For this choice of initial
conditions the homogeneous part of the solution of the second order equation is zero at all times.
As the perturbations are supposed to become more Gaussian the further back in time they are
considered, in the limit of the far past the second order perturbations should be zero. It remains
to be investigated whether the choice of initialisation time is sufficiently far in the past for
this assumption to be accurate. At first order it is known that the perturbations are well
 approximated by the Bunch-Davies vacuum initial conditions even just a few e-foldings before
horizon crossing. However, this choice of initialisation time may not be the most appropriate for
the second order perturbations. In future work it would be worth considering whether the analytic
Green's function solution for $\dvp2$ at very early times could be integrated until the numerical
initialisation time and used as the initial condition for the perturbation. 


To test the code, four different, large field, monomial potentials were used. These were the
standard quadratic and quartic potentials, a fractional index potential derived from the monodromy
string inflation model and a toy model in which inflation is stopped by hand and a blue spectrum is
produced.
Each potential depends on a single parameter, which was fixed by comparing the resultant scalar
curvature power spectrum with the WMAP5 normalisation. The slow roll
approximation can be applied to all four potentials. These potentials are not meant to represent an
exhaustive survey of single field slow roll models but are sufficiently different to exhibit
different power spectra and second order source terms.

We presented the results for each potential in Chapter~\ref{ch:results}. The first
order results match those in Refs.~\cite{Martin:2006rs, Ringeval:2007am, Salopek:1988qh}. The
results of the source term calculation show that before
horizon crossing, the source term amplitude decays rapidly for all four potentials.
The amplitude changes less after horizon crossing, until later times when it increases
as the slow roll approximation breaks down.

The four different potentials have similar amplitudes before horizon crossing but reach different
values after horizon crossing. The differences in the slow roll parameters for each potential are
compared with the source term values in Appendix~\ref{sec:apx-srcdisc}. The slow roll parameters
do not appear to be directly related to the amplitudes of the source terms, at least in a linear
fashion.

The choice of wavenumber range affects the amplitude of the source term, as
expected, due to the implementation of a sharp cutoff at large scales. This
dependence is only apparent, however, before horizon crossing. For a particular
range,  the magnitude of the source term decreases as wavenumber increases. However,
the ratio of the source term to the other terms in the Klein-Gordon equation increases with
wavenumber. 

As expected, the amplitude of the second order scalar perturbations is much smaller
than that of the first order ones. After the generation of the second order
perturbations at early times, their evolution is that of a damped harmonic
oscillator similar to the first order evolution.

We have shown that the magnitude of the source term can be important throughout the
full evolution and that it is not sufficient to calculate this term only for
modes either entirely inside or outside the horizon, \iec taking a short or long
wavelength approximation respectively. We have been able to access both of these
regimes, by solving the evolution equations of the
inflaton field perturbation. This is in contrast to other approaches which could
have been taken, for example using the $\Delta\N$
formalism, which is only applicable in the large scale limit.


%New stuff
The construction of any numerical code involves a considerable commitment of time
and resources so it is important to understand why such an endeavour has been
undertaken. 
The numerical calculation of first order cosmological perturbations is an invaluable
part of the cosmologist's toolkit. It allows analytic predictions of inflationary
models to be confirmed where these exist, but also generates predictions where no
analytic solution is possible. Another important use is to test predictions based
on the slow roll approximation  against the full evolution equations. 
% 
We have taken the first step towards upgrading this standard numerical calculation to
include second order scalar perturbations. 
% 
The ability to check the predictions of inflationary models at second order will
be a powerful tool to constrain these models and check the consistency of any
analytic assumptions that have been made. Solving the inflaton field equations
provides the most direct access to the non-linear effects that the increase in
available statistics have made observationally important.
% 

We have presented the
first numerical calculation of the Klein-Gordon equation for second order scalar
perturbations which was derived in \Rref{Malik:2006ir}. Although we have restricted
ourselves in this thesis to the single field, slow roll version of the second order
equation,
the expertise gained and the lessons learned in the development of the numerical
system will be of significant assistance when the next steps towards a full
multi-field calculation are taken. 
% 


In the past, a numerical calculation on the scale we have achieved would have been
the preserve of dedicated super-computing facilities. We have demonstrated that a
calculation of this scope is now possible using relatively modest local resources.
If the computational power available increases, the practical limits on the
resolution and extent of the $k$ ranges will ease. Further improvements in the
efficiency of the code will also loosen these constraints.
% 

Another consideration in the development of a numerical system is the possibility of
code re-use. One of our future goals is to develop our code into a numerical toolkit which
can be applied to a variety of physical situations.
The equations of motion of the inflaton scalar field are similar in form
to the governing equations of other important cosmological phenomena. Therefore, it
should be possible to adapt the numerical system we have constructed and apply it to
other areas of interest. The form of the second order equation and source term are similar to those
applicable in the evolution of tensor perturbations and the
generation of vorticity in the early universe. The flexibility of the numerical system we have
developed will be a positive factor in any attempt to apply our code to these physical systems. 


The numerical calculation described is the first step towards a system capable of
handling the multi-field, non-slow roll models for which non-linear contributions
are important. In Section~\ref{sec:next-res}, the next steps towards this goal were
outlined. 
% 
Continuing our work by calculating the second order perturbations for both the
non-slow roll, single field case and the slow
roll, multi-field case will pave the way for the eventual calculation of the
non-slow roll, multi-field equations.
% 

The full single field, non-slow roll, second order equation can be treated using the
method already described. 
Three more $\theta$ dependent terms are necessary to
compute the full convolution integral in this case. It will be important to find
analytic solutions for these three extra terms, as already done for the terms in the
slow roll case, in
order to gauge the effectiveness of the extended code. When the extension to non slow-roll models
is complete, it will be possible to investigate models with a step or other feature in their
potential. These models can exhibit large amounts of non-Gaussianity produced around the feature
with a shape dependence that is more general than that of the local and equilateral forms. 

The Klein-Gordon equation for the multi-field case introduces further complexity. We plan to
expand the numerical system to encompass two or three scalar fields. The differences
between single and multi-field models are already apparent for these cases. We have
presented the slow roll source term equation for multiple fields in vector notation.
The definitions of the four $\theta$ dependent terms used in the single
field, slow roll model were also extended to the multi-field case. Beyond the slow
roll approximation, the full multi-field equation should be treatable in a similar
manner to the single field case, by introducing further $\theta$ dependent terms.



To conclude, it is worth reiterating our opening remarks. Cosmology has moved from
being a theorists' playground to a genuine scientific discipline. Inflationary models can
now be strongly tested by observations and the next generation of experiments will
place even tighter limits on the viable parameter space of such models. In this thesis, analytic
arguments have constrained string theory inspired inflationary models and numerical
methods have paved the way to calculating higher order cosmological perturbations.