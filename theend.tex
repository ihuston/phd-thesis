% % % % % % % % % % % % % % % % % % % % % % 
% theend.tex - Ian Huston
% $Id: theend.tex,v 1.17 2009/11/22 18:32:41 ith Exp $
% % % % % % % % % % % % % % % % % % % % % % 
% Redefine CVSRevision for this section
\renewcommand{\CVSrevision}%
{\version$Id: theend.tex,v 1.17 2009/11/22 18:32:41 ith Exp $}

% % % % % % % % % % % % % % % % % % % % % % % % % % % % % % % % 
% =========================================================== %
% % % % % % % % % % % % % % % % % % % % % % % % % % % % % % % %
\chapter{Conclusion and Discussion}
\label{ch:conclusions}
% % % % % % % % % % % % % % % % % % % % % % % % % % % % % % % % 
% =========================================================== %
% % % % % % % % % % % % % % % % % % % % % % % % % % % % % % % %

As the number of viable cosmological models increases, the need to constrain them becomes more
important. At the same time, the quantity and quality of observational data continues to improve.
There now exists the opportunity to go beyond linear statistical analyses
and confront the predictions of models with observational data from the non-linear
regime. In this thesis both analytic and numerical methods have been developed to
constrain inflationary models.

The framework used in this thesis is the Friedmann-Robertson-Walker universe,
reviewed in Chapter~\ref{ch:introduction}. During the accelerated expansion of the
inflationary period, quantum fluctuations seeded energy density variations, which in turn
gave rise to the diverse structure of the present universe. First order
cosmological perturbation theory
is necessary to describe the evolution of these fluctuations. The main observable
quantities can be calculated at horizon crossing by using the Bunch-Davies vacuum
initial conditions. The departure of the perturbations from a purely Gaussian random
field is parametrised by $\fnl$, which is described in two limits, local and
equilateral. 
In addition to canonical actions, we also introduced
non-canonical models, for which the speed of sound of the perturbations plays a
crucial role. When the sound speed is small, the amplitude of $\fnleq$ for these
models is large. 



% DBI
In Part~\ref{part:dbi}, analytic methods were developed to constrain
string theory inspired non-canonical inflationary models. The Dirac-Born-Infeld
scenario was
outlined in Chapter~\ref{ch:dbi-intro}. In this model, a D3-brane propagates in a
six-dimensional warped throat. The radial position of the brane from the tip of the
throat assumes the role of the inflaton field. The non-canonical nature of the DBI
action restricts the kinetic energy of this field no matter how steep the potential.
This allows an inflationary period of sufficient duration to occur.
This model was widely regarded as a very promising realisation of an inflationary
model in a string-theory context.

In \Rref{bmpaper}, Baumann and McAllister used the Lyth bound \cite{lyth} to limit
the tensor-scalar ratio. Their analysis was based on the conservative assumption that the brane
could
not propagate further than the full length of the throat. In Chapter~\ref{ch:dbi},
we showed that this bound can be tightened by applying it over the
portion of the throat through which the brane passes during the directly observable
stage of inflation. Restricting the field variation to be over these
approximately four e-foldings constrains $r$ to be less than $10^{-7}$ for standard
parameter values.

The most optimistic estimates of advances in experimental techniques and data
analysis, including foreground reduction, indicate that observations of $r>10^{-4}$
might be achievable in the future \cite{Baumann:2008aq,vpj}. Therefore, the new bound in
\eq{eq:upperbound} immediately rules out the observation of a tensor mode signal
from this model.

In addition to this, we also derived a lower bound on $r$ in \eq{eq:lowerbound}.
This depends on observable quantities, namely the scalar spectral index and the
equilateral non-gaussianity. Saturating the WMAP5 observational limit on $\fnleq$
and taking the best fit value for $n_s$, we found that the most conservative lower
limit is $r>0.005$. This is clearly incompatible with the previously derived upper
bound. Therefore, for standard parameter values, the D3-brane DBI scenario is not
viable. Numerical simulations by Peiris \etal in \Rref{Peiris:2007gz} have
demonstrated the lower bound, in the relativistic limit, using the
Hamiltonian flow approach.

The discovery of these incompatible bounds for DBI inflation has had a noticeable
impact on the research community, spurring interest in finding models which evade
these bounds. Many such models have been proposed with varying degrees of success. In
Section~\ref{sec:others-dbi} these were categorised according to whether they
featured single or multiple fields, and single or multiple branes. 

In Section~\ref{sec:relaxing-dbi}, a phenomenological approach was taken to easing
the upper bound on the tensor-scalar ratio. By considering a DBI-type action with unspecified field
functions,
$f_i$, we showed that the generalised lower and upper bounds can be consistent
if the product of $f_1$ and $f_2$ is sufficiently large on observable scales.


% Multi
More general actions, which relax the bounds on $r$, were derived in
Chapter~\ref{ch:multibrane}. 
We found a class of actions similar in form to that of the DBI model. 
However, instead of a square-root in the kinetic term, the index of the main term of these
actions depends on the constant of proportionality between $\Lambda$ and the sound
speed of inflaton fluctuations.
The upper bound on $r$ can be derived for these general actions and when the index
is below the critical value of $1/2$ (corresponding to the standard DBI scenario), this bound is
significantly relaxed.  
When new models are proposed in the future, our phenomenological derivation of this family of
actions will allow those models for which the bound on $r$ is less stringent to be easily
identified.

Once such model is the single field, coincident brane scenario of
Thomas \& Ward \cite{thomasward}. When $n$ D3-branes propagate in a throat, the
non-Abelian interactions between the branes result in major departures from the
single brane case. This is in contrast to the non-interacting branes model, in which
the total action is simply the sum of copies of the single brane action.

In the limit of a large number of branes being coincident, the effective action is
similar to $n$ times the single brane model, with the addition of a fuzzy potential
term. The bounds on $r$ can be somewhat relaxed when this potential is large,
but the model is still strongly constrained by current observations. For an $AdS_5$
throat, standard parameter choices limit the number of allowed branes to be
less than 150, at which point the assumption of arbitrarily large $n$ becomes
questionable.

More promising is the finite $n$ limit of the coincident brane model. The
non-Abelian nature of the interactions leads, in this case, to a recursive relation for
the $n$-brane action in terms of the $n=2$ one. In
Section~\ref{sec:finiten-multi} we showed that the action for finite $n$ is one of
the class of bound-evading actions described above. This identification is possible
because the last term of the recursive sum dominates in the relativistic limit. This
approximation is valid at least when $n<10$ and the backreaction of the multiple
branes is kept well under control for this range of $n$.

Although the bounds on $r$ are eased
for this model, we showed that observations strongly constrain the possibility of
an observable tensor signal being generated. If an observable tensor-scalar ratio is
considered to be $r>10^{-4}$ then only the two or three brane cases are capable of
producing such a signal. This bound on $n$ depends on the WMAP5 limit on $\fnleq$.
If, as expected, the observational limits on $\fnleq$ tighten considerably in the
future, the possibility of an observable tensor signal from the multi-coincident
brane model could be ruled out.

% Numerical
In Part~\ref{part:numerical} of this thesis, numerical methods were used to test inflationary
models up to second order in cosmological perturbation theory. 
% 
% 
The Klein-Gordon equation at second order was derived in \Rref{Malik:2006ir} for the
multi-field case. 
In Chapter~\ref{ch:perts}, second order gauge transformations were outlined
and the Klein-Gordon equation reproduced for a single scalar field model. In contrast
to the $\Delta \N$ approach, this equation is valid on all scales, both inside and
outside the horizon. 


In Fourier space, the second order Klein-Gordon equation \eqref{eq:SOKG-real-num} contains a
convolution term of the first order scalar field perturbation. For this first
demonstration of the numerical system, a slow roll approximation of the full
equation was used. 

Calculating the second order scalar field perturbations provides the possibility of
a unique insight into the generation and evolution of non-linear contributions to
the scalar curvature perturbation. The long term aim of this continuing project is
to analyse multi-field, non-slow roll models, in which non-Gaussian effects are
expected to play an important role.

One advantage of using the inflaton field equations is that we can directly
investigate the generation of the perturbations. Integrating the evolution
equation for a derived observable quantity would, instead, add a degree of
separation from the physical origins of the perturbation. Indeed, there is, as yet,
no known evolution equation for one of the main observable quantities, the comoving
curvature perturbation, at second order. 


As a step towards this goal, we described the implementation of a numerical
calculation of the single field, slow roll, second order equation in
Chapter~\ref{ch:numericalsystem}. The construction and evaluation of the convolved
source term in \eq{eq:KG2-src-sr-aterms} proved to be the most numerically complex
step required. 

To allow a numerical calculation, an energy scale cutoff must be implemented. We used
a sharp cutoff at small wavenumbers below which the perturbations were taken to be
identically zero. Another cutoff at small scales (or equivalently large wavenumbers) was dictated
by practical considerations of calculation size and computation time. 

We defined, in \eq{eq:AtoD-num}, four $\theta$ dependent integrals
into which the convolution term can be decomposed. By comparison with an analytic
solution for a particular smooth choice of the first order perturbation, an estimate
was made of the relative error present in the integration of each term. The
number of discrete wavenumber values, their spacing, and the number of discrete
$\theta$ values were chosen to minimise the error in one of these. From these values, three
different finite ranges of discrete values of the wavenumber were defined.
These all contain the WMAP pivot scale at $\kwmap=0.002\Mpc^{-1}$ and cover the
WMAP observed scales to varying degrees.
% 
Despite the $k$ ranges having being chosen to minimise the relative error in only
one of the $\theta$ dependent terms, the three other terms also display
small relative errors for these ranges. 
% 
The analytic solutions which have been found will form  an
important part of the testing regime for any future modifications to the numerical
code. 



The execution of the code is in four stages, building on previous calculations of
first order perturbations in Refs.~\cite{Martin:2006rs, Ringeval:2007am,
Salopek:1988qh}. First the background equations are solved and the end time of
inflation is fixed. The initial conditions for the first order perturbation can then
be set and solutions found for the evolution equations.
Despite the large volume of calculations required at each
time step, the easily parallelisable nature of the source term calculations allows
the run time of the third stage to be reduced significantly. 
The final stage of the calculation uses the source term results to solve the second
order perturbation equations.


To test the code, four different, large field, monomial potentials were used. Each
potential depends on a single parameter, which was fixed by comparing the resultant scalar
curvature power spectrum with the WMAP5 normalisation. The slow roll
approximation can be applied to all four potentials.

We presented the results for each potential in Chapter~\ref{ch:results}. The first
order results match those in Refs.~\cite{Martin:2006rs, Ringeval:2007am, Salopek:1988qh}. The
results of the source term calculation show that before
horizon crossing, the source term amplitude decays rapidly for all four potentials.
The amplitude changes less after horizon crossing, until later times when it increases
as the slow roll approximation breaks down.

The choice of wavenumber range affects the amplitude of the source term, as
expected, due to the implementation of a sharp cutoff at large scales. This
dependence is only apparent, however, before horizon crossing. For a particular
range,  the magnitude of the source term decreases as wavenumber increases. However,
the ratio of the source term to the other terms in the Klein-Gordon equation increases with
wavenumber. 

As expected, the amplitude of the second order scalar perturbations is much smaller
than that of the first order ones. After the generation of the second order
perturbations at early times, their evolution is that of a damped harmonic
oscillator similar to the first order evolution.

We have shown that the magnitude of the source term can be important throughout the
full evolution and that it is not sufficient to only calculate this term for
modes either entirely inside or outside the horizon, \iec taking a short or long
wavelength approximation respectively. By solving the evolution equations of the
inflaton field perturbation we have been able to access both of these
regimes. This is in contrast to other approaches, particularly the $\Delta\N$
formalism, which is only applicable in the large scale limit.


%New stuff
The construction of any numerical code involves a considerable commitment of time
and resources so it is important to understand why such an endeavour has been
undertaken. 
The numerical calculation of first order cosmological perturbations is an invaluable
part of the cosmologist's toolkit. It allows analytic predictions of inflationary
models to be confirmed where these exist, but also generates predictions where no
analytic solution is possible. Another important use is to test predictions based
on the slow roll approximation  against the full evolution equations. 

We have taken the first step towards upgrading this standard numerical calculation to
include second order scalar perturbations. 
% 
The ability to check the predictions of inflationary models at second order will
be a powerful tool to constrain these models and check the consistency of any
analytic assumptions that have been made. 
% 
We have presented the
first numerical calculation of the Klein-Gordon equation for second order scalar
perturbations which was derived in \Rref{Malik:2006ir}. Although we have restricted
ourselves in this thesis to the single field, slow roll version of the second order
equation,
the expertise gained and the lessons learned in the development of the numerical
system will be of significant assistance when the next steps towards a full
multi-field calculation are taken. 
% 


In the past, a numerical calculation on the scale we have achieved would have been
the preserve of dedicated supercomputing facilities. We have demonstrated that a
calculation of this scope is now possible using relatively modest local resources.
Looking forward, as the computational power available continues to improve, the
second order numerical calculation may become as straightforward to utilise as the
first order one is now. 

The equations of motion of the inflaton scalar field are similar in form to the
governing equations of other important cosmological phenomena. Therefore, it should
be possible to adapt the numerical system we have constructed and apply it to other
areas of interest, such as tensor perturbations and vorticity generation. 


The numerical calculation described is the first step towards a system capable of
handling the multi-field, non-slow roll models, for which non-linear contributions
are important. In Section~\ref{sec:next-res}, the next steps towards this goal were
outlined. 
% 
Going forward with both the non-slow roll, single field calculation and the slow
roll, multi-field calculation will pave the way for calculating the
non-slow roll, multi-field equations.
% 

The full single field, non-slow roll equation can be treated using the
method already described. 
Three more $\theta$ dependent terms are necessary to
compute the full convolution integral in this case. It will be important to find
analytic solutions for these three extra terms, as done in the slow roll case, in
order to gauge the effectiveness of the extended code.

The Klein-Gordon equation for the multi-field case introduces further complexity. We plan to
expand the numerical system to encompass two or three scalar fields. The differences
between single and multi-field models are already apparent for these cases. We have
presented the slow roll source term equation for multiple fields in vector notation.
The definitions of the four $\theta$ dependent terms used in the single
field slow roll model were also extended to the multi-field case. Beyond the slow
roll approximation, the full multi-field equation should be treatable in a similar
manner to the single field case, by introducing further $\theta$ dependent terms.



To conclude, it is worth reiterating our opening remarks. Cosmology has moved from
being a theorists' playground to a genuine scientific discipline. Inflationary models can
now be strongly tested by observations and the next generation of experiments will
place even tighter limits on the viable parameter space of such models. In this thesis, analytic
arguments have constrained string theory inspired inflationary models and numerical
methods have paved the way to calculating higher order cosmological perturbations.