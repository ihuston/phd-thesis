% % % % % % % % % % % % % % % % % % % % % % 
% theend.tex - Ian Huston
% $Id: theend.tex,v 1.9 2009/11/09 19:38:46 ith Exp $
% % % % % % % % % % % % % % % % % % % % % % 
% Redefine CVSRevision for this section
\renewcommand{\CVSrevision}%
{\version$Id: theend.tex,v 1.9 2009/11/09 19:38:46 ith Exp $}

% % % % % % % % % % % % % % % % % % % % % % % % % % % % % % % % 
% =========================================================== %
% % % % % % % % % % % % % % % % % % % % % % % % % % % % % % % %
\chapter{Final Conclusions}
\label{ch:conclusions}
% % % % % % % % % % % % % % % % % % % % % % % % % % % % % % % % 
% =========================================================== %
% % % % % % % % % % % % % % % % % % % % % % % % % % % % % % % %

As the number of cosmological models grows, the need to constrain them is also
increasing. At the same time, the quantity and quality of observational data are
advancing. There now exists the opportunity to go beyond linear statistical analyses
and confront the predictions of models with observational data from the non-linear
regime. In this thesis both analytic and numerical methods have been used to
constrain inflationary models.

The framework used was the Friedmann-Robertson-Walker universe, which was reviewed
in Chapter~\ref{ch:introduction}. During the accelerated expansion of the
inflationary period, quantum fluctuations seeded energy density variations which
gave rise to the diverse structure in the universe. First order perturbation theory
was introduced to describe these fluctuations. Using the Bunch-Davies vacuum initial
conditions, observational predictions for the inflationary scenario were made.

The main observable quantities are the power spectrum of scalar perturbations, its
spectral index, and the ratio of tensor to scalar perturbations. In addition to
these the degree of departure of the perturbations from purely Gaussian statistics
was considered in the local and equilateral cases.


\pagebreak

In Part~\ref{part:dbi} of this thesis, analytic methods were used to constrain
string theory inspired inflationary models. The Dirac-Born-Infeld scenario was
outlined in Chapter~\ref{ch:dbi-intro}. In this model, a D3-brane propagates in a
six-dimensional warped throat. The radial position of the brane from the tip of the
throat assumes the role of the inflaton field. The non-canonical nature of the DBI
action restricts the kinetic energy of this field no matter how steep the potential
which enables an inflationary period of sufficient duration to occur.
This model was widely regarded as a very promising implementation of an inflationary
model in a string-theory context.

In \Rref{bmpaper}, Baumann and McAllister used the Lyth bound \cite{lyth} to limit
the tensor-scalar ratio based on the conservative assumption that the brane could
not propogate further than the full length of the throat. In Chapter~\ref{ch:dbi},
we showed that this bound can be tightened by applying it over the
portion of the throat which the brane passes through during the directly observable
e-foldings of inflation. Restricting the field variation to be over these,
approximately, four e-foldings constrains $r$ to be less than $10^{-7}$ for standard
parameter values.

The most optimistic estimates of advances in experimental techniques and data
analysis, including foreground reduction, indicate that observations of $r>10^{-4}$
might be achievable \cite{Baumann:2008aq,vpj}. Therefore, the new bound in
\eq{eq:upperbound} immediately rules out the observation of a tensor mode signal
from this model in the foreseeable future.

In addition to this, we also derived a lower bound on $r$ in \eq{eq:lowerbound}.
This depends on observable quantities, namely the scalar spectral index and the
equilateral non-gaussianity. Saturating the WMAP5 observational limit on $\fnleq$
and taking the best fit value for $n_s$, we found that the most conservative lower
limit is $r>0.005$. This is clearly incompatible with the previously derived upper
bound. Therefore, for standard parameter values, the D3-brane DBI scenario is not
viable. Numerical simulations by Peiris \etal in \Rref{Peiris:2007gz} have
demonstrated the lower bound, in the relativistic limit, using the
Hamiltonian flow approach.

The discovery of the inconsistent bounds for DBI inflation has had a noticeable
impact on the research community, spurring interest in finding models which evade
these bounds. Many such models have been proposed with varying degrees of success. In
Section~\ref{sec:others-dbi} these were categorised according to whether they
featured single or multiple fields and single or multiple branes. 

In Section~\ref{sec:relaxing-dbi}, a phenomenological approach was taken to easing
the upper bound. By considering a DBI action with unspecified field functions,
$f_i$, we showed that the generalised lower and upper bounds can be consistent
if the product of $f_1$ and $f_2$ is large on observable scales.


% Multi
More general actions, which relax the bounds on $r$, were derived in
Chapter~\ref{ch:multibrane}. By retaining the proportionality of $\Lambda$ and
$c_s^{-2}$, which is present in the DBI model, and requiring that $\fnleq$ is large,
we found a class of actions similar in form to DBI. 
Instead of a square-root in the kinetic term, the index of the main term of these
actions depends on the constant of proportionality between $\Lambda$ and the sound
speed.
The upper bound on $r$ can be derived for these general actions and when the index
is below the critical value of $1/2$ of the standard DBI scenario, this bound is
significantly relaxed. When new models are proposed, the
phenomenological description of this family of actions allows easy identification of
those models for which the bound on $r$ is less stringent. 

One proposed, multi brane, single field model is the coincident brane scenario of
Thomas \& Ward \cite{thomasward}. When $n$ D3-branes propogate in a throat, the
non-Abelian interactions between the branes result in major departures from the
single brane case. This is in contrast to the non-interacting branes model in which
the total action is simply the sum of copies of the single brane action.

In the limit of a large number of branes being coincident, the effective action is
similar to $n$ times the single brane one with the addition of a fuzzy potential
term. The bounds on $r$ can be somewhat relaxed when this potential term is large
but this model is still strongly constrained by current observations. For an $AdS_5$
throat, standard parameter choices limit the number of branes in this case to being
less than 150, at which point the assumption of arbitrarily large $n$ is
questionable.

More promising is the finite $n$ limit of the coincident brane model. The
non-Abelian nature of the interactions leads, in this case, to a recursive definition
of the $n$-brane action in terms of the $n=2$ one. In
Section~\ref{sec:finiten-multi} we showed that the action for finite $n$ is one of
the class of bound-evading actions described above. This identification is possible
because the last term of the recursive sum dominates in the relativistic limit. This
approximation is valid at least when $n<10$ and the backreaction of the multiple
branes is kept well under control for this range of $n$.

Although the bounds on $r$ are eased
for this model, we showed that observations strongly constrain the possibility of
production of an observable tensor signal. If an observable tensor-scalar ratio is
considered to be $r>10^{-4}$ then only the two or three brane cases are capable of
producing such a signal. This bound on $n$ depends on the WMAP5 limit on $\fnleq$.
If, as expected, the observational limits on $\fnleq$ tighten considerably in the
future, the possibility of an observable tensor signal from the multi coincident
brane model could be ruled out.

\pagebreak
% Numerical
In Part~\ref{part:numerical}, numerical methods were used to test inflationary
models up to second order in cosmological perturbation theory. The Klein-Gordon
equation at second order was derived in \Rref{Malik:2006ir} for the multi-field
case. In Chapter~\ref{ch:perts}, second order gauge transformations were outlined
and the KG equation reproduced for a single scalar field model. In contrast to the
$\Delta \N$ approach, this equation is valid on all scales, both inside and outside
the horizon. 

In Fourier space, the second order KG equation \eqref{eq:SOKG-real-num} contains a
convolution term of the first order scalar field perturbation. For this first
demonstration of the numerical system, a slow roll approximation of the full
equation was used. 

Calculating the second order scalar field perturbations provides the possibility of
a unique insight into the generation and evolution of non-linear contributions to
the scalar curvature perturbation. The long term aim of this continuing project is
to analyze multi-field, non slow roll models in which non-Gaussian effects are
expected to play an important role.

As a step towards that goal, we described the implementation of a numerical
calculation of the single field, slow roll, second order equation in
Chapter~\ref{ch:numericalsystem}. The construction and evaluation of the convolved
source term in \eq{eq:KG2-src-sr-aterms} proved to be the most numerically complex
step required. 

To allow a numerical calculation an energy scale cutoff must be implemented. We used
a sharp cutoff at small wavenumbers below which the perturbations were taken to be
identically zero. Another cutoff at small scales or large wavenumbers was dictated
by practical reasons of calculation size and computation time.

We defined, in \eq{eq:AtoD-num}, four $\theta$ dependent integrals
into which the convolution term can be decomposed. By comparison with an analytic
solution for a particular smooth choice of the first order perturbation, an estimate
was made of the relative error present in the integration of one of these terms. The
number of discrete wavenumber values, their spacing, and the number of discrete
$\theta$ values were chosen to minimise this error. From these values, three
different closed finite ranges of discrete values of the wavenumber were defined.
These all contain the WMAP pivot scale at $\kwmap=0.002\Mpc^{-1}$ and cover the
WMAP observed scales to varying degrees.
% 
The other $\theta$ dependent terms have small relative errors for the three chosen
ranges. 



The execution of the code is in four stages, building on previous calculations of
first order perturbations in Refs.~\cite{Martin:2006rs, Ringeval:2007am} and
\cite{Salopek:1988qh}. Despite the large volume of calculations required at each
time step, the easily parallelisable nature of the source term calculations allows
the run time to be reduced significantly.


To test the code, four different, large field, monomial potentials were used. Each
one depends on a single parameter which was fixed by comparing the resultant scalar
curvature power spectrum with the WMAP5 normalisation value. The slow roll
approximation can be applied to all four of the potentials.

We presented the results for each potential in Chapter~\ref{ch:results}. The first
order results match those in Refs.~\cite{Martin:2006rs, Ringeval:2007am} and
\cite{Salopek:1988qh}. The results of the source term calculation show that before
horizon crossing, the source term amplitude decays rapidly for all four potentials.
After horizon crossing it becomes steadier and eventually increase at later times,
when the slow roll approximation breaks down.

The choice of wavenumber range affects the amplitude of the source term, as
expected, due to the implementation of a sharp cutoff at large scales. This
dependence is only apparent, however, before horizon crossing. For a particular
range, as wavenumber increase, the magnitude of the source term decreases. However,
the ratio of the source term to other terms in the KG equation increases with
wavenumber. 

As expected the amplitude of the second order scalar perturbations is much smaller
than that of the first order ones. After the generation of the second order
perturbations at early times, their evolution is that of a damped harmonic
oscillator similar to the first order evolution.


The numerical calculation described is the first step towards a system capable of
handling the multi field, non slow roll models for which non-linear contributions
are important. In Section~\ref{sec:next-res}, the next steps towards this goal were
outlined. The full single field, non slow roll equation can be treated using the
method already described. Three more $\theta$ dependent terms are necessary to
compute the full convolution integral in this case. 

The KG equation for the multi field case introduces further complexity. We plan to
expand the numerical system to encompass two or three scalar fields. The differences
between single and multi field models are already apparent for these cases.