% % % % % % % % % % % % % % % % % % % % % % 
% theend.tex - Ian Huston
% $Id: theend.tex,v 1.4 2009/11/07 17:16:58 ith Exp $
% % % % % % % % % % % % % % % % % % % % % % 
% Redefine CVSRevision for this section
\renewcommand{\CVSrevision}%
{\version$Id: theend.tex,v 1.4 2009/11/07 17:16:58 ith Exp $}

% % % % % % % % % % % % % % % % % % % % % % % % % % % % % % % % 
% =========================================================== %
% % % % % % % % % % % % % % % % % % % % % % % % % % % % % % % %
\chapter{Final Conclusions}
\label{ch:conclusions}
% % % % % % % % % % % % % % % % % % % % % % % % % % % % % % % % 
% =========================================================== %
% % % % % % % % % % % % % % % % % % % % % % % % % % % % % % % %

As the number of cosmological models grows, the need to constrain them is also
increasing. At the same time, the quantity and quality of observational data are
advancing. There now exists the opportunity to go beyond linear statistical analyses
and confront the predictions of models with observational data from the non-linear
regime. In this thesis both analytic and numerical methods have been used to
constrain inflationary models.

The framework used was the Friedmann-Robertson-Walker universe, which was reviewed
in Chapter~\ref{ch:introduction}. During the accelerated expansion of the
inflationary period, quantum fluctuations seeded energy density variations which
gave rise to the diverse structure in the universe. First order perturbation theory
was introduced to describe these fluctuations. Using the Bunch-Davies vacuum initial
conditions, observational predictions for the inflationary scenario were made.

The main observable quantities are the power spectrum of scalar perturbations, its
spectral index, and the ratio of tensor to scalar perturbations. In addition to
these the degree of departure of the perturbations from purely Gaussian statistics
was considered in the local and equilateral cases.