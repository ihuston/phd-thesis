% % % % % % % % % % % % % % % % % % % % % 
% multibrane.tex - Ian Huston
% $Id: multibrane.tex,v 1.46 2009/09/28 15:52:41 ith Exp $
% % % % % % % % % % % % % % % % % % % % % 
% Redefine CVSRevision for this section
\renewcommand{\CVSrevision}{\version$Id: multibrane.tex,v 1.46 2009/09/28 15:52:41 ith Exp $}

% % % % % % % % % % % % % % % % % % % % % % % % % % % % % % % % 
% =========================================================== %
% % % % % % % % % % % % % % % % % % % % % % % % % % % % % % % % 
\chapter{Multi brane inflation}
\label{ch:multibrane}
% % % % % % % % % % % % % % % % % % % % % % % % % % % % % % % % 
% =========================================================== %
% % % % % % % % % % % % % % % % % % % % % % % % % % % % % % % % 

% % % % % % % % % % % % % % % % % % % % % % % % % % % % % % % % 
% =========================================================== %
% % % % % % % % % % % % % % % % % % % % % % % % % % % % % % % % 
\section{Introduction}
\label{sec:intro-multi}
% % % % % % % % % % % % % % % % % % % % % % % % % % % % % % % % 
% =========================================================== %
% % % % % % % % % % % % % % % % % % % % % % % % % % % % % % % % 
We have seen in Chapter~\ref{ch:dbi} that the standard DBI inflationary model
is severely constrained by current observational data. The amplitude of tensor
perturbations is bounded from above by $r\le 10^7$. When the brane is moving
towards the tip of the throat, a complementary lower bound
on $r$ can be derived such that the two bounds are incompatible using
current observational data. In this chapter we will explore
how to evade, reconcile and weaken these bounds by considering a more general
class of models that exhibit properties similar to the standard DBI case.


In Section \ref{sec:relaxing-multi} we consider the special algebraic
properties of the DBI action.
We identify a general class of non-canonical inflationary models
where the leading-order contribution to the non-Gaussianity of the 
curvature perturbation is determined 
entirely by the speed of sound of the inflaton fluctuations. 
In these models, the bounds on $r$ can be relaxed 
if significant non-Gaussianities are generated.


As reviewed in Section~\ref{sec:others-dbi} many alternative ways to relax these
bounds have been proposed, including
theories based upon multi-field models, the addition of
angular momentum as another degree of freedom and using
different throat geometries. However in most cases
the extra degrees of freedom introduced in these models do not solve the problem
\cite{Alabidi:2008ej}. The 
bounds are relaxed only by a small fraction, and therefore these models should still be 
regarded as being unsatisfactory since they require an extreme amount of fine tuning in order
to work.

One alternative possibility is to consider 
multiple brane configurations\footnote{In certain limits this approach 
is actually dual to considering wrapped branes \cite{Ward:2007gs}.}. 
In the case where  
$n$ branes are localised initially at equal distances $l > l_s$ and 
subsequently follow the same trajectory, 
the effective theory is equivalent to that of $n$ copies of the
action for a single brane. A more general initial condition, particularly
for branes created in the IR region of the throat
\cite{brane13, DeWolfe:2004qx, Kachru:2002gs}, is that
the branes should be separated over a range of scales, 
with a subset being coincident and the remainder being widely separated. 


In Section~\ref{sec:multibranes-multi} we introduce the multi-brane
model with $n$ coincident branes described in \Rref{thomasward}. We will
consider two limiting cases of this model. The large $n$ case is similar in
form to the original DBI model and we will show in
Section~\ref{sec:twlargen-multi} that it can be constrained using the
formalism derived in Chapter~\ref{ch:dbi}. In contrast, for a finite number of
branes the effective action in the relativistic limit is shown to be in the
class of actions for which the bounds on $r$ can be relaxed.
In Section~\ref{sec:finiten-multi} we find that
such models 
can in principle lead to a detectable 
gravitational wave background if 
the number of coincident branes is sufficiently small. 





% 
% 
% % % % % % % % % % % % % % % % % % % % % % % % % % % % % % % % 
% =========================================================== %
% % % % % % % % % % % % % % % % % % % % % % % % % % % % % % % % 
\section{Relaxing the Upper Bounds on the Tensor-Scalar Ratio}
\label{sec:relaxing-multi}
% % % % % % % % % % % % % % % % % % % % % % % % % % % % % % % % 
% =========================================================== %
% % % % % % % % % % % % % % % % % % % % % % % % % % % % % % % % 
% 
In Chapter~\ref{ch:dbi-intro} we described the standard DBI scenario, in
which the kinetic function $P$ takes the form in \eq{eq:Pdefn-dbiintro}:
%  
\begin{equation}
\label{eq:DBIkinetic}
P (\varphi , X) = -T(\varphi ) \sqrt{1-2T^{-1}(\varphi ) X} + T(\varphi ) - V(\varphi ) \,,
\end{equation}
% 
where $T(\varphi ) = T_3 h^4 (\varphi )$ 
is the warped brane tension and  $V(\varphi )$ is the 
inflaton potential. The standard DBI scenario is
algebraically special, 
in the sense that the kinetic function satisfies the constraints 
% 
\begin{equation}
\label{eq:cspx1}
c_s P_{,X} = 1 , \qquad  \Lambda = \frac{1}{2} \left( 
\frac{1}{c_s^2} -1 \right) \,.
\end{equation}
% 
We saw in Section~\ref{sec:relaxing-dbi} that the 
bounds \eqref{eq:genBMbound} and \eqref{eq:LHbound} 
on the tensor-scalar ratio could in principle be 
significantly relaxed in models where $(c_sP_{,X})_* \gg 1$. 
In view of the second relation in \eq{eq:cspx1}, it is of interest 
to take a phenomenological approach and consider the more  
general class of models where
%  
\begin{equation}
\label{eq:defalpha}
\frac{1}{c_s^2} -1 = \alpha \Lambda \,,
\end{equation}
% 
for some positive constant $\alpha$. 

% % % % % % % % % % % % % % % % % % % % % % % % % % % % % % % % 
% =========================================================== %
% % % % % % % % % % % % % % % % % % % % % % % % % % % % % % % % 
\subsection{Approximate Solution}
\label{sec:approx-multi}
% % % % % % % % % % % % % % % % % % % % % % % % % % % % % % % % 
% =========================================================== %
% % % % % % % % % % % % % % % % % % % % % % % % % % % % % % % % 
A large non-Gaussian signature in the curvature perturbation is 
typically generated in models where the sound speed of fluctuations 
is small. We will begin by considering 
scenarios in which this is the case and therefore the kinetic function satisfies
the inequalities: 
% 
\begin{equation}
\label{eq:Plimits}
X^2 P_{,XXX} \gg XP_{,XX} \gg P_{,X} \,.
\end{equation}
% 


In these limits the constraint (\ref{eq:defalpha}) reduces to the 
third-order, non-linear, partial differential equation
% 
\begin{equation}
\label{eq:pde-multi}
P^2_{,XX} = \frac{\alpha}{6} P_{,X} P_{,XXX} \,.
\end{equation}
% 
Changing the dependent variable to $Q \equiv P_{,XX}/P_{,X}$ 
reduces \eq{eq:pde-multi} to
%  
\begin{equation}
\label{eq:Qdefn-multi}
\alpha Q_{,X} = (6-\alpha )Q^2 \,,
\end{equation}
% 
and it is straightforward to integrate \eq{eq:Qdefn-multi} 
exactly. The remaining integrations can also be performed analytically 
and the general solution to \eq{eq:pde-multi} for $\alpha \ne
6$ is given by\footnote{The special case $\alpha =6$ results in an 
exponential dependence of the kinetic function on $X$, and is 
therefore an example of a higher-derivative theory. However we 
do not consider this model further, since it does not lead to a 
weakening of the gravitational wave constraints.}
% 
\begin{equation}
\label{eq:gensoln-multi}
P (\varphi , X) = -f_1 (\varphi ) \left[ 1-f_2 (\varphi ) X 
\right]^l -f_3(\varphi ) \,,
\end{equation}
% 
where $f_i (\varphi )$ are arbitrary functions of the scalar 
field and
%  
\begin{equation}
l \equiv \frac{2(\alpha -3 )}{\alpha -6} \, . 
\end{equation}
% 
It can be verified that the inequalities (\ref{eq:Plimits}) 
are satisfied in the `relativistic' limit, where $X \simeq 1/f_2$.
We consider the inflationary dynamics in this limit in what follows but for
completeness we show in the next section that \eq{eq:defalpha} can be solved
analytically without this approximation. 

% % % % % % % % % % % % % % % % % % % % % % % % % % % % % % % % 
% =========================================================== %
% % % % % % % % % % % % % % % % % % % % % % % % % % % % % % % % 
\subsection{Analytic General Solution}
\label{sec:apx-multi}
% % % % % % % % % % % % % % % % % % % % % % % % % % % % % % % % 
% =========================================================== %
% % % % % % % % % % % % % % % % % % % % % % % % % % % % % % % % 
\eq{eq:defalpha} can be analytically solved in full 
generality without imposing the limits (\ref{eq:Plimits}) on the 
derivatives of the kinetic function. This allows us to determine the 
most general class of models where the non-linearity parameter 
satisfies the condition $\fnleq \propto 1/c_s^2$ at leading order. 

In general \eq{eq:defalpha} takes the form 
% 
\begin{equation}
\label{eq:genPXeqn-multi}
(2-\alpha ) P_{,X}P_{,XX} + 4XP^2_{,XX} = \frac{2\alpha }{3}
X P_{,X}P_{,XXX}
\end{equation}
% 
and this reduces to 
% 
\begin{equation}
\label{eq:genreduce-multi}
\alpha Q_{,X} = (6-\alpha ) Q^2 + \frac{3(2-\alpha )}{2} \frac{Q}{X} \, ,
\end{equation}
% 
where $Q \equiv P_{,XX}/P_{,X}$. 
\eq{eq:genreduce-multi} can be transformed into the 
linear equation
% 
\begin{equation}
\label{eq:lineargen-multi}
U_{,X}+ \frac{3(2-\alpha )}{2\alpha} \frac{U}{X} = \frac{\alpha -6}{\alpha}
\end{equation}
% 
after the change of variables $U \equiv 1/Q$
and the general solution to \eq{eq:lineargen-multi} is given by
%  
\begin{equation}
\label{eq:gensolnlinear-multi}
\frac{P_{,XX}}{P_{,X}} = \frac{1}{X\left[ f_2(\varphi) X^{(\alpha -6)/2\alpha}
-2 \right] } \, .
\end{equation}
% 
Integrating a second time implies that
% 
\begin{equation}
\label{eq:secondint-multi}
P_{,X} = -f_1 (\varphi ) \left( 1- f_2(\varphi ) X^{-s} \right)^{1/(2s)}  \, ,
\end{equation}
% 
where $s \equiv (\alpha -6 )/(2 \alpha)$ and we have redefined 
the arbitrary integration functions $f_i(\varphi )$.  
Finally \eq{eq:secondint-multi} can be formally integrated 
in terms of a hypergeometric function
%  
\begin{equation}
 \label{eq:thirdint-multi}
 P= -f_1X \,{_2}F_1 \left( -\frac{1}{s}, -\frac{1}{2s}; 1-\frac{1}{s}, f_2X^{-s}
\right)  \, ,
\end{equation}
%  
which represents the most general solution for this class of models. 
Note that we have set the
remaining constant of integration to zero to ensure 
that the kinetic function vanishes in the limit of
zero velocity. In fact this expression admits many 
different classes of solution, arising as limits
of the expansion of the hypergeometric function.


% The special case of $\alpha =2$ $(s=-1)$ implies (after a 
% further redefinition of the functions $f_i (\varphi ))$ that 
% % 
% \begin{equation}
% \label{eq:DBIsoln-multi}
% P = -f_1 \sqrt{1-f_2 X} -f_3
% \end{equation}
% % 
% and this corresponds to the standard DBI action (\ref{eq:DBIkinetic}) 
% \cite{chenetal,lidser2}.          




% % % % % % % % % % % % % % % % % % % % % % % % % % % % % % % % 
% =========================================================== %
% % % % % % % % % % % % % % % % % % % % % % % % % % % % % % % % 
\subsection{Consequences}
\label{sec:consequences-multi}
% % % % % % % % % % % % % % % % % % % % % % % % % % % % % % % % 
% =========================================================== %
% % % % % % % % % % % % % % % % % % % % % % % % % % % % % % % % 

The standard DBI scenario is recovered from \eq{eq:gensoln-multi} for
$l=1/2, \alpha = 2$ (or $s=-1$ in the exact case following a redefinition of
$f_i$). 
More generally however, \eq{eq:gensoln-multi} implies that  
% 
\begin{eqnarray}
\label{eq:consequence-multi}
c_sP_{,X} \simeq \frac{l f_1 f_2}{\sqrt{2(1-l)}} \left( 
1- f_2X \right)^{(2l-1)/2} \,,
\\
c_s^2 \simeq \frac{1-f_2X}{2(1-l)} \,,
\end{eqnarray}
% 
when $X \simeq 1/f_2$. Self-consistency therefore
requires $l<1$. Moreover
we find from \eq{eq:fnldefn-dbiintro} that
%  
\begin{eqnarray}
\label{eq:consequence3-multi}
\fnleq \simeq \frac{-\beta}{1-f_2X} , \qquad \beta \equiv \frac{5(59-55l)}{486}
\,,
\\
\label{eq:consequence4-multi}
\fnleq \simeq -\frac{\sigma}{c_s^2}, \qquad \sigma \equiv 
\frac{5}{972} \left( \frac{59-55l}{1-l} \right) \,.
\end{eqnarray}
% 
Hence substituting Eqs.~\eqref{eq:consequence-multi} and
\eqref{eq:consequence3-multi} 
into the BM bound \eqref{eq:genBMbound} and the bound \eqref{eq:LHbound}
implies that
%  
\begin{equation}
\label{eq:weakBM-multi}
r_*< \frac{32}{N \Neff^2} \frac{l f_1 f_2}{\sqrt{2(1-l)}}
\left( -\frac{\fnleq}{\beta} \right)^{(1-2l)/2}
\end{equation}
% 
and
%  
\begin{equation}
\label{eq:weakLH-multi}
r_* < \frac{10}{(\Delta \N)_*^2} \left( \frac{T_3}{\Vol} \right)^{1/3} 
\left( \frac{h_*}{\Mpl} \right)^{4/3}
\frac{l f_1 f_2}{\sqrt{2(1-l)}}
\left( -\frac{\fnleq}{\beta} \right)^{(1-2l)/2}
\end{equation}
% 
respectively. 


We conclude, therefore, that 
the upper limit on the tensor-scalar ratio could be significantly 
relaxed if $l <1/2$, since the non-linearity parameter is at present only 
weakly constrained at $\fnleq >-151$. Although it is possible 
to phenomenologically construct a model which has a value of $l$ in this 
range, it is clearly preferable to identify  UV complete models
that satisfy this requirement within a string theory context. 
Unfortunately this is quite difficult to achieve since the inflaton 
will either be associated with an open or closed string mode. 
The open strings are governed by relativistic actions of the 
DBI form, whilst closed strings arise from compactification of Einstein gravity
and are typically put into canonical form.
However there do exist classes of open string
models which satisfy the above requirement, 
namely those associated with multiple coincident branes.


More specifically, if the branes are
spatially separated, the effective action is algebraically equivalent 
to that of a single brane. It will therefore not satisfy the 
bound on $l$ \footnote{In this discussion, we are ignoring 
the non-trivial backreaction of these branes on the background, and therefore 
one should be careful about the range of validity of the effective action.}. 
In the remainder of this chapter we will examine the case of $n$ coincident
branes as described in \Rref{thomasward}. The large $n$ limit will also
fall into the class of models with $l=1/2$ which are equivalent to the single
brane case. On the other hand, if it is assumed that 
$n$ is finite, the special properties associated with the 
matrix degrees of freedom become important and this 
results in a kinetic function satisfying $l \le 1/2$.



% 
% 
% 
% % % % % % % % % % % % % % % % % % % % % % % % % % % % % % % % 
% =========================================================== %
% % % % % % % % % % % % % % % % % % % % % % % % % % % % % % % % 
\section{The Multiple Coincident Branes model} 
\label{sec:multibranes-multi}
% % % % % % % % % % % % % % % % % % % % % % % % % % % % % % % % 
% =========================================================== %
% % % % % % % % % % % % % % % % % % % % % % % % % % % % % % % % 


We have seen how the form of the kinetic function
$P$ can significantly change the
strength of the bound in \eq{eq:LHbound} on the tensor-scalar ratio, depending
on its explicit
form. One model in which a suitable form for 
$P$ is realised is the multiple coincident
brane model as outlined by Thomas and Ward in \Rref{thomasward}.
In this model, the flux annihilation process 
generates $n$ coincident branes that are initially located at the 
bottom of a throat region. The dynamics of this configuration
is determined by the non-Abelian world-volume theory \cite{myers1,myers2}. 
This theory exhibits extra stringy degrees of freedom which arise due to the 
fuzzy nature of the geometry.

% The world-volume theory for coincident branes is not fully known, although
% a number of proposals have been made. We will restrict our analysis to Myers' 
% prescription, since this has been extensively discussed in the 
% literature\footnote{There is also a proposal by Tseytlin \cite{Tseytlin}
% for the non-Abelian theory of coincident D-branes.} \cite{myers1, myers2}. 
In general the open string degrees of freedom for $n$ coincident branes 
combine to fill out representations of $U(n)$ (as opposed to $U(1)^n$ 
in the case of separated branes). This introduces a non-Abelian 
structure into the theory. In the single brane case, the fluctuations of the 
brane are characterised by induced scalar fields on the world-volume. 
However for multiple branes
these scalars must be promoted to matrix representations of some gauge group. 
% 
Typically the transverse space of any given compactification will always admit
an $SO(3)$ isometry. Scalars can therefore be chosen to 
transform under representations of the algebra of $SO(3) \sim SU(2)$ by making 
the identifications
% 
\begin{equation}
\varphi^i = R_m \alpha^i \hspace{1cm} i =1, 2, 3 \,,
\end{equation}
% 
where $R_m$ is some scale with canonical mass dimension, and the $\alpha^i$ are
specified to be the irreducible generators satisfying the commutator
% 
\begin{equation}
[\alpha^i, \alpha^j] = 2i \epsilon_{ijk} \alpha^k \,,
\end{equation}
% 
and the conditions
% 
\begin{equation}
\frac{1}{n} Tr(\alpha^i \alpha^j) = \hat{C} \delta^{ij} = (n^2-1) \delta^{ij}
\,,
\end{equation}
% 
where $\hat{C}$ is the quadratic Casimir of the gauge group.
The irreducibility condition corresponds to the configuration being in the
lowest energy state. It is therefore an additional fine-tuning
of the initial conditions. 


The Myers prescription requires a symmetrised trace (denoted $STr$) to 
be made over the gauge group. This implies that 
the symmetric averaging must be taken over all the group dependence 
before taking the trace:
% 
\begin{equation}
 \label{eq:str-defn-multi}
 STr (A_1 \ldots A_s) = \frac{1}{s!} Tr(A_1 \ldots A_s + \text{all
permutations})\,.
\end{equation}
% 
For a large number of branes, $n \gg 1$, the symmetric trace can be
approximated with a trace, 
which results in the usual DBI action multiplied by a potential term (as
described in \cite{thomasward, Kachru:2002gs}). 
However for finite $n$, the symmetrisation becomes more important and it
is essential to have some means of performing this operation. A prescription for
treating the symmetric trace at finite $n$ was proposed
in Refs.~\cite{Ramgoolam:2004gw} and \cite{McNamara:2005ry}, 
using highest weight methods and chord diagrams. 
% 
The result is that the $STr$ acts on different spin representations of $SU(2)$ 
in the following manner:
% 
\begin{eqnarray}
\label{eq:str-even-multi}
STr (\alpha^i \alpha^i)^q &=& 2(2q+1)\sum_{i=1}^{n/2}(2i-1)^{2q} , 
\qquad n\; \mathrm{ even}\,, 
\\
\label{eq:str-odd-multi}
STr (\alpha^i \alpha^i)^q &=& 2(2q+1)\sum_{i=1}^{(n-1)/2} (2i)^{2q} , 
\qquad n\; \mathrm{ odd}\,.
\end{eqnarray}
% 


In order for the solution to converge in this prescription, 
it is also necessary to modify the definition of the radius of the 
$SU(2)$ sphere. In the large $n$ limit, this is given by
% 
\begin{equation}
\label{eq:largen-rho-multi}
\rho^2 = \lambda^2 R_m^2 \frac{1}{n} Tr(\alpha^i \alpha^i) = \lambda^2 R_m^2
\hat{C} \,,
\end{equation}
% 
where $\lambda \equiv 2\pi l_s^2 = 2\pi m_s^{-2}$, 
whereas for finite $n$, it becomes
% 
\begin{equation}
\label{eq:finiten-rho-multi}
\rho^2 = \lambda^2 R_m^2 \mathrm{Lim}_{q \to \infty} \left(\frac{STr (\alpha^i
\alpha^i)^{q+1}}{STr(\alpha^i \alpha^i)^q} \right) 
= \lambda^2 R_m^2 (n-1)^2 \,.
\end{equation}
% 
This converges to the large $n$ result in the appropriate limit.
This point is important, since the warp factor 
of the four-dimensional theory is typically of the form $h= h(\rho)$.
% 
The next two sections will examine this coincident brane model in both the
large and finite $n$ limits. 

% % % % % % % % % % % % % % % % % % % % % % % % % % % % % % % % 
% =========================================================== %
% % % % % % % % % % % % % % % % % % % % % % % % % % % % % % % % 
\section{Multi-brane inflation with a large number of branes}
\label{sec:twlargen-multi}
% % % % % % % % % % % % % % % % % % % % % % % % % % % % % % % % 
% =========================================================== %
% % % % % % % % % % % % % % % % % % % % % % % % % % % % % % % % 
Taking the limit of a large number of coincident branes significantly
simplifies the non-Abelian action. The symmetrised trace can
now be replaced with a normal trace operator and the expression for $\rho$
takes the form in \eq{eq:largen-rho-multi}. 
% 
 For the case where a fuzzy two-sphere is 
embedded in a three-cycle in the $X_5$ manifold, 
the kinetic structure of the action is given in the large $n$ limit by 
\cite{thomasward}
% 
\begin{equation}
\label{eq:largeP-multi}
P=-nT_3 \left[ h^4(\varphi ) W(\varphi ) 
\sqrt{1-2 T_3^{-1} h^{-4}(\varphi) X}
-h^4(\varphi ) + V (\varphi ) \right] \,,
\end{equation}
% 
where
%    
\begin{equation} 
\label{eq:defW}
W (\varphi ) \equiv \sqrt{1+ C^{-1}h^{-4}(\varphi ) \varphi^4}
\end{equation}
% 
defines the so-called `fuzzy' potential, 
$C = \pi^2 \hat{C}T_3^2/m_s^4$ is a model-dependent constant and 
$\hat{C} \simeq n^2$ in the large $n$ limit. 

The kinetic term \eqref{eq:largeP-multi} is clearly of the same form as the
single-brane DBI term, with $l=1/2$ in the scheme outlined in
Section~\ref{sec:relaxing-multi}. We can therefore apply the analysis of
Section~\ref{sec:relaxing-dbi} to see if the bounds described in
Chapter~\ref{ch:dbi} can be relaxed.
Comparison with \eq{eq:genaction-dbi} 
implies that $f_1f_2 =2nW$ and $f_2=2/(T_3h^4)$. Hence, 
the new features of this model relative to the standard single-brane 
scenario are parametrized in terms of the fuzzy potential $W (\varphi )$. 
This configuration is conjectured to be dual to 
a ${\rm D5}$-brane which is wrapped around a two-cycle 
of the throat \cite{dual1,dual2,dual3}. 


The regime $W \gg 1$ is of interest for 
relaxing the gravitational wave constraints\footnote{Note that 
the case $n \gg 1$ and
$W \sim 1$ will not significantly relax the BM bound, 
since we require $n \ll N$ for backreaction effects to be negligible.}. 
The generalized BM bound (\ref{eq:f2bound}) may now be expressed as 
a limit on the value of the warp factor $h(\varphi_*)$ on CMB scales: 
% 
\begin{equation}
\label{eq:warptorelax}
\frac{NT_3h^4_*}{\Mpl^4} < 
\frac{8\pi^2 (1-n_s)\Pr}{\sqrt{-3\fnleq}(\Delta \N_*)^2} \,.
\end{equation}
% 


We now consider whether this limit can be satisfied for reasonable choices 
of parameters when the warped compactification corresponds to 
an $AdS_5$ or KS throat, respectively. Recall that the warp 
factor for the $AdS_5$ throat is given by $h=\varphi/(\sqrt{T_3}L)$.  
Condition (\ref{eq:warptorelax}) therefore reduces to a constraint on the 
value of the inflation during observable inflation: 
% 
\begin{equation}
\label{eq:phivalue-dbi}
\frac{\varphi_*^4}{\Mpl^4} < 
\frac{8\pi^2 (1-n_s)\Pr}{\sqrt{-3\fnleq} ( \Delta \N_*)^2} 
\frac{\lambda}{N} \,,
\end{equation}
%  
where $\lambda \equiv \pi N/(2 \mathrm{Vol}(X_5))$. 
However, non-perturbative string effects are expected to become 
important below a cutoff scale, $\varphi_{\rm cut} = 
h_{\rm cut} \lambda^{1/4} m_s$, where $h_{\rm cut}$ is the value of the 
warp factor at that scale. For consistency, therefore, one requires 
$\varphi_*>\varphi_{\rm cut}$, which implies an upper limit on the 
${\rm D3}$-brane charge: 
% 
\begin{equation}
\label{eq:Nlimit-dbi}
N< \frac{8\pi^2 (1-n_s)\Pr}{\sqrt{-3\fnleq} ( \Delta \N_*)^2}
\left( \frac{\Mpl}{h_{\rm cut}m_s} \right)^4  \,.
\end{equation}
% 
Assuming the typical values $m_s \sim 10^{-2}\Mpl$, 
$\Delta \N_* \simeq 4$ and 
$h_{\rm cut} \sim 10^{-2}$ implies  
$N < 7 \times 10^7 (1-n_s)(-\fnleq)^{-1/2} < 2\times 10^6$, where 
the latter inequality follows for $1-n_s <0.05$ and $\fnleq <-5$. 


For an $AdS_5$ throat, the fuzzy potential 
is a constant and the condition that $W \gg 1$ becomes 
% 
\begin{equation}
\label{eq:Chatlimit}
\hat{C} \ll \frac{4\pi^2g_sN}{\mathrm{Vol}(X_5)} \,.
\end{equation}
% 
Hence, combining inequalities 
(\ref{eq:Nlimit-dbi}) and (\ref{eq:Chatlimit}) implies that
%  
\begin{equation}
\label{eq:nlimit-dbi}
\hat{C} \ll 
\frac{32 \pi^4 (1-n_s)\Pr}{\sqrt{-3\fnleq} ( \Delta \N_*)^2}
\frac{g_s}{\mathrm{Vol} (X_5) }
\left( \frac{\Mpl}{h_{\rm cut}m_s} \right)^4  \,,
\end{equation}
% 
and specifying $g_s \sim 10^{-2}$ and 
$\mathrm{Vol}(X_5) \simeq \pi^3$ then yields the limit  
$\hat{C} \ll 10^{6} (1-n_s)(-\fnleq)^{-1/2} < 2 \times 10^4$, or equivalently,  
$n \ll 150$. Furthermore, since $f_1f_2 \simeq {\rm constant}$, 
the inequality (\ref{eq:genBMbound}) may be strengthened by a 
factor of $(   \Neff /\Delta \N_*)^2$ by 
substituting 
$\Delta \N_* \rightarrow \Neff$. This ratio 
could be as 
high as $(60/4)^2 \simeq 200 $, which would rule out this particular
model. 


Since the branes are initially located at the tip of the 
throat, another case of interest is the IR limit of the KS geometry, where 
the warp factor asymptotes to a constant value 
\cite{gkp}:
% 
\begin{equation}
\label{eq:KStip-dbi}
h_{\rm tip} = \exp \left( - \frac{2\pi K}{3Mg_s} \right) \,.
\end{equation}
% 
% and $K,M \in {\rm \bf Z^+}$ are the units of flux associated 
% with the NS-NS and R-R three-forms, respectively, such that $N=MK$.
In this case, the generalized BM bound (\ref{eq:warptorelax}) becomes
%  
\begin{equation}
\label{eq:loglimit-dbi}
\frac{8\pi K}{3Mg_s} -\ln N > 4 \ln \left( \frac{m_s}{g_s^{1/4}\Mpl} \right)
-\ln \left( 
\frac{64\pi^5 (1-n_s)\Pr}{\sqrt{-3\fnleq} ( \Delta \N_*)^2}
\right) \,.
\end{equation}
% 
The radius of the three-sphere at the tip of the KS 
throat is of the order $(g_sM)^{1/2}$ in string units 
and this must be large (and at the very least should exceed
unity) for the supergravity approximation to be reliable. 
Substituting this requirement into expression (\ref{eq:loglimit-dbi}) 
results in a necessary (but not sufficient) condition 
on the ${\rm D3}$-brane charge for the 
generalized BM bound to be satisfied:
%  
\begin{equation}
\label{eq:Nlowerlimit-dbi}
\frac{1}{N} \exp \left( \frac{8\pi g_s N}{3}  \right)
> \frac{\sqrt{-3\fnleq} ( \Delta \N_*)^2}{64\pi^5 
(1-n_s)\Pr} \frac{1}{g_s} \left( \frac{m_s}{\Mpl} \right)^4 \,.
\end{equation}
% 


Recalling that a necessary condition for the backreaction 
of the branes to be negligible is 
$N \gg n \gg 1$ implies that the 
exponential term in (\ref{eq:Nlowerlimit-dbi}) will dominate unless 
the string coupling constant is extremely small. Hence, for 
the parameter estimations quoted above, we 
deduce the lower limit 
% 
\begin{equation}
\label{eq:Nconstraint-dbi}
N- 12 \ln N > -6.8 +12 \ln \left( \frac{\sqrt{-\fnleq}}{1-n_s} \right) \,,
\end{equation}
% 
which becomes $N \, \gsim \, 10^2$ for $1-n_s \simeq 0.05$ and $\fnleq<-5$.


In general, however, the $K$ and $M$ units of flux are not independent. 
F-theory compactification on Calabi-Yau four-folds
provides a geometric way of parameterizing  
type IIB string compactifications
\cite{witten1,witten2,witten3,sethi,gkp,klemm}. 
Global tadpole cancellation constrains the topology of the four-fold
and this restricts the brane and flux configurations.  
When the KS system is embedded into F-theory, the  
constraint is given by \cite{gkp}
% 
\begin{equation}
\label{eq:Ftheory}
\frac{\chi}{24} = n + MK \,,
\end{equation}
% 
where $\chi$ is the Euler characteristic of the four-fold.  
Hence, $N = MK < \chi /24$ and together with condition 
(\ref{eq:Nconstraint-dbi}), this implies that
% 
\begin{equation}
\label{eq:chilimit}
\chi > 2400 \,,
\end{equation}
% 
for $N > 10^2$.
It is known that the Euler number for four-folds 
corresponding to hypersurfaces in weighted projective spaces
can be as high as $\chi \le 1,820,448$ \cite{klemm},
so there are many compactifications that could 
in principle satisfy the generalized BM bound.
On the other hand, the above limit on the Euler characteristic 
does allow us to gain some insight into the 
topology of the extra dimensions, since compactifications 
which result in a small Euler characteristic would be  
incompatible with the generalized BM bound. 

% 
% 
% % % % % % % % % % % % % % % % % % % % % % % % % % % % % % % % 
% =========================================================== %
% % % % % % % % % % % % % % % % % % % % % % % % % % % % % % % % 
\section{Multi-brane inflation with a finite number of branes}
\label{sec:finiten-multi}
% % % % % % % % % % % % % % % % % % % % % % % % % % % % % % % % 
% =========================================================== %
% % % % % % % % % % % % % % % % % % % % % % % % % % % % % % % % 

% % % % % % % % % % % % % % % % % % % % % % % % % % % % % % % % 
% =========================================================== %
% % % % % % % % % % % % % % % % % % % % % % % % % % % % % % % % 
\subsection{The finite $n$ model}
\label{sec:finitensub-multi}
% % % % % % % % % % % % % % % % % % % % % % % % % % % % % % % % 
% =========================================================== %
% % % % % % % % % % % % % % % % % % % % % % % % % % % % % % % % 
The coincident brane model outlined in Section~\ref{sec:multibranes-multi}
takes a significantly different form if instead of assuming a large number of
coincident branes there are now only a small finite number. The presciption for
the symmetrised trace given in Eqs.~\eqref{eq:str-even-multi} and
\eqref{eq:str-odd-multi} must be used and $\rho$ is determined from
\eq{eq:finiten-rho-multi}. 
% 
The resulting kinetic function for $n$ coincident branes in 
the finite $n$ limit is therefore given by
% 
\begin{equation}
\label{eq:genP-multi}
P = -T_3 STr \left(h^4(\rho) \sum_{k,p=0}^{\infty}(-\XR)^k Y^p (\alpha^i
\alpha^i)^{k+p}{1/2 \choose k} {1/2 \choose p} + V(\rho)-h^4(\rho) \right) \,,
\end{equation}
% 
where 
% 
\begin{equation}
Z \equiv \lambda^2 h^{-4}(\rho), \hspace{0.5cm} Y \equiv 4\lambda^2 R_m^4
h^{-4}(\rho),
\hspace{0.5cm} {1/2 \choose q}
\equiv \frac{\Gamma(3/2)}{\Gamma(3/2-q)\Gamma(1+q)} \,.
\end{equation}
% 
Note that the second and third terms in \eq{eq:genP-multi} 
are singlets under the $STr$ and therefore contribute terms proportional 
to $n$. The physics of these branes away from the large $n$ limit is particularly interesting as
discussed further
in Refs.~\cite{thomasward} and \cite{Ward:2007gs}.


It was shown in \Rref{hltw} that $P$ and $E$ of all the solutions with $n>2$
can be derived recursively from the $n=2$ solution. We will use the notation
$P_n$ and $E_n$ for the first part of the kinetic and energy density terms for
the $n$ branes solution. The full pressure and energy densities are then 
given by $P = P_n - nT_3(V-h^4)$ and $E = E_n + nT_3 (V-h^4)$. Using
\eq{eq:str-even-multi} and the expressions
% 
\begin{eqnarray}
 \sum_{k=0}^\infty A^k {1/2 \choose k} &=& \sqrt{1+A}\,,\\
 \sum_{k=0}^\infty A^k {1/2 \choose k} 4k &=& \frac{2A}{\sqrt{1+A}}\,,
\end{eqnarray}
% 
the terms $P_2$ and $E_2$ can be derived:
% 
\begin{eqnarray}
\label{eq:2brane}
P_2 \left[Z,Y\right] &=& - 2 T_3 h^4 \left(\frac{(1+2Y -
(2+3Y)\XR)}{\sqrt{1+Y}\sqrt{1-\XR}}\right) \,, \nonumber \\
E_2 \left[Z,Y\right] &=& 2 T_3 h^4 \left(\frac{(1+2Y -
Y\XR)}{\sqrt{1+Y}(1-\XR)^{3/2}}\right) \,.
\end{eqnarray}
% 
% These quantities correspond to the pressure and energy density functions 
% when $n=2$ which arise solely from the DBI sector of the action.
% Since the symmetrised trace acts differently on the differing spin
% representations of $SU(2)$, we should expect this structure to follow
% through in the recursion relation. Indeed, we find that for odd $n$
% 
The recursion relation described in \Rref{hltw} can then be written for odd $n$:
% 
\begin{eqnarray}
\label{eq:oddbrane}
P_n^{(O)} &=& \left(\sum_{k=1}^{(n-1)/2} P_2 \left[(2k)^2Z, (2k)^2Y\right]
\right)-nT_3(V-h^4) \,, \nonumber \\
E_n^{(O)} &=& \left(\sum_{k=1}^{(n-1)/2} E_2 \left[(2k)^2Z, (2k)^2Y \right]
\right)+
nT_3(V-h^4) \,,
\end{eqnarray}
% 
and for even $n$:
%  
\begin{eqnarray}
\label{eq:evenbrane}
P_n^{(E)} &=& \left(\sum_{k=1}^{n/2} P_2 \left[(2k-1)^2Z, (2k-1)^2Y\right]
\right)-nT_3(V-h^4) \,, \nonumber \\
E_n^{(E)} &=& \left(\sum_{k=1}^{n/2} E_2 \left[(2k-1)^2Z, (2k-1)^2Y \right]
\right)+
nT_3(V-h^4) \,.
\end{eqnarray}
% 


% For example, we can employ these recursion relations to obtain the 
% solutions for $n=3$:  
% % 
% \begin{eqnarray}
% P &=& - 2T_3 \left(\frac{h^4
% (1+8Y-8\XR(1+6Y))}{\sqrt{1-4\XR}\sqrt{1+4Y}}\right)
%  -3T_3(V-h^4) \,, \nonumber \\
% E &=&  2T_3 \left(\frac{h^4
% (1+8Y(1-2\XR))}{(1-4\XR)^{3/2}\sqrt{1+4Y}}\right) +3T_3(V-h^4) \,,
% \end{eqnarray}
% % 
% which agrees precisely with the result computed by direct 
% expansion of the $STr$ prescription. Furthermore the $n=4$ case is given by 
% % 
% \begin{eqnarray}
% P = -2T_3\left(\frac{h^4
% (1+2Y-\XR(2+3Y))}{\sqrt{1+Y}\sqrt{1-\XR}} \right. 
% \nonumber \\
% \left. +\frac{h^4(1+18Y-9\XR(2+27Y))}
% {\sqrt{1+9Y}\sqrt{1-9\XR}}\right) 
%  -4T_3(V-h^4) \,, \nonumber \\
% E =  2T_3 \left(\frac{h^4(1+2Y-Y\XR)}{\sqrt{1+Y}(1-\XR)^{3/2}} +
% \frac{h^4(1+18Y-81\XR)}{\sqrt{1+9Y}(1-9X\dot{R_m}^2)^{3/2}} \right)
% +4T_3(V-h^4) \, .
% \end{eqnarray}
% % 
% It it clear that the relevant functions increase in complexity as 
% $n$ increases, since there are progressively more
% terms to include in the $STr$ expansion. However Eqs. (\ref{eq:oddbrane}) and
% (\ref{eq:evenbrane})
% represent the most general
% solutions. 


% One should also be aware that the backreaction of multiple branes will
% typically introduce corrections of the form $n/N$, therefore it is important
% for this ratio to be small in order for us to trust the supergravity analysis.
% Typically we can argue that wrapped branes are dual to multi-brane
% configurations when
% we are in the limit that $n>>1$. However since we also wish to keep $N>>1$ we
% must tune
% the solution so that $n/N<<1$ is satisfied. Therefore the origin of the
% backreaction effects
% is much clearer from this perspective. One can compute the $1/N$ corrections
% to the multi-brane
% action in the large $n$ limit \cite{Ward:2007gs} which, in the dual picture,
% correspond to
% backreactive corrections
% to the wrapped brane models. It would certainly be more useful to develop both
% these models in more
% detail.

The backreaction of the multiple branes introduces corrections of the form $n/N$
\cite{hltw}. Ensuring that this ratio is small allows the continued use of the
supergravity analysis. As we will see in the next section it will not be
difficult to let $N\gg n$ and still constrain this model.


% 
% 
% % % % % % % % % % % % % % % % % % % % % % % % % % % % % % % % 
% =========================================================== %
% % % % % % % % % % % % % % % % % % % % % % % % % % % % % % % % 
\subsection{Bounds on the Tensor-Scalar Ratio for finite $n$} 
\label{sec:multibounds-multi}
% % % % % % % % % % % % % % % % % % % % % % % % % % % % % % % % 
% =========================================================== %
% % % % % % % % % % % % % % % % % % % % % % % % % % % % % % % % 
In the last section we introduced the multi coincident brane model in the limit
of a finite number of branes. In this section we will consider this model in
the context of the class of actions derived in Section~\ref{sec:relaxing-multi}
and show that current observational data can strongly constrain the ability of
this model to produce an observable tensor signal.

The last terms appearing in the summations of Eqs.~\eqref{eq:oddbrane} 
and \eqref{eq:evenbrane} correspond  to the $k=(n-1)/2$ 
term when $n$ is odd and to the $k=n/2$ term when $n$ is even. This 
implies that for all $n$, these terms can be expressed in the form 
% 
\begin{equation}
\label{eq:unifiedgenP-multi}
P = -2T_3 \left\{ \frac{h^4 \left[ 1+2(n-1)^2Y
- [ 2+3(n-1)^2Y] (n-1)^2\XR  \right]}{\sqrt{1+(n-1)^2Y}
\sqrt{1-(n-1)^2\XR}} 
 \right\} -n T_3 \left( V -h^4 \right) \,.
\end{equation}
% 


Inspection of Eqs.~\eqref{eq:2brane}-\eqref{eq:evenbrane} implies that 
the relativistic limit is realised for any finite number of branes when 
$(n-1)^2 \XR \rightarrow 1$. In this case, the dominant contribution 
to the summations appearing in Eqs.~\eqref{eq:oddbrane} and
\eqref{eq:evenbrane}
will arise from the last term, \eq{eq:unifiedgenP-multi}. In this limit, 
therefore, the kinetic function appearing in the effective action simplifies to 
% 
\begin{equation}
\label{eq:unifiedP-multi}
P = 2T_3 \left\{ h^4 \sqrt{1+(n-1)^2Y} 
\left( 1- \frac{2X}{T_3h^4} \right)^{-1/2} 
 \right\} - n T_3 \left(V - h^4 \right) \,,
\end{equation}
% 
where 
% 
\begin{equation}
\label{eq:defY-multi}
Y \equiv \frac{4}{(n-1)^4 \lambda^2 T_3^2} \left( \frac{\varphi}{h} \right)^4 \,,
\end{equation}
% 
\begin{equation}
\label{eq:defXR2-multi}
\XR \equiv \frac{2}{(n-1)^2 h^4 T_3}X \,,
\end{equation} 
% 
and we have effectively imposed the relativistic condition 
% 
\begin{equation}
\label{eq:rellimit-multi}
X \simeq \frac{1}{2} T_3 h^4 \,,
\end{equation}
% 
in the numerator of \eq{eq:unifiedgenP-multi}.  
For the $n=2$ and $n=3$ cases, we have verified by direct 
calculation that when one calculates the speed of sound 
(\ref{eq:defcs-dbiintro}) and the non-linearity parameter 
(\ref{eq:fnldefn-dbiintro}) from the general expressions (\ref{eq:2brane}) and (\ref{eq:oddbrane}) 
and then imposes the relativistic
limit (\ref{eq:rellimit-multi}), one arrives at the identical result 
by starting explicitly with \eq{eq:unifiedP-multi}.


At this point we should consider the validity of 
the function in \eq{eq:unifiedP-multi}. Using the
recursion relations defined in the previous section, 
we see that in the large $n$ limit the kinetic function converges to 
the corresponding function defined in the large $n$ 
limit in \Rref{thomasward}. This is not the same function as that for
$n$ separated branes, as the matrix degrees of freedom 
lead to an additional potential term for the scalars. However it does belong
to the same class of models with $l=1/2$. We have 
verified this convergence numerically since the algebraic sums are
unfortunately not tractable. The key point is that there must exist some 
value of $n$, beyond which the function appears to look
more like the standard DBI action, rather than the 
approximate form proposed in (\ref{eq:unifiedP-multi}). For a range of background 
solutions, numerical calculations suggest that the approximation is 
valid up to terms of $\mathcal{O}(10)$ \cite{hltw}. Since there are a large
number of 
parameters in the theory, it is possible to find solutions 
where $n \gg 10$. However we will then be forced to 
generate a larger background flux, which will result 
in a situation where even the conformal 
Calabi-Yau condition is no longer valid. In view of this, we focus on
the sector of the theory where $n \le 10$, which implies that the 
backreaction is under control and that the kinetic function
is still of the required form. 


\eq{eq:unifiedP-multi} is precisely of the form given by the 
general solution (\ref{eq:gensoln-multi}), where $l=-1/2$ and
\footnote{This is the case $\alpha =18/5$ or $s=-1/3$ in the analytic solution
\eqref{eq:thirdint-multi} which after redefinition of the $f_i (\varphi)$
becomes:
% 
\begin{equation}
P = \frac{-f_1\left[8 - 4f_2X^{1/3}
-\left(f_2X^{1/3}\right)^2\right]}{\sqrt{1-f_2X^{1/3}}} -f_3 \, .
\end{equation}
% 
This expression appears in a slightly different 
form to that in (\ref{eq:unifiedgenP-multi}). 
However in deriving (\ref{eq:unifiedgenP-multi}) we assumed the
relativistic limit, which in turn imposes a non-trivial 
relation between $X$ and $\varphi$. Using this, and with a 
suitable redefinition of the functions, we can
transform the above expression into the required form.} 
% 
\begin{equation}
\label{eq:fdefns-multi}
f_1 (\varphi) = -2T_3 h^4 \sqrt{1+(n-1)^2Y} , \qquad 
f_2 (\varphi) = \frac{2}{T_3 h^4} \,.
\end{equation}
% 
We may therefore immediately conclude from \eq{eq:consequence4-multi} that
$\fnleq
\simeq -0.3/c_s^2$. Moreover, since $\beta \simeq 0.9$ in this scenario, 
Eqs.~\eqref{eq:consequence-multi} and \eqref{eq:consequence3-multi} reduce to  
% 
\begin{equation}
\label{eq:csPXlimit}
c_sP_{,X} \simeq -1.3 \sqrt{1+(n-1)^2Y} \fnleq \,.
\end{equation}
% 


We first consider the bound in \eq{eq:LHbound}. This applies at least for all
UV scenarios. It follows after substitution of the relativistic limit
(\ref{eq:rellimit-multi}) into the scalar perturbation amplitude,
\eq{eq:Ps-dbiintro},
that 
% 
\begin{equation}
\label{eq:usefulPs-multi}
\Pr \simeq -\frac{1}{50} \frac{H^4}{T_3 h^4\sqrt{1+(n-1)^2Y}}
\frac{1}{\fnleq}\,.
\end{equation}
% 
Substituting the tensor-scalar ratio (\ref{eq:rdefn-dbiintro}) into  
\eq{eq:usefulPs-multi} then results in a constraint on the magnitude of 
the warp factor during observable inflation:
%  
\begin{equation}
\label{eq:warpfactor-multi}
\frac{h^4_*}{\Mpl^4} \simeq \frac{-1}{2 T_3 \sqrt{1+(n-1)^2Y}} 
\frac{r^2 \Pr}{\fnleq} \,.
\end{equation}
% 
Eqs.~\eqref{eq:csPXlimit} and \eqref{eq:warpfactor-multi} may now be
substituted into 
the bound \eqref{eq:LHbound} to yield
%  
\begin{equation}
\label{eq:actualLH-multi}
r_* < \frac{1100}{(\Delta \N )_*^6} 
\frac{[1+(n-1)^2Y]}{\Vol} \Pr (\fnleq)^2 \,.
\end{equation}
% 


It is clear that the parameter $Y$ 
must be sufficiently large if the tensor perturbations 
are to be non-negligible. For the $AdS_5 \times X_5$ throat, this parameter  
takes the constant value    
% 
\begin{equation}
\label{eq:YAds-multi}
\YAdS \equiv \frac{4\pi^2 g_s N}{(n-1)^4 \Vol} \,.
\end{equation}
% 
As before, we chose natural field-theoretic values for the volume, 
$\Vol \simeq \pi^3$, and the string coupling, 
$g_s \simeq 10^{-2}$, and further assume that 
$(n-1)^2 Y \gg 1$. 
We again assume that the tensor-scalar ratio should not change significantly
over the entire range of scales that are accessible to cosmological
observation, which corresponds to $\Delta \N_* \simeq 4$. 
After substitution of the above values, therefore, 
the bound (\ref{eq:actualLH-multi}) simplifies to
%  
\begin{equation}
\label{eq:AdSupper-multi}
r_* < 2.8 \times 10^{-13} \frac{N}{(n-1)^2} (\fnleq)^2 \,.
\end{equation}
% 

As in Section~\ref{sec:twlargen-multi},
global tadpole cancellation constrains the magnitude of
the background charge $N$ such that $N < \chi /
24$. 
The maximal known value of the Euler number implies the upper limit of 
% 
\begin{equation}
\label{eq:Nlimit-multi} 
N < 75852
\end{equation}
% 
for known solutions, although in principle higher values are possible. 
Imposing the WMAP5 bound $\fnleq>-151$ in \eq{eq:AdSupper-multi}
and noting that $n \ge 2$ for consistency then implies an absolute
upper limit 
on the tensor-scalar ratio:
%  
\begin{equation}
\label{eq:rupper-multi}
r_* < 5 \times 10^{-4} \, .
\end{equation}
% 


This limit is below the sensitivity of the Planck satellite 
$(r \gsim 0.02 )$ \cite{planck}. On the other hand, 
the projected sensitivity of future CMB polarization experiments 
indicates that a background of primordial 
gravitational waves with $r_* \gsim 10^{-4}$ 
should be observable \cite{songknox,vpj, Baumann:2008aq}. In view of this, 
it is interesting to consider whether
a detectable gravitational wave background could in principle 
be generated in this class of multi-brane inflationary 
models. We find from \eq{eq:AdSupper-multi} that this would require 
% 
\begin{equation}
\label{eq:nlimit-multi}
n < 1 -5.3 \times 10^{-5} \sqrt{N} \fnleq < 1-0.014 \fnleq \,,
\end{equation}
% 
where the theoretic limit \eq{eq:Nlimit-multi} for 
known compactifications has been imposed in the 
second inequality. We may deduce, therefore, that  
since we require $n \ge 2$ for consistency, a detectable tensor 
signal will require $\fnleq < -70$. This implies that an observation of 
the tensors should also be 
accompanied by a sufficiently large -- and detectable -- non-Gaussianity. 
In other words, this class of models could  
be ruled out if tensors are observed in the absence of any
non-Gaussianity. On the other hand, the current 
limit of  $\fnleq >-151$ implies that $n \le 3$ is required 
for the tensors to be observable. 
Consequently, if tensor perturbations are detected, this would rule 
out all models with $n \ge  4$ or, alternatively, would require presently 
unknown configurations with $N$ exceeding bound (\ref{eq:Nlimit-multi}). 


In the above analysis we assumed that the string coupling 
took the value $g_s \simeq 10^{-2}$. For the $AdS_5 \times X_5$ throat, 
the bound (\ref{eq:actualLH-multi}) depends proportionally on $g_s$ and can 
therefore be weakened by allowing for larger values of the string coupling. 
For example, increasing this parameter by a factor of $4$ 
to $g_s \simeq 0.04$ (so that it is still in the perturbative regime)
relaxes the limit on the number of branes for the tensors to be detectable to 
$n \le 5$. Similarly, considering a smaller value for the 
volume of the Einstein manifold $X_5$ will also weaken the upper limit. 


Let us re-iterate that this limit on $n$ is well within the 
regime of validity for the theory, which we have argued is 
self-consistent for $n<10$. Moreover since the constraint (\ref{eq:nlimit-multi})
arises using the absolute maximal bound on the known 
Euler characteristics, it suggests that in realistic scenarios $n$ will 
always be much smaller than this. Indeed, one could argue that 
only the $n=2$ and $n=3$ theories are
likely to be valid over a large distribution of the flux landscape. 


We must also ensure that our approximation $(n-1)^2Y \gg 1$ 
is valid for consistency.  
For the parameter values we have chosen this requires that 
$g_s N \gg  (n-1)^2$ 
and this is satisfied if the condition \eqref{eq:nlimit-multi} 
holds. Note also that we require $N \gg n$ for the supergravity 
approximation to be under control and for backreaction effects to 
be negligible. This is also satisfied when \eq{eq:nlimit-multi}  
holds. 


For completeness we should also consider the 
BM bound \eqref{eq:genBMbound} for this class of models. This is given by 
% 
\begin{equation}
\label{eq:BMAdS-multi}
r_* < -\frac{42}{N \Neff^2}\sqrt{1 +(n-1)^2Y}\fnleq \,,
\end{equation}
%  
and in the case of an $AdS_5 \times X_5$ throat simplifies to
%  
\begin{equation}
\label{eq:bmadsbound}
r_* < -\frac{5}{\Neff^2} 
\frac{\fnleq}{(n-1)\sqrt{N}} \,.
\end{equation}
%  
Comparing the limits in Eqs.~\eqref{eq:AdSupper-multi}) and
\eqref{eq:bmadsbound} 
implies that the bound \eqref{eq:LHbound} is stronger than the corresponding BM
bound \eqref{eq:genBMbound} if 
% 
\begin{equation}
\label{eq:LHstrongerads}
n > 1 -5.5 \times 10^{-14} N^{3/2} \Neff^2 \fnleq \,,
\end{equation}
% 
and this condition is always satisfied if 
% 
\begin{equation}
\label{eq:allNbound-multi}
-5.5 \times 10^{-14} N^{3/2} \Neff^2 \fnleq  <1  \, .
\end{equation}
% 
Moreover, the bound \eqref{eq:allNbound-multi} will itself be satisfied for 
all values of $\fnleq$ and $N$ if it is satisfied when the limits 
$\fnleq =-151$ and $N=75852$ are imposed. Hence, we conclude that the bound
\eqref{eq:LHbound} 
is stronger for $\Neff < 75$. 
In general, it is difficult to quantify 
the magnitude of $\Neff$ without 
imposing further restrictions on the parameters of the models 
and, in particular, on the functional form of the inflaton potential. 
However, if the ratio $\epsilon_H/P_{,X}$ remains approximately 
constant during the final stages of inflation, one would anticipate that 
$\Neff \lsim 60$. Nevertheless, if $N \ll 75852$, the bound 
\eqref{eq:LHstrongerads} will only be violated for $n \le 3$ if 
$\Neff \gg 60$.  


Finally, it should be emphasized that the derivation of the bound in
\eq{eq:LHbound} 
underestimates the Planck mass by assuming that 
the volume of the throat is much smaller 
than the volume of the compactified Calabi-Yau 
three-fold. It is likely, therefore, 
that the actual constraint on $r$ would be much stronger. Consequently, 
although the bound \eqref{eq:nlimit-multi}  
does marginally allow for detectable tensors if $n$ is sufficiently 
small, in practice this constraint would be further tightened by a more 
complete calculation. Nonetheless, our analysis does not necessarily 
rule out these models as viable candidates for inflation. Rather, it  
suggests that it will be difficult to construct a working model 
that results in a detectable tensor signal.   


% 
% 
% 
% % % % % % % % % % % % % % % % % % % % % % % % % % % % % % % % 
% =========================================================== %
% % % % % % % % % % % % % % % % % % % % % % % % % % % % % % % % 
\section{Discussion}
\label{sec:disc-multi}
% % % % % % % % % % % % % % % % % % % % % % % % % % % % % % % % 
% =========================================================== %
% % % % % % % % % % % % % % % % % % % % % % % % % % % % % % % % 


The relativistic DBI brane scenario represents an attractive, 
string-inspired realisation of the inflationary scenario. In
Chapter~\ref{ch:dbi} we showed that
cosmological data has placed very strong constraints on the simplest 
models based on a single ${\rm D3}$-brane. The strength 
of these constraints follows from field-theoretic upper limits 
on the tensor-scalar ratio, $r$, which in turn arise because 
the effective DBI action satisfies special  
algebraic properties. This provides motivation 
for considering generalisations of the scenario, in particular to 
multi-brane configurations. 


In this chapter we have identified a phenomenological class of 
effective actions for which the constraints 
on $r$ are relaxed if significant (and detectable) 
non-Gaussian curvature perturbations are generated during inflation. 
We have provided approximate and exact derivations of this class which coincide
in the relativistic limit. It would be interesting to 
investigate whether the effective action (\ref{eq:gensoln-multi}) with values
of $l \ne - 1/2$ arises in string-inspired settings or elsewhere.

In Section~\ref{sec:multibranes-multi} we introduced the coincident $n$ branes
model of Thomas and Ward \cite{thomasward}. We examined the predictions of this
model in two limits, arbitrarily large $n$ and small finite $n$. The large $n$
model
is similar to the single brane case. Using the results of
Section~\ref{sec:relaxing-dbi} we showed that it is strongly
constrained by current observations.

The finite $n$ model is of more theoretical interest as it exhibits the
non-Abelian nature of the scenario. In \Rref{hltw} a recursive approach was
derived to calculate the pressure and energy densities for $n>2$ models using
the $n=2$ results. In the relativistic limit these finite $n$ models are
included in the class of actions derived in Section~\ref{sec:relaxing-multi}
which relax the bounds on $r$. The backreaction of these models can also be
kept well under control.

We proceeded to consider the question of whether the upper limits on 
$r$ could be relaxed to such an extent 
that a background of primordial gravitational waves 
might be detectable in future CMB experiments. The vast majority of 
string-inspired inflationary models that have been proposed to date 
generate an unobservable tensor background. We 
found that a detectable signal is possible, in principle, 
for typical string-theoretic parameter values 
if the number of coincident branes is either $2$~or~$3$. 
This is consistent with known F-theory configurations and 
current WMAP5 limits on the non-Gaussianity. Furthermore, 
we found that the level of non-Gaussianity must exceed $\fnleq 
\lsim -70$ if such configurations are to generate a detectable tensor 
signal. This is well within the projected sensitivity 
of the Planck satellite \cite{planck}.   


Our analysis invoked an $AdS_5 \times X_5$ warped throat geometry. However we 
made no assumptions regarding the form of the inflaton potential, other 
than imposing the implicit requirement that the universe underwent a phase of 
quasi-exponential expansion. In this sense, therefore, we have 
yet to explicitly establish that these inflationary models will 
be able to generate a measurable tensor signal. 
Nonetheless, since such a detection would provide a unique observational 
window into high energy physics, our results 
provide strong motivation for considering the cosmological consequences 
of these multi-brane configurations further when specific choices for the 
inflaton potential are made. In particular, it would be interesting 
to employ the techniques developed in
\cite{bean,Peiris:2007gz,Lorenz:2007ze,Bean:2007eh,Bean:2008ga} 
to identify the ranges of parameter space that are consistent 
with current cosmological observations.  

In Part~\ref{part:dbi} of this work we have concentrated on using theoretical
techniques to constrain inflationary models. In Chapter~\ref{ch:dbi} we showed
how the single-brane UV DBI model is subject to two incompatible bounds on the
tensor-scalar ratio. In Chapter~\ref{ch:multibrane} we described a class of
actions
which would ease those bounds. We examined the coincident $n$ branes model and
showed that the scenario in which an observable tensor signal is produced is
strongly constrained by current observations. In Part~\ref{part:numerical}
numerical techniques will be developed with the goal of constraining
inflationary models using perturbation theory.

