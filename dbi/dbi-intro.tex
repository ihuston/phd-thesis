% % % % % % % % % % % % % % % % % % % % % % 
% dbi-intro.tex - Ian Huston
% $Id: dbi-intro.tex,v 1.26 2009/08/04 22:18:56 ith Exp $
% % % % % % % % % % % % % % % % % % % % % % 
% Redefine CVSRevision for this section
\renewcommand{\CVSrevision}{\version$Id: dbi-intro.tex,v 1.26 2009/08/04 22:18:56 ith Exp $}
% % % % % % % % % % % % % % % % % % % % % % % % % % % % % % % % 
% =========================================================== %
% % % % % % % % % % % % % % % % % % % % % % % % % % % % % % % % 
\chapter{Introduction to Dirac Born Infeld Inflation}
\label{ch:dbi-intro}
% % % % % % % % % % % % % % % % % % % % % % % % % % % % % % % % 
% =========================================================== %
% % % % % % % % % % % % % % % % % % % % % % % % % % % % % % % % 

% % % % % % % % % % % % % % % % % % % % % % % % % % % % % % % % 
% =========================================================== %
% % % % % % % % % % % % % % % % % % % % % % % % % % % % % % % % 
\section{Introduction}
% 
\label{sec:dbi-intro}
% % % % % % % % % % % % % % % % % % % % % % % % % % % % % % % % 
% =========================================================== %
% % % % % % % % % % % % % % % % % % % % % % % % % % % % % % % % 


The inflationary scenario provides the 
theoretical framework for the early history 
of the universe. It has now been successfully tested by observations, 
including the five year data from the Wilkinson Microwave Anisotropy 
Probe (WMAP5) \cite{Komatsu:2008hk}. Despite this success, however, the high energy 
physics that resulted in a phase of accelerated expansion is still 
not well understood. String/M-theory attempts to unify the fundamental interactions 
including gravity and it is natural and important to 
develop inflationary models within string theory and to confront them with 
cosmological observations. 
 

One class of string-theoretic models that has received 
considerable attention is D-brane inflation
\cite{brane1,brane2,brane3,brane4,brane5,
brane6,brane7,brane8,brane9,brane10,brane11,brane12,brane13,
brane14,brane15,brane16,brane17,Brodie:2003qv,Vikman:2006hk, 
Mukhanov:2005bu,Kallosh:2007wm,brane18,
brane19,brane20,brane21}. 
(For recent reviews, see, e.g., \cite{tyereview,cline,McAllister:2007bg,Lorenz:2007ze,
Bean:2007eh,bean}). 
% Of particular interest 
% is the Dirac-Born-Infeld (DBI) inflationary scenario \cite{brane6,brane11}.
The Dirac-Born-Infeld (DBI) scenario 
of the compactified type IIB theory is a well-motivated model \cite{brane6,brane11}, 
in which inflation is driven by one or more ${\rm D}$-branes 
propagating in a warped `throat' background. 
% 
In the simplest version of the scenario, 
the inflaton parametrizes the radial 
position in the throat of a single ${\rm D3}$-brane. 
The brane dynamics are determined by the DBI action in such a 
way that the inflaton's kinetic energy is bounded from above by the warped 
brane tension. The regime where this bound is nearly saturated is 
known as the `relativistic' limit.

% % % % % % % % % % % % % % % % % % % % % % % % % % % % % % % 
% ========================================================= %
% % % % % % % % % % % % % % % % % % % % % % % % % % % % % % %
% Intro from multibrane paper, removed but could be useful.
% % % % % % % % % % % % % % % % % % % % % % % % % % % % % % % % 
% The quest to realise inflation within string/M-theory continues to 
% attract considerable attention. The Dirac-Born-Infeld (DBI) scenario 
% of the compactified type IIB theory is a well-motivated model, 
% in which inflation is driven by one or more ${\rm D}$-branes 
% propagating in a warped `throat' background
% \cite{brane1,brane11,brane12,brane13, brane2,brane20,brane3, brane4,brane5,brane6}
% . Such a background is generated 
% by the non-trivial form-field fluxes over the internal dimensions. 
% (For recent reviews, see
% \cite{tyereview,McAllister:2007bg,Lorenz:2007ze,Kallosh:2007wm,Bean:2007eh,bean,
% cline}.) 
% In the simplest version of the scenario, 
% the inflaton parametrizes the radial 
% position in the throat of a single ${\rm D3}$-brane. 
% % % % % % % % % % % % % % % % % % % % % % % % % % % % % % % % 





In Chapters \ref{ch:dbi-intro}, \ref{ch:dbi} and \ref{ch:multibrane} we will 
explore the observational consequences of DBI inflation. 
In general, primordial gravitational wave fluctuations
and non-Gaussian statistics in the curvature perturbation provide 
two powerful discriminants of inflationary scenarios. 
The nature of the DBI action is such that the sound 
speed of fluctuations in the inflaton can be much less than the speed of 
light. This induces a large and potentially detectable non-Gaussian 
signal in the density perturbations \cite{brane6,brane11,lidser3,chenetal}. 


In this chapter we introduce non-canonical inflation and DBI inflation.
In Chapter \ref{ch:dbi} we will derive upper and lower 
limits on the amplitude of the tensor perturbations in Sections \ref{sec:upper-dbi} and  
\ref{sec:lower-dbi} respectively.  
We will explore how these bounds may be relaxed in Chapter \ref{ch:multibrane} and discuss multi-brane 
scenarios which permit observational tensor signals. 

% % % % % % % % % % % % % % % % % % % % % % % % % % % % % % % % 
% =========================================================== %
% % % % % % % % % % % % % % % % % % % % % % % % % % % % % % % % 
\section{String theory and extra dimensions}
\label{sec:extradims}
% % % % % % % % % % % % % % % % % % % % % % % % % % % % % % % % 
% =========================================================== %
% % % % % % % % % % % % % % % % % % % % % % % % % % % % % % % % 
Unification of seemingly disparate theories has been a driving force in
theoretical physics for more than 100 years. This effort culminated in the
Standard Model (SM) of particle physics which unifies three of the four
fundamental forces in a robust theoretical framework. Since the realization of
the SM a clear goal of theoretical physics has been the unification of the
final force, gravity, into this framework. In this section we will introduce
some of the concepts that will be required later to understand DBI inflation.
\addtodo{Include more citations}
Many review articles and text books have been about string theory and its
application to cosmology, a short list of recent works includes
\cite{Polchinski1996,cline,Johnson2000}.


String theory is one of the most successful current attempts to achieve the
unification of physics. According to this theory, the fundamental
building blocks of the
universe are vibrating bundles of energy. These ``strings'', and their analogues
in other dimensions, constitute all matter and force particles in existence.
There are two main
types of strings, closed and open, distinguished by the fact that closed
strings form a continuous loop while open strings have two unconnected ends. 
% The physics of their interactions dictates the physics of the SM which has
% been
% so successfully tested.
% 
% This at least is the goal of the string theory project but the complex nature
% of the mathematical and physical structures involved means that no single
% well-defined theory can claim to have achieved this goal. Indeed many
% different
% string theories have been discovered. This might appear to be a significant
% set-back in the attempt to find a single unifying theory. 


There are several different string theories linked in pairs by a process
called
duality. Physical descriptions in one theory can be translated into a
dual description in the other. The dual version often exhibits properties that
are useful for solving problems in the original setting.
% These dualities have demonstrated the existence of an underlying theory of
% which the individual string theories are only facets. 
We will work in the string theory which has proven
to be the most
fruitful for models of cosmological inflation, Type IIB. 
% 
% The most striking prediction of string theory is the existence of other
% dimensions beyond the four space-time dimensions that we
% experience. Recovering the physics of our familiar $3+1$ dimensional universe
% clearly requires that these extra dimensions be hidden in some way. How this
% is
% achieved is one of the signature features of any complete string theoretical
% description of the universe.

\subsection{Cosmology and String Theory}
Temperatures in the early univerese were far in excess of those attainable by
any earth-bound particle accelerator. Hence a stringent test of any high energy
theory is whether it predicts the same properties of the early universe that are
now being observed in the CMB and elsewhere.
For this reason any string theory description of the universe must include a
consistent cosmological narrative. 

In Chapter \ref{ch:introduction} inflation was put forward as the leading
explanation of the early development of the universe. For string theory to be
a consistent framework in the inflationary paradigm it must produce a period of
inflationary expansion. Inflation requires the input of physics beyond the SM
for its existence. Although usually described by SM scalar fields, there is no
inherent reason for the existence of such inflaton fields in the SM. Higher
energy theories such as string theory could provide an explanation for the
existence of inflaton fields. In this way inflation and string theory are
ideal partners and it is natural to consider them together when investigating
the early universe.

\addtodo{Add paras on constraining string theory with cosmology, genericity
argument, unexpected signals. Tie to introduction chapter and fnl as a
discriminator.}

\subsection{Extra dimensions}
One of the basics results of string theory is that the 3 spacial dimensions that we see
around us are not the full extent of the space-time manifold. Instead
we live in a ten or
eleven dimensional space-time and physical theories in 3+1 dimensions must
compensate for the
other 6 or 7 dimensions that are hidden from us. One of the challenges of string theory is how 
to hide these extra dimensions in such a way as to recover standard physical
results in the usual
ones. 

The early work of Kaluza and Klein in formulating higher dimensional 
theories laid the groundwork for the current treatment of extra dimensions in
string theory \cite{Kaluza1921, Klein1926}. They found that by
compactifying an extra dimension onto a circle of finite radius an infinite
tower of extra fields would be introduced into the lower dimensional theory. The
mass of these fields is inversely proportional to the size of the extra
dimension. By taking the radius to be extremely small the appearance of these
unobserved massive fields is avoided. This leaves a
massless degree of freedom which must be accounted for in the four-dimensional
effective action. 


A similar procedure is undertaken when compactifying string theory from a
10 or 11 dimensional description down to four dimensions (for reviews see
\cite{douglas,grana}).
In ten dimensional type IIB theory the six extra dimensions are
compactified into a Ricci flat Calabi-Yau (CY) manifold which can be described
by three complex coordinates \cite{Yau1977}. 
Because any Ricci flat metric can be
rescaled into another Ricci flat metric, there is no unique solution for the
metric on the CY manifold. Instead a family of solutions exists with many free
parameters. These remain after the compactification, in
analogy to the size of the extra dimension in KK compactification, and can vary
dependent on position in the four-dimensional space-time. They appear as fields
in the four-dimensional theory and are known as moduli. 
These fields are not subject to any symmetry and so their individual values
at different spacetime points can affect the physics at those points.





% Suppose we wish to reduce a theory in $(D+1)$ dimensions to one in
% $D$ dimensions. 
% We will ``compactify'' the 
% (here single) extra dimension on a circle of finite radius $L$ and
% denote the coordinate on the circle as $z$. 
% Assume that in $(D+1)$ 
% dimensions we have a metric tensor $\hat{g}_{MN}(x,z)$ where $M$ and
% $N$ run over all $(D+1)$ dimensions and 
% x signifies the $D$ coordinates of the lower-dimensional spacetime.
% Hats  denote quantities in the $(D+1)$ space-time.
% A Fourier decomposition of $\hat{g}_{MN}$ gives an infinite number
% of fields in $D$ dimensions:
% \begin{equation}
%  \hat{g}_{MN}(x,z) = \sum_n g_{MN}^{(n)}(x) e^{i\frac{nz}{L}}\,.
% \end{equation}
% The mode number $n$ is related to the mass of the resulting fields;
% modes with $n=0$ will be massless and those with
% $n \ne 0$ massive. This can be seen by considering the Klein-Gordon
% equation for a $(D+1)$ dimensional field $\hat{\varphi}$:


\subsection{T-duality}
\label{sec:tduality-dbiintro}
% T-Duality
Suppose we have a string theory compactified on a circle of radius $R$. Instead
of being a continuum, the momentum of a closed string takes discrete values
$n/R$ for $n \in \mathbb{Z}$. This is a KK tower of massive states. As we
complete a circuit around the compact dimension, the value of the coordinate
function embedding the string in the background will also increase by $2\pi w
R$ for $w \in \mathbb{Z}$. This $w$, called the winding number of the string,
can only be non-zero for closed strings which can be wrapped around the periodic
dimension.

The total mass of the string contains terms with both the KK tower of states
and the new tower of winding states:
\begin{equation}
\label{eq:closedmass-dbiintro}
 M^2 = \frac{n^2}{R^2} + \frac{w^2 R^2}{\alpha'^2} + \cdots\,.
\end{equation}
If $R$ is taken to infinity the $w\ne 0$ states become infinitely massive and
only the $w=0$ state is left with a continuum of momentum values. Thus the
uncompactified result is recovered. However if $R\rightarrow0$ the $n\ne 0$
states are now infinitely massive as in the standard KK picture. Unlike that
scenario, there is now a continuum of winding states $w\ne 0$, again giving an
uncompactified dimension. This major departure from the standard
compactification result is a purely stringy effect. 

The formula for the mass spectrum, \eq{eq:closedmass-dbiintro}, is invariant
when $n$ and $w$ are exchanged given the transformation
\begin{equation}
 R \rightarrow R' = \frac{\alpha'}{R}\,.
\end{equation}
The new theory which is compactified on a circle of radius $R'$ is known as the
T-dual theory and the transformation is called T-duality
\cite{Sakai1986,Kikkawa1984b}. The two
theories,
original and T-dual, are physically identical; T-duality is an exact symmetry
of string theory for closed strings.

% 
\subsection{D-branes}
\label{sec:dbranes-dbiintro}
% Open string + T-duality = Dirichlet boundary conditions
% 
% Mixed Dirichlet and von Neumann boundary conditions.
% 
% D-p branes, open strings end on p+1 ? dimensional manifold.
% 
% For Type IIB we have D-1,3,5,7 branes
% 
% See Johnson p80 for D-brane action.
% 
% Brane inflation was first considered without moduli stabilization (pg21
% Douglas
% and Kachru for refs). This gave a potential similar to hybrid inflation with
% waterfall field. KKLMMT argued that the corrections from including moduli
% stabilization were too big to ignore.
Despite using string theory we will not be dealing with strings themselves.
Instead we will consider the dynamics of extended objects known as branes and
in particular we will use Dirichlet- or D-branes. As mentioned above string
theories are linked by T-duality. Fundamental parameters are translated by the
duality including the size of the extra-dimensions, the string coupling and the
coordinate solutions of the strings.


In Section \ref{sec:tduality-dbiintro} we introduced T-duality by explaining
its effects on closed strings. But what happens when open strings are
T-dualised? Open strings, as their name suggests, have two open ends and
consequently cannot have a conserved winding number like $w$. Again suppose one
of the $D$ dimensions is compactified. As $R\rightarrow0$ the non-zero momentum
states get infinitely massive but in contrast to the closed case there is now
no continuum of winding states. Thus the theory now lives in one less
dimension similar to standard KK compactification.

There are still closed strings in this theory however, and these continue to
move in the full $D$ dimensions. The endpoints of the open strings now observe
Dirichlet boundary conditions, taking fixed values in the compactified
direction. This follows through if more than one coordinate is made periodic.
If $D-p-1$ spatial dimensions are compactified then the ends of the open
strings ca still move freely in the other $p$ stpatial dimensions on a $p+1$
dimensional hypersurface called a Dirichlet or D$p$-brane. 
% 
In Type IIB theory with supersymmetric string an extra condition means that
only D$p$-branes with $p=1,3,5,7,9$ are allowed \footnotemark.
\footnotetext{There is also a $p=-1$ D-instanton in which the time direction
along with all spatial directions is subject to Dirichlet boundary conditions
\cite{Green1992,Green1988,Green1977}.}
% 

D$p$-branes can be considered as dynamical objects in their own right with a
tension\footnote{$T_p$ here is $\tau_p$ in \cite{Johnson2000}.}
% 
\begin{equation}
\label{eq:branetensiondefn-dbiintro}
 T_p = \frac{m_s^{p+1}}{(2\pi)^p g_s}\,,
\end{equation}
% 
and an action given by
% 
\begin{equation}
\label{eq:gendbiaction-dbiintro}
 S = -T_p \int d^{p+1}\xi \sqrt{-\hat{g}}\,,
\end{equation}
where $\hat{g}_{ab}$ is the induced metric on the brane with internal
coordinates $\xi^a$ for $a=0,\ldots,p$. This is the general form of the DBI
action which will be used later.

\subsection{D-brane inflation}
At its most basic slow-roll inflation requires only a single scalar field with a
sufficiently flat potential to satisfy the slow roll conditions outlined in
Section \ref{sec:slowroll-intro}. D-branes are charged (with Ramond-Ramond
charge) and we can consider a system of a D-brane and an anti-D-brane
($\overline{\mathrm{D}}$). As the two branes attract each other the distance
between them can be used as a decreasing scalar degree of freedom, the
inflaton \cite{brane1,brane2,brane3,brane7}. 
When the branes collide they will annihilate and produce energy which would
reheat the universe. 

As described in Section \ref{sec:extradims} compacitifying dimensions
introduces scalar fields known as moduli. These massless fields must be
accounted for in the dynamics unless some way is found to ``stabilize'' them by
fixing their mass to be large enough that they do not interfere.
Initial efforts to induce inflation using D-branes ignored the issue of moduli
stabilization. Instead they
assumed that whatever stabilization technique was used would have no discernable
effect on the inflationary physics [cite kklt***]. Kac

\subsection{Warped throats}
As mentioned above, moduli fields must
be stabilized, that is given an effective mass so that they do not interfere as
massless fields. This can be achieved by switching on background fluxes and is
equivalent to turning on a magnetic field in the compact space. Instead of the
manifold having zero total charge due to Gauss' theorem the integrated charge
will now be a non-zero integer.
In the presence of fluxes a general form of metric with a warp factor is
possible
\begin{equation}
\label{eq:warpedmetric-dbiintro}
 ds^2 = e^{2A(y)}\eta_{\mu\nu} dx^\mu dx^\nu + e^{-2A(y)} g_{mn}dy^m dy^n\,,
\end{equation}
where $A(y)$ varies across the compact dimensions $y^m$.
Compactifications in which $A$ varies significantly with $y$ are called warped
compactifications.

Warped compactifications are similar to the Randall-Sundrum picture.

Simplest realisation in string theory uses warped deformed conifolds (Klebanov
and Strassler). 
If fluxes are turned on and
\begin{equation}
 \int_A F_3 = M \,, \int_B H_3 = -K\,,
\end{equation}
then $e^A \simeq e^{-2\pi K/3g_s M}$ giving an exponential warping in the
throat.

In the bulk of the Calabi-Yau space $e^A \simeq 1$ so there is no warping. The
SM brane could live here and not feel any effects of the warping.



% 
% 
% % % % % % % % % % % % % % % % % % % % % % % % % % % % % % % % 
% =========================================================== %
% % % % % % % % % % % % % % % % % % % % % % % % % % % % % % % % 
\section{Non-Canonical Inflation} 
\label{sec:noncanoninfl}
% % % % % % % % % % % % % % % % % % % % % % % % % % % % % % % % 
% =========================================================== %
% % % % % % % % % % % % % % % % % % % % % % % % % % % % % % % % 


The low-energy, world-volume dynamics of a 
${\rm D3}$-brane in a warped background is determined 
by an effective action of the form 
% 
\begin{equation}
\label{eq:DBIaction-dbiintro2}
S=\int  d^4x \sqrt{|g|} \left[ \frac{\Mpl^2}{2} R 
+ P (\varphi , X) \right] \,,
\end{equation}
% 
where $R$ is the Ricci curvature scalar, 
$X \equiv -\frac{1}{2}g^{\mu \nu}\nabla_\mu \varphi \nabla_\nu \varphi$
denotes the kinetic energy of the inflaton field $\varphi$, and the function  
$P (\varphi , X)$ is referred to as the `kinetic function'.  


We assume that the four-dimensional universe is   
spatially flat and isotropic and sourced by an  
homogeneous inflaton field, $\varphi =\varphi (t)$, with energy 
density $E = 2X\PX - P$, where a subscripted comma denotes partial
differentiation. 
We further assume that the inflaton dynamics  
generates a quasi-exponential expansion of the universe 
where $\epsilon_H \equiv -\dot{H}/H^2 \ll1$,
as described in Section~\ref{sec:frw-intro}. 


It proves convenient to define two parameters in terms of the 
kinetic  function $P$ and its derivatives \cite{lidser1,lidser3}: 
% 
\begin{eqnarray}
\label{eq:defcs-dbiintro}
 c_s^2 \equiv \frac{\PX}{\PX + 2X P_{,XX}} \,,
\\
\label{eq:deflambda-dbiintro}
\Lambda \equiv  \frac{X^2 P_{,XX} +
\frac{2}{3}X^3 P_{,XXX}}{X P_{,X} +
2X^2 P_{,XX}}\,.
\end{eqnarray}
% 
The first parameter, $c_s$, determines the sound speed of fluctuations 
in the inflaton field. This can be significantly less than unity, 
in contrast to slow-roll inflation driven by a canonical 
field such that $P_{,X} =1$.
The amplitudes of the scalar and tensor perturbations 
generated during inflation change in this case and are given by \cite{gm}
% 
\begin{eqnarray} 
\label{eq:Ps-dbiintro}
 \Ps = \frac{H^4}{8\pi^2 X}\frac{1}{c_s \PX} \,,
\\
\label{eq:Pt-dbiintro}
\Pt = \frac{2}{\pi^2} \frac{H^2}{\Mpl^2} \,,
\end{eqnarray}
% 
respectively, and the ratio of these amplitudes 
is defined as \cite{gm} 
% 
\begin{equation}
\label{eq:rdefn-dbiintro}
r\equiv \frac{\Pt}{\Ps} = 16c_s \epsilon_H \,.
\end{equation}
%   
The WMAP5 normalization of the CMB power spectrum 
implies that $\Ps= 2.5\times10^{-9}$ and 
the experimental upper bound on the tensor-scalar 
ratio is $r <0.55$ \cite{Komatsu:2008hk}.
\addtodo{Change to r<0.20 for no-running or r<0.54 for running (WMAP5+BAO+SN).
Does this change any later calculations?}

Deviations from Gaussian statistics in the curvature perturbation, ${\cal{R}}$,
are parametrized in terms of the non-linearity parameter, 
$\fnl$, as defined in Section~\ref{sec:fnl-intro} by 
${\cal{R}} = {\cal{R}}_G + \frac{3}{5} \fnl  (
{\cal{R}}_G^2 -\langle {\cal{R}}_G^2 \rangle )$. Here the 
quadratic component represents a convolution and 
${\cal{R}}_G$ denotes the Gaussian contribution \cite{maldacena}\footnotemark.
\footnotetext{We use the WMAP sign convention for $\fnl$ throughout. 
This is the opposite of the Maldacena convention: $\fnl^\mathrm{WMAP}=-\fnl^\mathrm{Maldacena}$.}
 In the limit  
where the three momenta have equal magnitude (corresponding to the equilateral  
triangle limit), the leading-order contribution to the non-linearity 
parameter is given by \cite{chenetal,lidser3}
% 
\begin{equation} 
\label{eq:fnldefn-dbiintro}
 \fnl = -\frac{35}{108}\left(\frac{1}{c_s^2} -1 \right) +
\frac{5}{81}\left( \frac{1}{c_s^2} -1 -2\Lambda \right) \,.
\end{equation}
%  
One should note that the sign convention is that employed
by the WMAP data set.
Data from WMAP3 imposes the bound $|\fnl| < 300$ on this parameter
\cite{spergel}. The corresponding bounds for other triangle configurations 
may be much tighter than this and this may be particularly relevant if 
non-Gaussian signatures have indeed been detected in the 
CMB \cite{Yadav:2007yy,crim}. The more recent WMAP5 data set
\cite{Komatsu:2008hk} improves on this bound somewhat, and
also indicates that it is distinctly asymmetric. At the $95 \%$ confidence level, the bound on the 
equilateral triangle becomes $-151 < \fnl < 253$.


% 
% 
% % % % % % % % % % % % % % % % % % % % % % % % % % % % % % % % 
% =========================================================== %
% % % % % % % % % % % % % % % % % % % % % % % % % % % % % % % % 
\section{DBI inflation} 
% 
\label{sec:dbiinflation}
% % % % % % % % % % % % % % % % % % % % % % % % % % % % % % % % 
% =========================================================== %
% % % % % % % % % % % % % % % % % % % % % % % % % % % % % % % % 
\addtodo{ More introduction of type IIB theory}
We will now concentrate on one particular non-canonical action. 
The DBI scenario is based on the compactification of type IIB string theory on a 
Calabi-Yau (CY) three-fold, where the form-field fluxes generate locally
warped regions known as `throats'.  The propagation of a 
${\rm D3}$-brane in such a region can drive inflation, where the inflaton 
field is identified with the radial position of the brane 
along the throat. Since this is an open string mode, the field 
equation for the inflaton is determined by a DBI action.

Flux compactification of type IIB string theory to four dimensions 
results in a warped geometry, where the six-dimensional CY  
manifold contains one or more throats (see \cite{grana,douglas} for reviews). 
The metric inside a throat has the generic form
% 
\begin{equation}
\label{eq:conemetric-dbiintro}
ds_{10}^2= h^2 ( \rho) ds_4^2 + h^{-2} (\rho ) 
\left( d\rho^2 +\rho^2 ds_{X_5}^2 \right) \,,
\end{equation}
%  
where the warp factor $h (\rho)$ is a function of the 
radial coordinate $\rho$ along the throat and $X_5$
is a Sasaki-Einstein five-manifold. 
In many cases, the ten-dimensional metric (\ref{eq:conemetric-dbiintro}) can be 
approximated locally by the geometry $AdS_5 \times X_5$, where the 
warp factor is given by $h=\rho /L$ and 
the radius of curvature of the $AdS_5$ space is defined by
%  
\begin{equation}
L^4 \equiv \frac{4\pi^4 g_s N}{{\rm Vol} (X_5) m_s^4} \,,
\end{equation}
% 
such that $\mathrm{Vol}(X_5)$ is the dimensionless volume of 
$X_5$ with unit radius, $N$ is the ${\rm D3}$  charge of the throat, 
$g_s$ is the string coupling and $m_s$ is the string mass scale.
In the Klebanov-Strassler (KS) background \cite{ks}, the throat 
is a warped deformed conifold and 
corresponds to a cone over the manifold 
$X_5 = T^{1,1}= {\rm SU(2)} \otimes {\rm SU(2)}/{\rm U(1)}$
in the UV limit ($\rho \rightarrow \infty$). This has   
a volume $\mathrm{Vol} (T^{1,1}) = 16\pi^3/27$ and  topology
$S^2\times S^3$, where the $S^2$ is fibred over the $S^3$.
The conical singularity at the tip of the throat 
is smoothed out by an 
$S^3$ `cap' due to the wrapping of the fluxes along the cycles of the conifold
\cite{ks,kt} and the warp factor asymptotes to 
a constant value in this region.   
 

In general, the low-energy world-volume dynamics
of a probe ${\rm D3}$-brane in a warped throat is determined 
by an effective, four-dimensional DBI action. 
The inflaton field is related to the radial 
position of the brane by 
$\varphi \equiv \sqrt{T_3} \rho$, where $T_3 =m_s^4/[(2\pi )^3 g_s ]$ 
is the brane tension. The action is then given by 
% 
\begin{eqnarray}
\label{eq:DBIaction-dbiintro}
S=\int  d^4x \sqrt{|g|} \left[ \frac{\Mpl^2}{2} R 
+ P (\varphi , X) \right] \\
\label{defP}
P( \varphi ,X) = - T (\varphi)  \sqrt{1 - 2T^{-1} (\varphi ) X}
+T (\varphi)  -V(\varphi)  \,,
\end{eqnarray}
% 
where $R$ is the Ricci curvature scalar,
$X \equiv - \frac{1}{2} g^{\mu\nu} \nabla_{\mu} \varphi \nabla_{\nu} \varphi$
is the kinetic energy of the inflaton, $V(\varphi )$ denotes 
the field's interaction 
potential and $T(\varphi ) = T_3 h^4 (\varphi )$
defines the warped brane tension. We refer to $P(\varphi , X)$ as the 
`kinetic function' for the inflaton. 


We consider a spatially flat and isotropic cosmology 
sourced by a homogeneous scalar field. 
In this case, the Friedmann equations for a monotonically 
varying inflaton can be expressed in the form \cite{brane6} 
% 
\begin{eqnarray}
\label{eq:Friedmann-dbiintro}
3 \Mpl^2 H^2(\varphi ) &=& V(\varphi ) -T(\varphi ) 
\left[ 1- \sqrt{1+4\Mpl^4T^{-1} H'^2} \right] \,,\\
\label{useful}
\dot{\varphi} &=& - \frac{2\Mpl^2H'}{\sqrt{1+4\Mpl^4 T^{-1} H'^2}} \, ,
\end{eqnarray}
% 
where $H=H(\varphi )$ represents the Hubble parameter
as a function of the field and a prime denotes $d/d\varphi$. 


An important consequence of the non-trivial kinetic structure 
of the DBI action is that the sound speed of fluctuations in the inflaton 
differs from unity:   
% 
\begin{equation}
\label{eq:csdefn-dbiintro}
c_s = \frac{1}{P_{,X}} = \sqrt{1 -2T^{-1}X}  \,,
\end{equation}
% 
where a subscripted comma denotes partial differentiation. 
Furthermore, the condition that the sound speed be real 
imposes an upper bound on the kinetic energy 
of the inflaton, $\dot{\varphi}^2 < T(\varphi)$, which 
is independent of the steepness of the potential.
The motion of the brane is said to be `relativistic' when this bound is 
close to saturation.
 

We now define the epoch that is directly 
accessible to cosmological observations as `observable inflation'. 
Depending on the reheating 
temperature, this occurred some 30 to 60 e-foldings before the end of 
inflation. We will assume that the slow roll conditions described in 
Section~\ref{sec:slowroll-intro} apply during observable inflation.
The definitions of the slow roll parameters described in that section 
change when $c_s$ is not unity and we
will include a third parameter $s$ which quantifies the rate of change of $c_s$.
The inflationary dynamics during this phase can  
be quantified in terms of these three parameters: 
% 
\begin{eqnarray}
\label{eq:epsdefn-dbiintro}
\epsilon_H \equiv -\frac{\dot{H}}{H^2}
= \frac{XP_{,X}}{\Mpl^2H^2} 
= \frac{2\Mpl^2}{\gamma} \left( \frac{H'}{H} \right)^2 \\
\label{eq:defeta-dbiintro}
\eta_H \equiv  \frac{2\Mpl^2}{\gamma}\frac{H''}{H} \\
\label{eq:defs-dbiintro}
s \equiv \frac{\dot{c_s}}{c_sH} 
= \frac{2\Mpl^2}{\gamma} \frac{H'}{H}\frac{\gamma'}{\gamma}  \,,
\end{eqnarray}
% 
where $\gamma \equiv 1/c_s$. 
We will assume that the `slow-roll' conditions 
$\{ \epsilon_H, |\eta_H | , |s | \}  \ll 1$ applied during observable inflation. 
In this regime, the amplitudes and spectral indices of the two-point functions 
for the scalar and tensor perturbations are given by \cite{gm}
% 
\begin{eqnarray}
\label{eq:spectra-dbiintro}
P_S^2= \frac{1}{8 \pi^2 \Mpl^2} \frac{H^2}{c_s \epsilon_H}
= \frac{H^4}{4\pi^2\dot{\varphi}^2}
\\
P_T^2 = \frac{2}{\pi^2} \frac{H^2}{\Mpl^2} 
\\
\label{indices}
1-n_s = 4 \epsilon_H -2\eta_H  +2s 
\\
 n_t = -2\epsilon_H  
\end{eqnarray}
% 
respectively, where the quantities on the right-hand sides are evaluated 
when the scale with comoving wavenumber $k=aH\gamma$ crossed 
the Hubble radius during inflation.  
\addtodo{ Talk about difference with standard k=aH and compare consistency relation with 
standard one.}
The tensor-scalar ratio, $r \equiv P_T^2/P_S^2$, is directly related to 
the tensor spectral index by \cite{gm}
% 
\begin{equation}
\label{eq:consistency-dbiintro}
r= -8c_sn_t = 16c_s \epsilon_H\,.
\end{equation}
%
Hence, a sound speed different to unity leads to a violation of the 
standard inflationary consistency equation, which might be 
detectable in the foreseeable future \cite{lidser1,lidser2}. 
\addtodo{WMAP5}
Recent observations of the CMB 
indicate that $P^2_S=2.5 \times 10^{-9}$ and $r < 0.55$ \cite{spergel}. 
The best-fit value for a constant spectral index is 
$n_s = 0.987^{+0.019}_{-0.037}$ if $r\ne 0$ is assumed as a prior. 
This is strengthened to 
$n_s = 0.948^{+0.015}_{-0.018}$ for a prior of $r = 0$ \cite{spergel}. 


A further important consequence of a small sound speed is that departures  
from purely Gaussian statistics may be large 
\cite{brane6,brane11,lidser3,chenetal}. 
\addtodo{Link to intro section on $\fnl$. Remove duplicate information. Add note about using equilateral in
next two chapters.}
It is conventional 
to quantify such deviations by expressing the non-Gaussian curvature 
perturbation ${\cal{R}}$ in the form 
${\cal{R}} ={\cal{R}}_G + \frac{3}{5}\fnl
({\cal{R}}^2_G -\langle {\cal{R}}^2_G \rangle)$, where 
${\cal{R}}_G$ represents the Gaussian contribution, 
the quadratic piece is a convolution and $\fnl$ defines 
the `non-linearity' parameter \cite{maldacena}. 
In general, this parameter is a function of the three momenta which 
form a triangle in Fourier space. However, in the limit where 
these momenta have equal magnitude, the non-linearity parameter 
is given to leading-order by \cite{chenetal,lidser2}  
% 
\begin{equation}
\label{eq:fnlcs-dbiintro}
\fnl = -\frac{1}{3} \left( \frac{1}{c_s^2} -1 \right) \,.
\end{equation}
% 
\addtodo{ Change to WMAP5 numbers}
Currently, WMAP3 constrains this parameter  
to be ${| \fnl |} \, {\lsim} \, {300}$ \cite{spergel,crim}.  


Finally, Eqs. (\ref{eq:csdefn-dbiintro}), (\ref{eq:spectra-dbiintro}) and (\ref{eq:fnlcs-dbiintro}) 
may be combined to constrain the warped brane tension 
during observable inflation: 
% 
\begin{equation}
\label{eq:obswarp-dbiintro}
\frac{T (\varphi_*)}{\Mpl^4}  = 
\frac{\pi^2}{16} r^2P_S^2 \left( 1-\frac{1}{3\fnl} \right) \,,
\end{equation}
% 
where a subscript `$*$' denotes that the quantity is to be evaluated 
at that epoch. 


\addtodo{ Change to chapter references}
In the following, we first consider the UV version of DBI inflation,
where the brane moves relativistically 
towards the tip of the throat. We will assume implicitly 
that the sound speed is sufficiently small to generate a non-Gaussianity with magnitude in 
excess of $|\fnl| \, {\gsim} \, 5$, since this is the projected 
sensitivity of the Planck satellite \cite{planck}. Furthermore, we consider   
an arbitrary warp factor and inflaton potential, 
subject only to the condition that a sufficiently long phase of 
quasi-exponential expansion can be maintained to solve the horizon problem of
the hot big bang model. 
% 
%




% % % % % % % % % % % % % % % % % % % % % % % % % % % % % % % % 
% =========================================================== %
% % % % % % % % % % % % % % % % % % % % % % % % % % % % % % % % 
\section{The Lyth bound}
\label{sec:lyth-dbiintro}
% % % % % % % % % % % % % % % % % % % % % % % % % % % % % % % % 
% =========================================================== %
% % % % % % % % % % % % % % % % % % % % % % % % % % % % % % % % 

Eqs. (\ref{eq:Ps-dbiintro}) and (\ref{eq:Pt-dbiintro}) imply 
that the variation of the inflaton field during inflation  
is related to the tensor-scalar amplitude by \cite{lyth,bmpaper}
% 
\begin{equation}
\label{eq:genlyth-dbiintro}
\frac{1}{\Mpl^2}
\left( \frac{d \varphi}{d \cal{N}} \right)^2 = \frac{r}{8 c_s P_{,X}} \,,
\end{equation}
% 
where ${\cal{N}} \equiv \int dt \, H$ denotes the number of e-foldings.
We will refer to the epoch of inflation that can be directly 
constrained by cosmological observations as 
`observable inflation' and will assume that this phase 
occurred when the brane was located within a 
throat region\footnote{We denote the values of all parameters 
evaluated during observable inflation by a subscript 
`$*$'.}. Observable inflation corresponds to no more than about 4 e-foldings  
of inflationary expansion, $\Delta {\cal{N}}_* \simeq 4$. 
The total variation in the inflaton field between the epoch of observable 
inflation and the end of inflation is then given by
% 
\begin{equation}
\label{eq:totalfield-dbiintro}
\frac{\Delta \varphi_{\rm inf}}{\Mpl} = 
\left( \frac{r}{8c_sP_{,X}} \right)_*^{1/2} {\cal{N}}_{\rm eff} \,,
\end{equation}
% 
where
% 
\begin{equation}
\label{eq:Neff-dbiintro}
{\cal{N}}_{\rm eff} \equiv \left( \frac{c_sP_{,X}}{r}\right)_*^{1/2}
\int_0^{{\cal{N}}_{\rm end}}  
\left( \frac{r}{c_sP_{,X}} \right)^{1/2} d {\cal{N}} \,.
\end{equation}
% 
If $r/(c_s P_{,X})$ varies 
sufficiently slowly during observable inflation, 
the corresponding change in the value of the inflaton  
field is given approximately by \cite{lyth,bmpaper}
% 
\begin{equation}
\label{eq:approxlyth-dbiintro}
\left( \frac{\Delta \varphi}{\Mpl} \right)_*^2 \simeq 
\frac{(\Delta {\cal{N}}_*)^2}{8} \left( \frac{r}{c_sP_{,X}} \right)_* \,.
\end{equation}
% 

