% % % % % % % % % % % % % % % % % % % % % % 
% dbi-intro.tex - Ian Huston
% $Id: dbi-intro.tex,v 1.16 2009/07/29 14:01:17 ith Exp $
% % % % % % % % % % % % % % % % % % % % % % 
% Redefine CVSRevision for this section
\renewcommand{\CVSrevision}{\version$Id: dbi-intro.tex,v 1.16 2009/07/29 14:01:17 ith Exp $}
% % % % % % % % % % % % % % % % % % % % % % % % % % % % % % % % 
% =========================================================== %
% % % % % % % % % % % % % % % % % % % % % % % % % % % % % % % % 
\chapter{Introduction to Dirac Born Infeld Inflation}
\label{ch:dbi-intro}
% % % % % % % % % % % % % % % % % % % % % % % % % % % % % % % % 
% =========================================================== %
% % % % % % % % % % % % % % % % % % % % % % % % % % % % % % % % 

% % % % % % % % % % % % % % % % % % % % % % % % % % % % % % % % 
% =========================================================== %
% % % % % % % % % % % % % % % % % % % % % % % % % % % % % % % % 
\section{Introduction}
% 
\label{sec:dbi-intro}
% % % % % % % % % % % % % % % % % % % % % % % % % % % % % % % % 
% =========================================================== %
% % % % % % % % % % % % % % % % % % % % % % % % % % % % % % % % 


The inflationary scenario provides the 
theoretical framework for the early history 
of the universe. It has now been successfully tested by observations, 
including the five year data from the Wilkinson Microwave Anisotropy 
Probe (WMAP5) \cite{Komatsu:2008hk}. Despite this success, however, the high energy 
physics that resulted in a phase of accelerated expansion is still 
not well understood. String/M-theory unifies the fundamental interactions 
including gravity and it is natural and important to 
develop inflationary models within string theory and to confront them with 
cosmological observations. 
 

One class of string-theoretic models that has received 
considerable attention is D-brane inflation
\cite{brane1,brane2,brane3,brane4,brane5,
brane6,brane7,brane8,brane9,brane10,brane11,brane12,brane13,
brane14,brane15,brane16,brane17,Brodie:2003qv,Vikman:2006hk, 
Mukhanov:2005bu,Kallosh:2007wm,brane18,
brane19,brane20,brane21}. 
(For recent reviews, see, e.g., \cite{tyereview,cline,McAllister:2007bg,Lorenz:2007ze,
Bean:2007eh,bean}). 
Of particular interest 
is the Dirac-Born-Infeld (DBI) inflationary scenario \cite{brane6,brane11}. 
This is based on the compactification of type IIB string theory on a 
Calabi-Yau (CY) three-fold, where the form-field fluxes generate locally
warped regions known as `throats'.  The propagation of a 
${\rm D3}$-brane in such a region can drive inflation, where the inflaton 
field is identified with the radial position of the brane 
along the throat. Since this is an open string mode, the field 
equation for the inflaton is determined by a DBI action.
% 
The brane dynamics are determined by the DBI action in such a 
way that the inflaton's kinetic energy is bounded from above by the warped 
brane tension. The regime where this bound is nearly saturated is 
known as the `relativistic' limit.

% % % % % % % % % % % % % % % % % % % % % % % % % % % % % % % 
% ========================================================= %
% % % % % % % % % % % % % % % % % % % % % % % % % % % % % % %
% Intro from multibrane paper, removed but could be useful.
% % % % % % % % % % % % % % % % % % % % % % % % % % % % % % % % 
% The quest to realise inflation within string/M-theory continues to 
% attract considerable attention. The Dirac-Born-Infeld (DBI) scenario 
% of the compactified type IIB theory is a well-motivated model, 
% in which inflation is driven by one or more ${\rm D}$-branes 
% propagating in a warped `throat' background
% \cite{brane1,brane11,brane12,brane13, brane2,brane20,brane3, brane4,brane5,brane6}
% . Such a background is generated 
% by the non-trivial form-field fluxes over the internal dimensions. 
% (For recent reviews, see
% \cite{tyereview,McAllister:2007bg,Lorenz:2007ze,Kallosh:2007wm,Bean:2007eh,bean,
% cline}.) 
% In the simplest version of the scenario, 
% the inflaton parametrizes the radial 
% position in the throat of a single ${\rm D3}$-brane. 
% % % % % % % % % % % % % % % % % % % % % % % % % % % % % % % % 





In Chapters \ref{ch:dbi-intro}, \ref{ch:dbi} and \ref{ch:multibrane} we will 
explore the observational consequences of DBI inflation. 
In general, primordial gravitational wave fluctuations
and non-Gaussian statistics in the curvature perturbation provide 
two powerful discriminants of inflationary scenarios. 
The nature of the DBI action is such that the sound 
speed of fluctuations in the inflaton can be much less than the speed of 
light. This induces a large and potentially detectable non-Gaussian 
signal in the density perturbations \cite{brane6,brane11,lidser3,chenetal}. 


In this chapter we introduce non-canonical inflation and DBI inflation.
In Chapter \ref{ch:dbi} we derive upper and lower 
limits on the amplitude of the tensor perturbations in Sections \ref{sec:upper-dbi} and  
\ref{sec:lower-dbi} respectively.  
We explore how these bounds may be relaxed in Chapter \ref{ch:multibrane} and discuss multi-brane 
scenarios which permit observational tensor signals. 


% 
% 
% % % % % % % % % % % % % % % % % % % % % % % % % % % % % % % % 
% =========================================================== %
% % % % % % % % % % % % % % % % % % % % % % % % % % % % % % % % 
\section{Non-Canonical Inflation} 
\label{sec:noncanoninfl}
% % % % % % % % % % % % % % % % % % % % % % % % % % % % % % % % 
% =========================================================== %
% % % % % % % % % % % % % % % % % % % % % % % % % % % % % % % % 


The low-energy, world-volume dynamics of a 
${\rm D3}$-brane in a warped background is determined 
by an effective action of the form 
% 
\begin{equation}
\label{eq:DBIaction-dbiintro2}
S=\int  d^4x \sqrt{|g|} \left[ \frac{\Mpl^2}{2} R 
+ P (\varphi , X) \right] \,,
\end{equation}
% 
where $R$ is the Ricci curvature scalar, 
$X \equiv -\frac{1}{2}g^{\mu \nu}\nabla_\mu \varphi \nabla_\nu \varphi$
denotes the kinetic energy of the inflaton field $\varphi$, and the function  
$P (\varphi , X)$ is referred to as the `kinetic function'.  


We assume that the four-dimensional universe is   
spatially flat and isotropic and sourced by an  
homogeneous inflaton field, $\varphi =\varphi (t)$, with energy 
density $E = 2X\PX - P$, where a subscripted comma denotes partial
differentiation. 
We further assume that the inflaton dynamics  
generates a quasi-exponential expansion of the universe 
where $\epsilon_H \equiv -\dot{H}/H^2 \ll1$,
as described in Section~\ref{sec:frw-intro}. 


It proves convenient to define two parameters in terms of the 
kinetic  function $P$ and its derivatives \cite{lidser1,lidser3}: 
% 
\begin{eqnarray}
\label{eq:defcs-dbiintro}
 c_s^2 \equiv \frac{\PX}{\PX + 2X P_{,XX}} \,,
\\
\label{eq:deflambda-dbiintro}
\Lambda \equiv  \frac{X^2 P_{,XX} +
\frac{2}{3}X^3 P_{,XXX}}{X P_{,X} +
2X^2 P_{,XX}}\,.
\end{eqnarray}
% 
The first parameter, $c_s$, determines the sound speed of fluctuations 
in the inflaton field. This can be significantly less than unity, 
in contrast to slow-roll inflation driven by a canonical 
field such that $P_{,X} =1$.
The amplitudes of the scalar and tensor perturbations 
generated during inflation change in this case and are given by \cite{gm}
% 
\begin{eqnarray} 
\label{eq:Ps-dbiintro}
 \Ps = \frac{H^4}{8\pi^2 X}\frac{1}{c_s \PX} \,,
\\
\label{eq:Pt-dbiintro}
\Pt = \frac{2}{\pi^2} \frac{H^2}{\Mpl^2} \,,
\end{eqnarray}
% 
respectively, and the ratio of these amplitudes 
is defined as \cite{gm} 
% 
\begin{equation}
\label{eq:rdefn-dbiintro}
r\equiv \frac{\Pt}{\Ps} = 16c_s \epsilon_H \,.
\end{equation}
%   
The WMAP5 normalization of the CMB power spectrum 
implies that $\Ps= 2.5\times10^{-9}$ and 
the experimental upper bound on the tensor-scalar 
ratio is $r <0.55$ \cite{spergel}.
\addtodo{Change to r<0.20 for no-running or r<0.54 for running (WMAP5+BAO+SN).
Does this change any later calculations?}

Deviations from Gaussian statistics in the curvature perturbation, ${\cal{R}}$,
are parametrized in terms of the non-linearity parameter, 
$\fnl$, as defined in Section~\ref{sec:fnl-intro} by 
${\cal{R}} = {\cal{R}}_G + \frac{3}{5} \fnl  (
{\cal{R}}_G^2 -\langle {\cal{R}}_G^2 \rangle )$. Here the 
quadratic component represents a convolution and 
${\cal{R}}_G$ denotes the Gaussian contribution \cite{maldacena}. In the limit  
where the three momenta have equal magnitude (corresponding to the equilateral  
triangle limit), the leading-order contribution to the non-linearity 
parameter is given by \cite{chenetal,lidser3}
% 
\begin{equation} 
\label{eq:fnldefn-dbiintro}
 \fnl = -\frac{35}{108}\left(\frac{1}{c_s^2} -1 \right) +
\frac{5}{81}\left( \frac{1}{c_s^2} -1 -2\Lambda \right) \,.
\end{equation}
%  
One should note that the sign convention is that employed
by the WMAP data set.
Data from WMAP3 imposes the bound $|\fnl| < 300$ on this parameter
\cite{spergel}. The corresponding bounds for other triangle configurations 
may be much tighter than this and this may be particularly relevant if 
non-Gaussian signatures have indeed been detected in the 
CMB \cite{Yadav:2007yy,crim}. The more recent WMAP5 data set
\cite{Komatsu:2008hk} improves on this bound somewhat, and
also indicates that it is distinctly asymmetric. At the $95 \%$ confidence level, the bound on the 
equilateral triangle becomes $-151 < \fnl < 253$.


% 
% 
% % % % % % % % % % % % % % % % % % % % % % % % % % % % % % % % 
% =========================================================== %
% % % % % % % % % % % % % % % % % % % % % % % % % % % % % % % % 
\section{DBI inflation} 
% 
\label{sec:dbiinflation}
% % % % % % % % % % % % % % % % % % % % % % % % % % % % % % % % 
% =========================================================== %
% % % % % % % % % % % % % % % % % % % % % % % % % % % % % % % % 
\addtodo{ More introduction of type IIB theory}
Flux compactification of type IIB string theory to four dimensions 
results in a warped geometry, where the six-dimensional CY  
manifold contains one or more throats (see \cite{grana,douglas} for reviews). 
The metric inside a throat has the generic form
% 
\begin{equation}
\label{eq:conemetric-dbiintro}
ds_{10}^2= h^2 ( \rho) ds_4^2 + h^{-2} (\rho ) 
\left( d\rho^2 +\rho^2 ds_{X_5}^2 \right) \,,
\end{equation}
%  
where the warp factor $h (\rho)$ is a function of the 
radial coordinate $\rho$ along the throat and $X_5$
is a Sasaki-Einstein five-manifold. 
In many cases, the ten-dimensional metric (\ref{eq:conemetric-dbiintro}) can be 
approximated locally by the geometry $AdS_5 \times X_5$, where the 
warp factor is given by $h=\rho /L$ and 
the radius of curvature of the $AdS_5$ space is defined by
%  
\begin{equation}
L^4 \equiv \frac{4\pi^4 g_s N}{{\rm Vol} (X_5) m_s^4} \,,
\end{equation}
% 
such that $\mathrm{Vol}(X_5)$ is the dimensionless volume of 
$X_5$ with unit radius, $N$ is the ${\rm D3}$  charge of the throat, 
$g_s$ is the string coupling and $m_s$ is the string mass scale.
In the Klebanov-Strassler (KS) background \cite{ks}, the throat 
is a warped deformed conifold and 
corresponds to a cone over the manifold 
$X_5 = T^{1,1}= {\rm SU(2)} \otimes {\rm SU(2)}/{\rm U(1)}$
in the UV limit ($\rho \rightarrow \infty$). This has   
a volume $\mathrm{Vol} (T^{1,1}) = 16\pi^3/27$ and  topology
$S^2\times S^3$, where the $S^2$ is fibred over the $S^3$.
The conical singularity at the tip of the throat 
is smoothed out by an 
$S^3$ `cap' due to the wrapping of the fluxes along the cycles of the conifold
\cite{ks,kt} and the warp factor asymptotes to 
a constant value in this region.   
 

In general, the low-energy world-volume dynamics
of a probe ${\rm D3}$-brane in a warped throat is determined 
by an effective, four-dimensional DBI action. 
The inflaton field is related to the radial 
position of the brane by 
$\varphi \equiv \sqrt{T_3} \rho$, where $T_3 =m_s^4/[(2\pi )^3 g_s ]$ 
is the brane tension. The action is then given by 
% 
\begin{eqnarray}
\label{eq:DBIaction-dbiintro}
S=\int  d^4x \sqrt{|g|} \left[ \frac{\Mpl^2}{2} R 
+ P (\varphi , X) \right] \\
\label{defP}
P( \varphi ,X) = - T (\varphi)  \sqrt{1 - 2T^{-1} (\varphi ) X}
+T (\varphi)  -V(\varphi)  \,,
\end{eqnarray}
% 
where $R$ is the Ricci curvature scalar,
$X \equiv - \frac{1}{2} g^{\mu\nu} \nabla_{\mu} \varphi \nabla_{\nu} \varphi$
is the kinetic energy of the inflaton, $V(\varphi )$ denotes 
the field's interaction 
potential and $T(\varphi ) = T_3 h^4 (\varphi )$
defines the warped brane tension. We refer to $P(\varphi , X)$ as the 
`kinetic function' for the inflaton. 


We consider a spatially flat and isotropic cosmology 
sourced by a homogeneous scalar field. 
In this case, the Friedmann equations for a monotonically 
varying inflaton can be expressed in the form \cite{brane6} 
% 
\begin{eqnarray}
\label{eq:Friedmann-dbiintro}
3 \Mpl^2 H^2(\varphi ) &=& V(\varphi ) -T(\varphi ) 
\left[ 1- \sqrt{1+4\Mpl^4T^{-1} H'^2} \right] \,,\\
\label{useful}
\dot{\varphi} &=& - \frac{2\Mpl^2H'}{\sqrt{1+4\Mpl^4 T^{-1} H'^2}} \, ,
\end{eqnarray}
% 
where $H=H(\varphi )$ represents the Hubble parameter
as a function of the field and a prime denotes $d/d\varphi$. 


An important consequence of the non-trivial kinetic structure 
of the DBI action is that the sound speed of fluctuations in the inflaton 
differs from unity:   
% 
\begin{equation}
\label{eq:csdefn-dbiintro}
c_s = \frac{1}{P_{,X}} = \sqrt{1 -2T^{-1}X}  \,,
\end{equation}
% 
where a subscripted comma denotes partial differentiation. 
Furthermore, the condition that the sound speed be real 
imposes an upper bound on the kinetic energy 
of the inflaton, $\dot{\varphi}^2 < T(\varphi)$, which 
is independent of the steepness of the potential.
The motion of the brane is said to be `relativistic' when this bound is 
close to saturation.
 

We now define the epoch that is directly 
accessible to cosmological observations as `observable inflation'. 
Depending on the reheating 
temperature, this occurred some 30 to 60 e-foldings before the end of 
inflation. We will assume that the slow roll conditions described in 
Section~\ref{sec:slowroll-intro} apply during observable inflation.
The definitions of the slow roll parameters described in that section 
change when $c_s$ is not unity and we
will include a third parameter $s$ which quantifies the rate of change of $c_s$.
The inflationary dynamics during this phase can  
be quantified in terms of these three parameters: 
% 
\begin{eqnarray}
\label{eq:epsdefn-dbiintro}
\epsilon_H \equiv -\frac{\dot{H}}{H^2}
= \frac{XP_{,X}}{\Mpl^2H^2} 
= \frac{2\Mpl^2}{\gamma} \left( \frac{H'}{H} \right)^2 \\
\label{eq:defeta-dbiintro}
\eta_H \equiv  \frac{2\Mpl^2}{\gamma}\frac{H''}{H} \\
\label{eq:defs-dbiintro}
s \equiv \frac{\dot{c_s}}{c_sH} 
= \frac{2\Mpl^2}{\gamma} \frac{H'}{H}\frac{\gamma'}{\gamma}  \,,
\end{eqnarray}
% 
where $\gamma \equiv 1/c_s$. 
We will assume that the `slow-roll' conditions 
$\{ \epsilon_H, |\eta_H | , |s | \}  \ll 1$ applied during observable inflation. 
In this regime, the amplitudes and spectral indices of the two-point functions 
for the scalar and tensor perturbations are given by \cite{gm}
% 
\begin{eqnarray}
\label{eq:spectra-dbiintro}
P_S^2= \frac{1}{8 \pi^2 \Mpl^2} \frac{H^2}{c_s \epsilon_H}
= \frac{H^4}{4\pi^2\dot{\varphi}^2}
\\
P_T^2 = \frac{2}{\pi^2} \frac{H^2}{\Mpl^2} 
\\
\label{indices}
1-n_s = 4 \epsilon_H -2\eta_H  +2s 
\\
 n_t = -2\epsilon_H  
\end{eqnarray}
% 
respectively, where the quantities on the right-hand sides are evaluated 
when the scale with comoving wavenumber $k=aH\gamma$ crossed 
the Hubble radius during inflation.  
\addtodo{ Talk about difference with standard k=aH and compare consistency relation with 
standard one.}
The tensor-scalar ratio, $r \equiv P_T^2/P_S^2$, is directly related to 
the tensor spectral index by \cite{gm}
% 
\begin{equation}
\label{eq:consistency-dbiintro}
r= -8c_sn_t = 16c_s \epsilon_H\,.
\end{equation}
%
Hence, a sound speed different to unity leads to a violation of the 
standard inflationary consistency equation, which might be 
detectable in the foreseeable future \cite{lidser1,lidser2}. 
\addtodo{WMAP5}
Recent observations of the CMB 
indicate that $P^2_S=2.5 \times 10^{-9}$ and $r < 0.55$ \cite{spergel}. 
The best-fit value for a constant spectral index is 
$n_s = 0.987^{+0.019}_{-0.037}$ if $r\ne 0$ is assumed as a prior. 
This is strengthened to 
$n_s = 0.948^{+0.015}_{-0.018}$ for a prior of $r = 0$ \cite{spergel}. 


A further important consequence of a small sound speed is that departures  
from purely Gaussian statistics may be large 
\cite{brane6,brane11,lidser3,chenetal}. 
\addtodo{Link to intro section on $\fnl$. Add note about using equilateral in
next two chapters.}
It is conventional 
to quantify such deviations by expressing the non-Gaussian curvature 
perturbation ${\cal{R}}$ in the form 
${\cal{R}} ={\cal{R}}_G+f_{NL}
({\cal{R}}^2_G -\langle {\cal{R}}^2_G \rangle)$, where 
${\cal{R}}_G$ represents the Gaussian contribution, 
the quadratic piece is a convolution and $f_{NL}$ defines 
the `non-linearity' parameter \cite{maldacena}. 
In general, this parameter is a function of the three momenta which 
form a triangle in Fourier space. However, in the limit where 
these momenta have equal magnitude, the non-linearity parameter 
is given to leading-order by \cite{chenetal,lidser2}  
% 
\begin{equation}
\label{eq:fnlcs-dbiintro}
f_{NL} = \frac{1}{3} \left( \frac{1}{c_s^2} -1 \right) \,.
\end{equation}
% 
\addtodo{ Change to WMAP5 numbers}
Currently, WMAP3 constrains this parameter  
to be ${| f_{NL} |} \, {\lsim} \, {300}$ \cite{spergel,crim}.  


Finally, Eqs. (\ref{eq:csdefn-dbiintro}), (\ref{eq:spectra-dbiintro}) and (\ref{eq:fnlcs-dbiintro}) 
may be combined to constrain the warped brane tension 
during observable inflation: 
% 
\begin{equation}
\label{eq:obswarp-dbiintro}
\frac{T (\varphi_*)}{\Mpl^4}  = 
\frac{\pi^2}{16} r^2P_S^2 \left( 1+\frac{1}{3f_{NL}} \right) \,,
\end{equation}
% 
where a subscript `$*$' denotes that the quantity is to be evaluated 
at that epoch. 


\addtodo{ Change to chapter references}
In the following, we first consider the UV version of DBI inflation,
where the brane moves relativistically 
towards the tip of the throat. We will assume implicitly 
that the sound speed is sufficiently small to generate a non-Gaussianity in 
excess of $f_{NL} \, {\gsim} \, 5$, since this is the projected 
sensitivity of the Planck satellite \cite{planck}. Furthermore, we consider   
an arbitrary warp factor and inflaton potential, 
subject only to the condition that a sufficiently long phase of 
quasi-exponential expansion can be maintained to solve the horizon problem of
the hot big bang model. 
% 
%




% % % % % % % % % % % % % % % % % % % % % % % % % % % % % % % % 
% =========================================================== %
% % % % % % % % % % % % % % % % % % % % % % % % % % % % % % % % 
\section{The Lyth bound}
\label{sec:lyth-dbiintro}
% % % % % % % % % % % % % % % % % % % % % % % % % % % % % % % % 
% =========================================================== %
% % % % % % % % % % % % % % % % % % % % % % % % % % % % % % % % 

Eqs. (\ref{eq:Ps-dbiintro}) and (\ref{eq:Pt-dbiintro}) imply 
that the variation of the inflaton field during inflation  
is related to the tensor-scalar amplitude by \cite{lyth,bmpaper}
% 
\begin{equation}
\label{eq:genlyth-dbiintro}
\frac{1}{\Mpl^2}
\left( \frac{d \varphi}{d \cal{N}} \right)^2 = \frac{r}{8 c_s P_{,X}} \,,
\end{equation}
% 
where ${\cal{N}} \equiv \int dt \, H$ denotes the number of e-foldings.
We will refer to the epoch of inflation that can be directly 
constrained by cosmological observations as 
`observable inflation' and will assume that this phase 
occurred when the brane was located within a 
throat region\footnote{We denote the values of all parameters 
evaluated during observable inflation by a subscript 
`$*$'.}. Observable inflation corresponds to no more than about 4 e-foldings  
of inflationary expansion, $\Delta {\cal{N}}_* \simeq 4$. 
The total variation in the inflaton field between the epoch of observable 
inflation and the end of inflation is then given by
% 
\begin{equation}
\label{eq:totalfield-dbiintro}
\frac{\Delta \varphi_{\rm inf}}{\Mpl} = 
\left( \frac{r}{8c_sP_{,X}} \right)_*^{1/2} {\cal{N}}_{\rm eff} \,,
\end{equation}
% 
where
% 
\begin{equation}
\label{eq:Neff-dbiintro}
{\cal{N}}_{\rm eff} \equiv \left( \frac{c_sP_{,X}}{r}\right)_*^{1/2}
\int_0^{{\cal{N}}_{\rm end}}  
\left( \frac{r}{c_sP_{,X}} \right)^{1/2} d {\cal{N}} \,.
\end{equation}
% 
If $r/(c_s P_{,X})$ varies 
sufficiently slowly during observable inflation, 
the corresponding change in the value of the inflaton  
field is given approximately by \cite{lyth,bmpaper}
% 
\begin{equation}
\label{eq:approxlyth-dbiintro}
\left( \frac{\Delta \varphi}{\Mpl} \right)_*^2 \simeq 
\frac{(\Delta {\cal{N}}_*)^2}{8} \left( \frac{r}{c_sP_{,X}} \right)_* \,.
\end{equation}
% 

