\section{Mairi's Corrections}

\subsection{Point 2: Drawbacks of Inflation}
\textbf{Added to discussion on Pg 36:}

In this thesis inflation is taken to be the mechanism by which inhomogeneities in matter are seeded
and the horizon and flatness problems of the Big Bang are solved. However, the inflationary
paradigm is not without its own challenges.

Chief amongst these is the lack of a unique underlying theory. Many high energy theories have been
shown to produce an inflationary phase. Often, however, these require a great deal of fine-tuning
in order to produce a sufficient number of e-foldings of inflation. Lack of knowledge about the
governing physics at high energy scales hampers our understanding of the cause of inflation and
undermines any analysis of the generic nature of the initial conditions required.

The overall duration of inflation is also unknown. Observations only require that currently
observable scales were previously inside the horizon. Thus the onset of inflation is not
constrained and could occur far in the past. However, allowing such a long inflationary period
typically increases the fine-tuning necessary and can lead to other issues. 

There are further problems with the inflationary paradigm, including the lack of an explanation for
how energy in the inflaton field is transferred to the other constituent parts of the universe, and
indeed the fact that no scalar field has yet been directly observed.

We will continue to employ the inflationary paradigm in this thesis but it is important to
acknowledge that some challenges remain to be overcome.

\subsection{Point 3: T-duality discussion}
\textbf{Inserted on Pg 40 just below Section 3.2.2 header:}

In string theory an extra space time symmetry is present which relates physical properties in
theories with large compactification radius with those in theories with small radius. 
Suppose we have a string theory compactified on a circle of radius $L$. The ``T-duality''
transformation which relates two physical theories with this one compactified dimension is
% 
\begin{equation}
\label{eq:tdualtransform-dbiintro}
 L \rightarrow \wt{L} = \frac{\alpha'}{L}\,.
\end{equation}
% 
Now consider what effect this transformation will have on the momentum of a closed string. Instead
of being a continuum, the momentum takes discrete values \ldots
\\

\textbf{Rephrased end of Section 3.2.2 on Pg 41 as follows:}

The formula for the mass spectrum, \eq{eq:closedmass-dbiintro}, is invariant
when $j$ and $w$ are exchanged given the transformation in \eq{eq:tdualtransform-dbiintro}.
Writing the equations of motion in terms of $\wt{L}$, having interchanged $j$ and
$w$, gives a new theory which is compactified on a circle of radius $\wt{L}$. This is known as the
T-dual theory \cite{Sakai1986,Kikkawa1984b}. The two
theories are physically identical since T-duality is an exact symmetry
of string theory for closed strings. The T-duality applies to all physics in the theory and in
particular also affects open string modes. These behave in a different way under T-duality to closed
strings as will be described below.
\\

\textbf{Rephrased part of Section 3.2.3 on Pg 41 as follows:}

We introduced T-duality by explaining
its effects on closed strings. But what happens to the open strings in a T-dualised theory? Open
strings, as their name suggests, have two open ends and
consequently cannot have a conserved winding number such as $w$. Suppose once more
that one
of the $D$ dimensions is compactified. As $L\rightarrow0$, the non-zero momentum
states become infinitely massive, but in contrast to the closed case there is now
no continuum of winding states. Thus, the open string now lives in $D-1$ dimensions
similar to the result of standard KK compactification \cite{Johnson2000}.
The endpoints of the open strings 
then observe Dirichlet boundary conditions, taking fixed values in the compactified
direction.
There are still closed strings in this theory, however, and these continue to
move in the full $D$ dimensions after being T-dualised.  

The result is similar if more than
one coordinate is made periodic.
If $D-p-1$ spatial dimensions are compactified, for some $p$, then the ends of
the open
strings can still move freely in the other $p$ spatial dimensions on a $p+1$
dimensional hypersurface. This hypersurface is called a Dirichlet brane or
D$p$-brane. The closed string modes move in the full $D$ dimensions.



\subsection{Point 4: Validity of Eq 3.28}
\textbf{Added to end of Pg 48:}

In deriving \eq{eq:approxlyth-dbiintro} we have assumed that $r/\cs \PX$ varies slowly during
observable inflation. For the DBI case, $\cs \PX = 1$ and the change in $r$ can be related to the
change in $\epsilon_H$ and $\cs$ through \eq{eq:rdefn-dbiintro}. As we have taken $\epsilon_H,
|\eta_H|,|s|\ll 1$ the tensor-scalar ratio will indeed vary slowly over the observable epoch.
% 
For more general models where $\cs \PX \ne 1$ we have that
% 
\begin{equation}
 \frac{\d }{\d \N}\left[ \frac{r}{\cs \PX}\right] = 16\frac{\epsilon_H}{\PX}\left( 2\epsilon_H
-2\eta_H\right)\,.
\end{equation}
% 
Therefore $r/\cs\PX$ varies slowly as long as $\PX$ is not too small, \iec close to
$\mathcal{O}(\epsilon_H^2)$. This will not be the case in the models studied in Chapters
\ref{ch:dbi} and \ref{ch:multibrane}.

\subsection{Point 5: No isocurvature mode considered}
\textbf{Added to Perturbation section on Pg 24:}

For single field models no mixing
of adiabatic and non-adiabatic modes occurs \cite{Weinberg200804}. Therefore, throughout this
thesis we will only consider adiabatic perturbations and ignore any isocurvature mode present.

\subsection{Point 6: Assumptions leading to $r<10^{-7}$}
\textbf{Added to Section 4.2 on Pgs 54 and 55:}


Before concluding this section, we should explicitly outline all the assumptions that have lead to
\eq{eq:standardbound}. First, we are considering the relativistic limit where $\cs\ll1$. We are
also restricting ourselves to considering the UV scenario where a brane moves towards the tip of
the throat. This ensures that \eq{eq:importantbound} is satisfied. For the Lyth bound to take the
form in \eq{eq:approxlyth-dbiintro}, we have assumed that $r$ varies slowly during the observable
period of inflation. This is justified as the change in $r$ can be written in terms of the
quasi deSitter parameters $\epsilon_H, \eta_H$ and $s$ and we have assumed their magnitudes are
much less than unity.
\\
\ldots
\\
In order to neglect the $\fnleq$ term in \eq{eq:boundpower6} we have assumed that $-\fnleq>5$. As
$\cs$ has been taken to be small this is expected to be the case. The volume of the Sasaki-Einstein
manifold $X_5$ is taken to be $\mathcal{O}(\pi^3)$ in keeping with the values for known solutions.
The WMAP5 normalisation of the scalar perturbation power spectrum has also been used. Finally, in
going from \eq{eq:upperbound} to the final numerical figure in \eq{eq:standardbound} the most
``optimistic'' value, $\Delta \N_*\simeq 1$, has been chosen as this leads to the least
restrictive bound on $r$.  As described above a more realistic value of $4$ would severly
constrain $r$ due to the strong dependence of \eq{eq:upperbound} on $\Delta\N_*$.

\subsection{Point 7: Discussion of $Y^{p,q}$}
\textbf{Added to Section 4.3 on Pg 57:}

Previously only two five-dimensional Sasaki-Einstein
metrics were explicity known, $S^5$ and $T^{1,1}$ on $S^2\times S^3$. The $Y^{p,q}$ metrics
described in \Rref{gauntlett} are a countably infinite number of Sasaki-Einstein metrics on
$S^2\times S^3$. The metrics are parametrised by the two topological numbers $p$ and $q$, which are
coprime when the $Y^{p,q}$ is topologically $S^2\times S^3$. The volume of one of these manifolds
is proportional to $1/p$. Hence by setting $q=1$ and letting $p$ become large, this volume can be
made arbitrarily small \cite{gauntlett}. On the other hand, the largest volume occurs for $p=2$,
$q=1$ giving $\mathrm{Vol}(Y^{2,1})\simeq 0.29\pi^3$. 

\subsection{Point 8: Rewrite Section 4.4 more clearly}
\textbf{Many changes have been made to Section 4.4 including splitting the later parts into two
subsections. The functions $f_1$, $f_2$ and $f_3$ have been renamed $f_A$ etc in order to avoid
confusion with Section 5.2. Many equations from previous sections have been repeated in order to
improve readability.}

\textbf{Major changes:}

\textbf{Change to introductory section on Pg 58:}

In this section, we take a phenomenological 
approach and consider the following kinetic function which has a more general form than the DBI one
but still contains a square root term:
% 
\begin{equation}
\label{eq:genaction-dbi}
P= -f_A (\varphi ) \sqrt{1-f_B (\varphi ) X} -f_C (\varphi) \,,
\end{equation}
% 
where $f_i (\varphi )$ are unspecified functions of the inflaton 
field. We will assume 
implicitly that these functions have a suitable form for 
generating a successful phase of inflation.
A direct comparison with \eq{eq:Pdefn-dbiintro} 
indicates that the standard DBI action can be recovered by setting $f_A f_B =2$. This implies
that $\cs P_{,X} =1$ and greatly simplifies the form of \eq{eq:rtheory}. 
Another important property in the DBI case is that the warp factor uniquely determines 
the kinetic structure of the action, i.e., $h^4 \propto f_A \propto f_B^{-1}$.  
In view of this, it is interesting to consider whether
the gravitational wave constraints could be weakened by relaxing one 
or both of these conditions. 

% 
% \eq{eq:genaction-dbi} 
% can be transformed into a similar form to that of 
% \eq{eq:Pdefn-dbiintro} through the field redefinition 
% % 
% \begin{equation}
% \label{eq:defvarphi}
% \tilde{\varphi} \equiv \int d \varphi \sqrt{\frac{f_Af_B}{2}}  \,.
% \end{equation}
% 

We can differentiate $P(X, \vp)$ in \eq{eq:genaction-dbi} to find:
% 
\begin{align}
 \PX &= \frac{f_A f_B}{2\sqrt{1-f_B X}} \,, \\
 P_{,XX} &= \frac{f_A f_B}{2}\frac{f_B}{2(1-f_B X)^{\frac{3}{2}}} \,.
\end{align}
% 
% 
The sound speed of fluctuations in 
the inflaton, defined in \eq{eq:defcs-dbiintro}, is then given by
%  
\begin{equation}
\label{eq:cs-dbi}
\cs = \sqrt{1-f_B X} = \frac{f_Af_B}{2} \frac{1}{P_{,X}}  \,,
\end{equation}
% 
and the scalar power spectrum \eqref{eq:Ps-dbiintro} by
% 
\begin{equation}
\label{eq:spectrum-dbi}
\Pr = \frac{1}{2\pi^2}\frac{H^4}{f_Af_B\dot{\varphi}^2}  \,.
\end{equation}

\textbf{Explanation of generalisation of BM bound on Pg 59 expanded and revised:}

The BM bound \eqref{eq:BMboundr} restricts the maximal 
variation of the scalar field $\varphi$ in the full throat region for DBI inflation. 
This is determined by expression \eqref{eq:BMbound-dbi} 
for generic warped geometries that are asymptotically 
$AdS_5 \times X_5$ away from the tip of the throat. However, in Section~\ref{sec:lyth-dbiintro} the
Lyth bound was also defined for general
non-canonical actions. For the more general kinetic function
\eqref{eq:genaction-dbi}, the BM bound becomes
% 
\begin{equation}
\label{eq:bmboundgen-dbi}
 r < \frac{32}{N(\Neff)^2}\cs\PX = \frac{16}{N(\Neff)^2}f_A f_B \,.
\end{equation}
% 
To use this bound we must be able to calculate $\Neff$ over the full range of e-foldings of
inflation. This requires knowledge of the behaviour of $f_A$ and $f_B$ over that range. 

A more cautious approach would be to restrict our attention to 
the observable stage of inflation.
Assuming that the variation
of $f_Af_B = 2 \cs\PX$ is negligible during that epoch, we can use
\eq{eq:approxlyth-dbiintro} which states that 
% 
\begin{equation}
\label{eq:genphivary1}
\left( \frac{\Delta \varphi}{\Mpl} \right)^2_{*} \simeq 
\frac{(\Delta \N_{*} )^2}{8} \left(\frac{r}{\cs\PX}\right)_*  
= \frac{(\Delta \N_{*} )^2}{4} \left(\frac{r}{f_A f_B}\right)_*  \,.
\end{equation}
% 
In addition, 
if observable scales leave the horizon 
while the brane is inside the throat, the change in the field value 
must satisfy $| \Delta \varphi_*|<\varphi_{UV}$. It follows from \eqs{eq:bmboundgen-dbi} and
\eqref{eq:genphivary1}, therefore, that 
% 
\begin{equation}
\label{eq:genBMbound}
r_*< \frac{32}{N (\Delta \N_{*})^2} (\cs\PX)_* = \frac{16}{N (\Delta \N_{*})^2}
(f_Af_B)_* \,.
\end{equation}
% 

Condition 
(\ref{eq:genBMbound}) will be referred to as the 
generalised BM bound. 
We have been
conservative by restricting our discussion to the 
observable phase of inflation. A stronger condition is obtained by using
\eq{eq:bmboundgen-dbi}, which is equivalent to substituting $\Delta \N_* \rightarrow 
\Neff$. If 
$f_Af_B$ remains nearly constant over the last $\N$ 
e-foldings of inflation, 
then $\Neff$ may be as large as $60$ and the right hand side of \eq{eq:genBMbound} will be reduced
by a factor of $225$. 
Thus, the generalised bound 
(\ref{eq:genBMbound}) should be regarded as a necessary 
(but not sufficient) condition to be satisfied by the tensor modes.


\subsection{Point 9: Naturalness of multi-coincident brane inflation}
\textbf{Added to Section 5.1 on Pg 65:}

One scenario in which multiple branes are expected is after brane flux annihilation, in which
branes travelling down the throat annihilate with the trapped flux, creating new branes
\cite{thomasward, DeWolfe:2004qx, Kachru:2002gs}. These are then attracted by other branes and
fluxes in other throats and propagate toward the bulk.
In \Rref{thomasward} Thomas \& Ward argue that it is unlikely that only a single brane is left
after the flux annihilation process, due to the large amount of fine tuning necessary to achieve
this. Instead it is more likely that a number of branes remain.

\subsection{Point 10: Rephrasing of Section 5.4 about large n limit}
\textbf{Two subsection headers have been inserted and the main bulk of Section 5.4 has been
rephrased. A factor of 2 error has been fixed but conclusions remain the same.}

\textbf{Reworked section:}

The regime $W \gg 1$ is of interest for 
relaxing the gravitational wave constraints\footnote{Note that 
the case $n \gg 1$ and
$W \sim 1$ will not significantly relax the BM bound, 
since we require $n \ll N$ for backreaction effects to be negligible.}. 
The generalised BM bound for IR models, with branes propagating towards the bulk, is given by 
\eq{eq:f2IRlower}:
% 
\begin{equation}
\label{eq:f2IRlower-multi}
\frac{f_2\Mpl^4}{N} > \frac{(\Delta \N_*)^2}{4\pi^2}
\frac{\sqrt{-3\fnleq}}{(1-n_s)\Pr}  \,.
\end{equation}
% 
As we know $f_2$, this may be expressed as 
a limit on the value of the warp factor $h(\varphi_*)$ on CMB scales: 
% 
\begin{equation}
\label{eq:warptorelax}
N T_3 \left(\frac{h_*}{\Mpl}\right)^4 < 
\frac{8\pi^2 (1-n_s)\Pr}{\sqrt{-3\fnleq}(\Delta \N_*)^2} \,.
\end{equation}
% 


We now consider whether this limit can be satisfied for reasonable choices 
of parameters when the warped compactification corresponds to 
an $AdS_5$ or KS throat, respectively. Recall that the warp 
factor for the $AdS_5$ throat is given by $h=\varphi/(\sqrt{T_3}L)$.  
Condition (\ref{eq:warptorelax}) therefore reduces to a constraint on the 
value of the inflaton during observable inflation: 
% 
\begin{equation}
\label{eq:phivalue-dbi}
\frac{\varphi_*^4}{\Mpl^4} < 
\frac{8\pi^2 (1-n_s)\Pr}{\sqrt{-3\fnleq} ( \Delta \N_*)^2} 
\frac{T_3 L^4}{N} \,.
\end{equation}
%  
However, non-perturbative string effects are expected to become 
important below a cutoff scale, $\varphi_{\rm cut} = 
h_{\rm cut} \sqrt{T_3} L$, where $h_{\rm cut}$ is the value of the 
warp factor at that scale. For consistency, therefore, one requires 
$\varphi_*>\varphi_{\rm cut}$, so that 
% 
\begin{equation}
 N T_3 \left(\frac{h_\mathrm{cut} }{\Mpl}\right)^4 < 
\frac{8\pi^2 (1-n_s)\Pr}{\sqrt{-3\fnleq} ( \Delta \N_*)^2} \,,
\end{equation}
% 
which implies an upper limit on the 
${\rm D3}$-brane charge: 
% 
\begin{equation}
\label{eq:Nlimit-dbi}
N< \frac{64\pi^5 \gs (1-n_s)\Pr}{\sqrt{-3\fnleq} ( \Delta \N_*)^2}
\left( \frac{\Mpl}{h_{\rm cut} \ms} \right)^4   \,.
\end{equation}
% 
Assuming the typical values $\ms \sim 10^{-2}\Mpl$, 
$\Delta \N_* \simeq 4$ and 
$h_{\rm cut} \sim 10^{-2}$ implies
%   
\begin{equation}
N < 1.76 \times 10^8 (1-n_s)(-\fnleq)^{-1/2} \,, 
\end{equation}
% 
and for $1-n_s <0.05$ and $-\fnleq>5$ the inequality becomes
% 
\begin{equation}
 N < 4\times 10^6\,.
\end{equation}
%  


For an $AdS_5$ throat, the fuzzy potential $W$ is a constant,
and the condition that $W \gg 1$ becomes 
% 
\begin{equation}
\label{eq:Chatlimit}
\hat{C} \ll \frac{4\pi^2\gs N}{\mathrm{Vol}(X_5)} \,.
\end{equation}
% 
Hence, combining inequalities 
(\ref{eq:Nlimit-dbi}) and (\ref{eq:Chatlimit}) implies that
%  
\begin{equation}
\label{eq:nlimit-dbi}
\hat{C} \ll 
\frac{2 (2\pi)^7 (1-n_s)\Pr}{\sqrt{-3\fnleq} ( \Delta \N_*)^2}
\frac{\gs^2}{\mathrm{Vol} (X_5) }
\left( \frac{\Mpl}{h_{\rm cut}\ms} \right)^4  \,,
\end{equation}
% 
and specifying $\gs \sim 10^{-2}$ and 
$\mathrm{Vol}(X_5) \simeq \pi^3$ then yields the limit  
% 
\begin{equation}
\hat{C} \ll 2.25\times 10^{6} (1-n_s)(-\fnleq)^{-1/2} < 5\times 10^4,
\end{equation}
% 
 or equivalently,
%   
\begin{equation}
\label{eq:nbound-multi}
n \ll 225.
\end{equation}
% 
In deriving the action \eqref{eq:largeP-multi} the number of coincident branes was assumed to be
large. However we have now found that for the case of branes propagating towards the bulk, the
number of such branes is bounded from above.
Furthermore, since $f_1f_2 \simeq {\rm constant}$ for the $AdS_5$ throat, 
the stronger form of the inequality (\ref{eq:genBMbound}) may be used. The right hand side of
inequality \eqref{eq:nlimit-dbi} would be reduced by a factor of $(   \Neff /\Delta \N_*)^2$ by
substituting 
$\Delta \N_* \rightarrow \Neff$. This ratio 
could be as high as $(60/4)^2 \simeq 200 $, leading to $n$ being less than $15$. In this case
the assumption of large $n$ would clearly be inconsistent and the model would be ruled out. 

\subsection{Point 11: Changed ``hybrid'' to ``toy model''}
\textbf{All mentions of the hybrid model have been changed.}

\textbf{Change in list of models on Pg 102:}
\begin{enumerate}
\item[4.] $V(\vp) = \msqphisqwithV$. This is a contrived toy model which requires inflation to
be terminated by hand. We will set inflation to end when $\vp \simeq 8$. By taking a value
of $U_0 = 5\e{-10}\Mpl^{4}$ a blue spectrum ($n_s>1$) can then be obtained
\cite{Linde:1993cn,Komatsu:2008hk}.
\end{enumerate}

\textbf{Change in description on Pg 129:}
The fourth model, with potential $V(\vp)=\msqphisqwithV$, is a contrived toy model.
As described in Section~\ref{sec:pots-num}, in order to perform the single field
calculation, the end time of inflation must be specified by hand. 

\textbf{Change in caption of Fig 8.12 on Pg 131:}
Comparison of the
source term
evolution for the four different models. After horizon crossing the magnitude of the
source term is larger for the quadratic and quartic models than for the other two.
Towards the end of the numerical calculation there is a marked increase in $|S|$ for
three of the models as $\bar{\varepsilon}_H$ increases towards unity. The end time of
inflation is specified by hand for the contrived toy model, so this effect is
not seen.

