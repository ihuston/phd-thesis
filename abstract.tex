% % % % % % % % % % % % % % % % % % % % % % % % % 
% abstract.tex - Ian Huston
% $Id: abstract.tex,v 1.11 2009/08/21 10:44:51 ith Exp $
% % % % % % % % % % % % % % % % % % % % % % % % % 
% Redefine CVSRevision for this section
\renewcommand{\CVSrevision}{\version$Id: abstract.tex,v 1.11 2009/08/21 10:44:51 ith Exp $}
% 
% 
\chapter*{Abstract}
\label{ch:abstract}
\addcontentsline{toc}{chapter}{Abstract}
\section*{}
\singlespacing
Inflationary cosmology is the leading explanation of the very early universe. 
Many different models of inflation have been constructed which fit current observational data.
In this work theoretical and numerical methods for constraining the parameter space of a wide class
of such models are described.

First, string-theoretic models with large non-gaussian signatures are investigated.
% First paper
An upper bound is placed on the amplitude of primordial gravitational waves produced by ultra-violet
Dirac-Born-Infeld inflation. In all but the most finely tuned cases this bound is incompatible with
a lower bound derived for inflationary models which exhibit a red spectrum and detectable
non-gaussianity. 

% The bound is insensitive
% to the form of the inflaton potential and the warp factor of the compactified
% dimensions and can be expressed entirely in terms of observational parameters
% once the volume of the five-dimensional sub-manifold of the throat has been
% specified. 
% The bound predicts
% undetectably small tensor perturbations with a tensor-scalar ratio $r <
% 10^{-7}$. 
% This is incompatible with a corresponding lower limit of $r > 0.1
% (1-n_s)$, which applies to any model that generates a red spectral index $n_s
% <1$ and a potentially detectable non-gaussianity.
% % Possible ways of evading these bounds in more general DBI-type scenarios are
% % discussed and a multiple-brane model is investigated as a specific example. 
% 
% % Second Paper
% Extending this analysis, a class of non-canonical inflationary models is
% identified where the leading-order contribution to the non-gaussianity is
% determined by the sound speed of the fluctuations in
% the inflaton field. 
% This class of models includes the effective action for
% multiple coincident branes in the finite $n$ limit. 
% % The action for this
% % configuration is determined using an iterative technique, based upon the
% % fundamental representation of SU(2). 
% The upper bounds on $r$ in the single-brane
% DBI scenario can be relaxed in multi-brane configurations if a large and
% detectable non-gaussianity is generated. 
% % Moreover models with a small number of
% % coincident branes could generate a gravitational wave background that will be
% % observable to future experiments. 
By analysing general non-canonical actions, a class of models is found which can
evade the upper bound when the phase speed of perturbations is small. The multi-coincident brane
scenario with a finite number of branes is one such model. 
For models with a potentially observable gravitational wave spectrum the number of coincident branes
is shown to take only small values. 

% Third paper
The second method of constraining inflationary models is the numerical calculation
of second order perturbations for a general class of single field models.
The Klein-Gordon equation at second order, written in terms
of scalar field variations only, is numerically solved. 
The slow-roll version of the second order source term is used and the method is
shown to be extendable to the full equation.
This procedure allows the evolution of second order
perturbations in general and the calculation of the non-gaussianity parameter in cases
where there is no analytical solution available.  


