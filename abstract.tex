% % % % % % % % % % % % % % % % % % % % % % % % % 
% abstract.tex - Ian Huston
% $Id: abstract.tex,v 1.7 2009/08/15 17:00:02 ith Exp $
% % % % % % % % % % % % % % % % % % % % % % % % % 
% Redefine CVSRevision for this section
\renewcommand{\CVSrevision}{\version$Id: abstract.tex,v 1.7 2009/08/15 17:00:02 ith Exp $}
% 
% 
\chapter*{Abstract}
\label{ch:abstract}
\addcontentsline{toc}{chapter}{Abstract}
\section*{}
\singlespacing
Inflationary cosmology is the leading explanation of the very early universe. 
Many specific models have been derived which agree with current observations.
The non-gaussianity of the universe shortly after the Big Bang has emerged as
an important tool to differentiate between these models. In this work
two aspects of the search for non-gaussianity are described.

First, string-inspired models with large non-gaussianity are investigated.
% First paper
An upper bound on the amplitude of the primordial gravitational wave spectrum
generated during ultra-violet Dirac-Born-Infeld inflation is derived. 
% The bound is insensitive
% to the form of the inflaton potential and the warp factor of the compactified
% dimensions and can be expressed entirely in terms of observational parameters
% once the volume of the five-dimensional sub-manifold of the throat has been
% specified. 
The bound predicts
undetectably small tensor perturbations with a tensor-scalar ratio $r <
10^{-7}$. 
This is incompatible with a corresponding lower limit of $r > 0.1
(1-n_s)$, which applies to any model that generates a red spectral index $n_s
<1$ and a potentially detectable non-gaussianity in the curvature perturbation.
% Possible ways of evading these bounds in more general DBI-type scenarios are
% discussed and a multiple-brane model is investigated as a specific example. 

% Second Paper
Extending this analysis, a class of non-canonical inflationary models is
identified where the leading-order contribution to the non-gaussianity of the
curvature perturbation is determined by the sound speed of the fluctuations in
the inflaton field. 
This class of models includes the effective action for
multiple coincident branes in the finite n limit. 
% The action for this
% configuration is determined using an iterative technique, based upon the
% fundamental representation of SU(2). 
In principle, the upper bounds on $r$ that arise in the single-brane
DBI scenario can be relaxed in multi-brane configurations if a large and
detectable non-Gaussianity is generated. 
% Moreover models with a small number of
% coincident branes could generate a gravitational wave background that will be
% observable to future experiments. 

% Third paper
The second aspect of non-gaussianity described is the numerical calculation
of the non-gaussian parameter $\fnl$ for a general class of single field models.
The Klein-Gordon equation at second
order in cosmological perturbation theory is numerically solved in closed form. 
The slow-roll
version of the second order source term is used and the method is
shown to be extendable to the full equation.
% We consider two standard single field models and find that
% the results agree with previous calculations using analytic methods, where
% comparison is possible. 
The procedure allows the evolution of second order
perturbations in general and the calculation of the parameter
$\fnl$ to be examined in cases where there is no analytical solution available. 
