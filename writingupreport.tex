\documentclass[a4paper,11pt]{article}
\usepackage{ihmacros}
\usepackage{fullpage}

%opening
\title{Application to transfer to writing up status}
\author{Ian Huston}

\begin{document}

\maketitle

This document outlines my plan to complete my thesis by 1st November 2009. I am requesting a
transfer to writing-up status beginning on the 20th September 2009. I have already begun the
process of writing-up and believe that I can complete my work in the timetable provided below. 

\section*{Title and abstract}
The title of my thesis is ``Investigating inflation with branes and perturbations'' and the current
abstract is:
\\

Inflationary cosmology is the leading explanation of the very early universe. 
Many specific models have been derived which agree with current observations.
The non-gaussianity of the universe shortly after the Big Bang has emerged as
an important tool to differentiate between these models. In this work
two aspects of the search for non-gaussianity are described.

First, string-inspired models with large non-gaussianity are investigated.
% First paper
An upper bound on the primordial gravitational wave spectrum
generated during ultra-violet Dirac-Born-Infeld inflation is derived. 
% The bound is insensitive
% to the form of the inflaton potential and the warp factor of the compactified
% dimensions and can be expressed entirely in terms of observational parameters
% once the volume of the five-dimensional sub-manifold of the throat has been
% specified. 
The bound predicts
undetectably small tensor perturbations with a tensor-scalar ratio $r <
10^{-7}$. 
This is incompatible with a corresponding lower limit of $r > 0.1
(1-n_s)$, which applies to any model that generates a red spectral index $n_s
<1$ and a potentially detectable non-gaussianity.
% Possible ways of evading these bounds in more general DBI-type scenarios are
% discussed and a multiple-brane model is investigated as a specific example. 

% Second Paper
Extending this analysis, a class of non-canonical inflationary models is
identified where the leading-order contribution to the non-gaussianity is
determined by the sound speed of the fluctuations in
the inflaton field. 
This class of models includes the effective action for
multiple coincident branes in the finite $n$ limit. 
% The action for this
% configuration is determined using an iterative technique, based upon the
% fundamental representation of SU(2). 
The upper bounds on $r$ in the single-brane
DBI scenario can be relaxed in multi-brane configurations if a large and
detectable non-gaussianity is generated. 
% Moreover models with a small number of
% coincident branes could generate a gravitational wave background that will be
% observable to future experiments. 

% Third paper
The second aspect of non-gaussianity described is the numerical calculation
of the non-gaussian parameter $\fnl$ for a general class of single field models.
The Klein-Gordon equation at second
order in cosmological perturbation theory is numerically solved in closed form. 
The slow-roll
version of the second order source term is used and the method is
shown to be extendable to the full equation.
% We consider two standard single field models and find that
% the results agree with previous calculations using analytic methods, where
% comparison is possible. 
The procedure allows the evolution of second order
perturbations in general and the calculation of $\fnl$ in cases
where there is no analytical solution available. 

\section*{Table of contents}
\begin{verbatim}
1. Introduction
        Timescale: 1/2 month
1.1. The Friedmann-Robertson-Walker Universe
1.2. Canonical Slow Roll Inflation
1.3. Non-Canonical Inflation
1.4. Non-Gaussianity
1.5. Current Observations
1.6. Outline and goals

I. DBI inflation

2. Introduction to Dirac Born Infeld Inflation
        Timescale: draft version completed
2.1. Introduction
2.2. String theory and extra dimensions
2.2.1. Cosmology and String Theory
2.2.2. Extra dimensions
2.2.3. T-duality
2.2.4. D-branes
2.2.5. Warped throats
2.3. DBI inflation
2.4. The Lyth bound

3. Observational Bounds on DBI Inflation
        Timescale: 1/2 month, consists of already published paper
3.1. An Upper Bound on the Primordial Gravitational Waves
3.2. A Lower Bound on the Primordial Gravitational Waves
3.3. Relaxing the Baumann-McAllister Bound
3.4. Review of other models evading the bounds 
3.5. Conclusion

4. Multi brane inflation
        Timescale: 1/2 month, consists of already published paper
4.1. Introduction
4.2. IR Inflation and Large n Limit
4.3. Theoretic Upper Bounds on the Tensor-Scalar Ratio
4.4. Relaxing the Upper Bounds on the Tensor-Scalar Ratio
4.5. Action for Multiple Coincident Branes
4.6. Bounds on the Tensor-Scalar Ratio for Multi-Brane Inflation
4.7. Discussion

II. Numerical simulations of perturbations

5. Second Order Perturbations
        Timescale: 1/2 month, consists of already published paper
5.1. Introduction
5.2. Perturbations
5.2.1. First and Second Order
5.2.2. Slow Roll approximation

6. Numerical simulations
        Timescale: 1/2 month, consists of already published paper
6.1. Equations
6.2. Initial Conditions
6.3. Implementation
6.4. Code Tests
6.5. Results
6.6. Discussion and conclusion

7. Overall Conclusions

Bibliography
\end{verbatim}
\end{document}
