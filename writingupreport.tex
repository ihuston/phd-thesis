\documentclass[a4paper,11pt]{article}
\usepackage{ihmacros}
\usepackage{fullpage}

%opening
\title{Application to transfer to writing up status}
\author{Ian Huston}

\begin{document}

\maketitle

This document outlines my plan to complete my thesis by 1st November 2009. I am requesting a
transfer to writing-up status beginning on the 20th September 2009. I have already begun the
process of writing-up and believe that I can complete my work in the timetable provided below. 

\section*{Title and abstract}
The title of my thesis is ``Constraining inflationary scenarios with braneworld models and second
order cosmological perturbations'' and the current
abstract is:
\\

Inflationary cosmology is the leading explanation of the very early universe. 
Many different models of inflation have been constructed which fit current observational data.
In this work theoretical and numerical methods for constraining the parameter space of these models
are described.

First, string-theoretic models with large non-gaussian signatures are investigated.
% First paper
An upper bound is placed on the amplitude of primordial gravitational waves produced by ultra-violet
Dirac-Born-Infeld inflation. In all but the most finely tuned cases this bound is incompatible with
a lower bound derived for inflationary models which exhibit a red spectrum and detectable
non-gaussianity. 


By analysing general non-canonical actions, a class of models is found which can
evade the upper bound when the phase speed of perturbations is small. The multi-coincident brane
scenario with a finite number of branes is one such model. An iterative technique is used to
determine the action for this case.
For models with a possibly observable gravitational wave spectrum the number of coincident branes is
shown to take only small values. 

% Third paper
The second method of constraining inflationary models is the numerical calculation
of second order perturbations for a general class of single field models.
The Klein-Gordon equation at second order is numerically solved in closed form. 
The slow-roll version of the second order source term is used and the method is
shown to be extendable to the full equation.
This procedure allows the evolution of second order
perturbations in general and the calculation of the non-gaussianity parameter in cases
where there is no analytical solution available.  

\section*{Table of contents}
\begin{verbatim}
1. Introduction
        Timescale: 1/2 month
1.1. The Friedmann-Robertson-Walker Universe
1.2. Canonical Slow Roll Inflation
1.3. Non-Canonical Inflation
1.4. Non-Gaussianity
1.5. Current Observations
1.6. Outline and goals

I. DBI inflation

2. Introduction to Dirac Born Infeld Inflation
        Timescale: draft version completed
2.1. Introduction
2.2. String theory and extra dimensions
2.2.1. Cosmology and String Theory
2.2.2. Extra dimensions
2.2.3. T-duality
2.2.4. D-branes
2.2.5. Warped throats
2.3. DBI inflation
2.4. The Lyth bound

3. Observational Bounds on DBI Inflation
        Timescale: 1/2 month, consists of already published paper
3.1. An Upper Bound on the Primordial Gravitational Waves
3.2. A Lower Bound on the Primordial Gravitational Waves
3.3. Relaxing the Baumann-McAllister Bound
3.4. Review of other models evading the bounds 
3.5. Conclusion

4. Multi brane inflation
        Timescale: 1/2 month, consists of already published paper
4.1. Introduction
4.2. IR Inflation and Large n Limit
4.3. Theoretic Upper Bounds on the Tensor-Scalar Ratio
4.4. Relaxing the Upper Bounds on the Tensor-Scalar Ratio
4.5. Action for Multiple Coincident Branes
4.6. Bounds on the Tensor-Scalar Ratio for Multi-Brane Inflation
4.7. Discussion

II. Numerical simulations of perturbations

5. Second Order Perturbations
        Timescale: 1/2 month, consists of already published paper
5.1. Introduction
5.2. Perturbations
5.2.1. First and Second Order
5.2.2. Slow Roll approximation

6. Numerical simulations
        Timescale: 1/2 month, consists of already published paper
6.1. Equations
6.2. Initial Conditions
6.3. Implementation
6.4. Code Tests
6.5. Results
6.6. Discussion and conclusion

7. Overall Conclusions

Bibliography
\end{verbatim}
\end{document}
