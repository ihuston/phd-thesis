% % % % % % % % % % % % % % % % % % % % % % % % % % % % 
% mainintro.tex - Ian Huston
% $Id: mainintro.tex,v 1.4 2009/07/28 10:14:07 ith Exp $
% % % % % % % % % % % % % % % % % % % % % % % % % % % % 
% Redefine CVSRevision for this section
\renewcommand{\CVSrevision}{\version$Id: mainintro.tex,v 1.4 2009/07/28 10:14:07 ith Exp $}

% % % % % % % % % % % % % % % % % % % % % % % % % % % % % % % % 
% =========================================================== %
% % % % % % % % % % % % % % % % % % % % % % % % % % % % % % % % 
\chapter{Introduction}
\label{ch:introduction}
% % % % % % % % % % % % % % % % % % % % % % % % % % % % % % % % 
% =========================================================== %
% % % % % % % % % % % % % % % % % % % % % % % % % % % % % % % % 

% % % % % % % % % % % % % % % % % % % % % % % % % % % % % % % % 
% =========================================================== %
% % % % % % % % % % % % % % % % % % % % % % % % % % % % % % % % 
\section{Conventions}
\label{sec:conventions}
% % % % % % % % % % % % % % % % % % % % % % % % % % % % % % % % 
% =========================================================== %
% % % % % % % % % % % % % % % % % % % % % % % % % % % % % % % % 
Units are chosen such that $\Mpl \equiv (8\pi G )^{-1/2}= 2 
.4 \times 10^{18}\, {\rm GeV}$ 
defines the reduced Planck mass and $c=\hbar =1$. 
