% % % % % % % % % % % % % % % % % % % % % % % % % % % % 
% mainintro.tex - Ian Huston
% $Id: mainintro.tex,v 1.10 2009/08/06 15:32:42 ith Exp $
% % % % % % % % % % % % % % % % % % % % % % % % % % % % 
% Redefine CVSRevision for this section
\renewcommand{\CVSrevision}{\version$Id: mainintro.tex,v 1.10 2009/08/06 15:32:42 ith Exp $}

% % % % % % % % % % % % % % % % % % % % % % % % % % % % % % % % 
% =========================================================== %
% % % % % % % % % % % % % % % % % % % % % % % % % % % % % % % % 
\chapter{Introduction}
\label{ch:introduction}
% % % % % % % % % % % % % % % % % % % % % % % % % % % % % % % % 
% =========================================================== %
% % % % % % % % % % % % % % % % % % % % % % % % % % % % % % % % 
\section{The Friedmann-Robertson-Walker Universe}
\label{sec:frw-intro}
% % % % % % % % % % % % % % % % % % % % % % % % % % % % % % % % 
% =========================================================== %
% % % % % % % % % % % % % % % % % % % % % % % % % % % % % % % % 
\section{Canonical Slow roll inflation}
\label{sec:slowroll-intro}
% % % % % % % % % % % % % % % % % % % % % % % % % % % % % % % % 
% =========================================================== %
% % % % % % % % % % % % % % % % % % % % % % % % % % % % % % % % 

% % % % % % % % % % % % % % % % % % % % % % % % % % % % % % % % 
% =========================================================== %
% % % % % % % % % % % % % % % % % % % % % % % % % % % % % % % % 
\section{Non-Canonical Inflation} 
\label{sec:noncanoninfl}
% % % % % % % % % % % % % % % % % % % % % % % % % % % % % % % % 
% =========================================================== %
% % % % % % % % % % % % % % % % % % % % % % % % % % % % % % % % 


The low-energy, world-volume dynamics of a 
${\rm D3}$-brane in a warped background is determined 
by an effective action of the form 
% 
\begin{equation}
\label{eq:DBIaction-dbiintro2}
S=\int  d^4x \sqrt{|g|} \left[ \frac{\Mpl^2}{2} R 
+ P (\varphi , X) \right] \,,
\end{equation}
% 
where $R$ is the Ricci curvature scalar, 
$X \equiv -\frac{1}{2}g^{\mu \nu}\nabla_\mu \varphi \nabla_\nu \varphi$
denotes the kinetic energy of the inflaton field $\varphi$, and the function  
$P (\varphi , X)$ is referred to as the `kinetic function'.  


We assume that the four-dimensional universe is   
spatially flat and isotropic and sourced by an  
homogeneous inflaton field, $\varphi =\varphi (t)$, with energy 
density $E = 2X\PX - P$, where a subscripted comma denotes partial
differentiation. 
We further assume that the inflaton dynamics  
generates a quasi-exponential expansion of the universe 
where $\epsilon_H \equiv -\dot{H}/H^2 \ll1$,
as described in Section~\ref{sec:frw-intro}. 


It proves convenient to define two parameters in terms of the 
kinetic  function $P$ and its derivatives \cite{lidser1,lidser3}: 
% 
\begin{eqnarray}
\label{eq:defcs-dbiintro}
 c_s^2 \equiv \frac{\PX}{\PX + 2X P_{,XX}} \,,
\\
\label{eq:deflambda-dbiintro}
\Lambda \equiv  \frac{X^2 P_{,XX} +
\frac{2}{3}X^3 P_{,XXX}}{X P_{,X} +
2X^2 P_{,XX}}\,.
\end{eqnarray}
% 
The first parameter, $c_s$, determines the sound speed of fluctuations 
in the inflaton field. This can be significantly less than unity, 
in contrast to slow-roll inflation driven by a canonical 
field such that $P_{,X} =1$.
The amplitudes of the scalar and tensor perturbations 
generated during inflation change in this case and are given by \cite{gm}
% 
\begin{eqnarray} 
\label{eq:Ps-dbiintro}
 \Ps = \frac{H^4}{8\pi^2 X}\frac{1}{c_s \PX} \,,
\\
\label{eq:Pt-dbiintro}
\Pt = \frac{2}{\pi^2} \frac{H^2}{\Mpl^2} \,,
\end{eqnarray}
% 
respectively, and the ratio of these amplitudes 
is defined as \cite{gm} 
% 
\begin{equation}
\label{eq:rdefn-dbiintro}
r\equiv \frac{\Pt}{\Ps} = 16c_s \epsilon_H \,.
\end{equation}
%   
The WMAP5 normalization of the CMB power spectrum 
implies that $\Ps= 2.5\times10^{-9}$ and 
the experimental upper bound on the tensor-scalar 
ratio is $r <0.55$ \cite{Komatsu:2008hk}.
\addtodo{Change to r<0.20 for no-running or r<0.54 for running (WMAP5+BAO+SN).
Does this change any later calculations?}

Deviations from Gaussian statistics in the curvature perturbation, ${\cal{R}}$,
are parametrized in terms of the non-linearity parameter, 
$\fnl$, as defined in Section~\ref{sec:fnl-intro} by 
${\cal{R}} = {\cal{R}}_G + \frac{3}{5} \fnl  (
{\cal{R}}_G^2 -\langle {\cal{R}}_G^2 \rangle )$. Here the 
quadratic component represents a convolution and 
${\cal{R}}_G$ denotes the Gaussian contribution \cite{maldacena}\footnotemark.
\footnotetext{We use the WMAP sign convention for $\fnl$ throughout. 
This is the opposite of the Maldacena convention:
$\fnl^\mathrm{WMAP}=-\fnl^\mathrm{Maldacena}$.}
 In the limit  
where the three momenta have equal magnitude (corresponding to the equilateral  
triangle limit), the leading-order contribution to the non-linearity 
parameter is given by \cite{chenetal,lidser3}
% 
\begin{equation} 
\label{eq:fnldefn-dbiintro}
 \fnl = -\frac{35}{108}\left(\frac{1}{c_s^2} -1 \right) +
\frac{5}{81}\left( \frac{1}{c_s^2} -1 -2\Lambda \right) \,.
\end{equation}
%  
One should note that the sign convention is that employed
by the WMAP data set.
Data from WMAP3 imposes the bound $|\fnl| < 300$ on this parameter
\cite{spergel}. The corresponding bounds for other triangle configurations 
may be much tighter than this and this may be particularly relevant if 
non-Gaussian signatures have indeed been detected in the 
CMB \cite{Yadav:2007yy,crim}. The more recent WMAP5 data set
\cite{Komatsu:2008hk} improves on this bound somewhat, and
also indicates that it is distinctly asymmetric. At the $95 \%$ confidence
level, the bound on the 
equilateral triangle becomes $-151 < \fnl < 253$.


% % % % % % % % % % % % % % % % % % % % % % % % % % % % % % % % 
% =========================================================== %
% % % % % % % % % % % % % % % % % % % % % % % % % % % % % % % % 
\section{Non-gaussianity}
\label{sec:fnl-intro}
% % % % % % % % % % % % % % % % % % % % % % % % % % % % % % % % 
% =========================================================== %
% % % % % % % % % % % % % % % % % % % % % % % % % % % % % % % % 

% % % % % % % % % % % % % % % % % % % % % % % % % % % % % % % % 
% =========================================================== %
% % % % % % % % % % % % % % % % % % % % % % % % % % % % % % % % 
\section{Current observations}
\label{sec:obs-intro}
% % % % % % % % % % % % % % % % % % % % % % % % % % % % % % % % 
% =========================================================== %
% % % % % % % % % % % % % % % % % % % % % % % % % % % % % % % % 

% % % % % % % % % % % % % % % % % % % % % % % % % % % % % % % % 
% =========================================================== %
% % % % % % % % % % % % % % % % % % % % % % % % % % % % % % % % 
\section{Using non-gaussianity to test models}
\label{sec:fnltest-intro}
% % % % % % % % % % % % % % % % % % % % % % % % % % % % % % % % 
% =========================================================== %
% % % % % % % % % % % % % % % % % % % % % % % % % % % % % % % % 


% 
% 



% % % % % % % % % % % % % % % % % % % % % % % % % % % % % % % % 
% =========================================================== %
% % % % % % % % % % % % % % % % % % % % % % % % % % % % % % % % 
\section{Outline and goals}
\label{sec:goals-intro}
% % % % % % % % % % % % % % % % % % % % % % % % % % % % % % % % 
% =========================================================== %
% % % % % % % % % % % % % % % % % % % % % % % % % % % % % % % % 

% % % % % % % % % % % % % % % % % % % % % % % % % % % % % % % % 
% =========================================================== %
% % % % % % % % % % % % % % % % % % % % % % % % % % % % % % % % 
\section{Conventions}
\label{sec:conventions}
% % % % % % % % % % % % % % % % % % % % % % % % % % % % % % % % 
% =========================================================== %
% % % % % % % % % % % % % % % % % % % % % % % % % % % % % % % % 
Units are chosen such that $\Mpl \equiv (8\pi G )^{-1/2}= 2 
.4 \times 10^{18}\, {\rm GeV}$ 
defines the reduced Planck mass and $c=\hbar =1$. 
