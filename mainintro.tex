% % % % % % % % % % % % % % % % % % % % % % % % % % % % 
% mainintro.tex - Ian Huston
% $Id: mainintro.tex,v 1.17 2009/08/18 18:08:32 ith Exp $
% % % % % % % % % % % % % % % % % % % % % % % % % % % % 
% Redefine CVSRevision for this section
\renewcommand{\CVSrevision}{\version$Id: mainintro.tex,v 1.17 2009/08/18 18:08:32 ith Exp $}

% % % % % % % % % % % % % % % % % % % % % % % % % % % % % % % % 
% =========================================================== %
% % % % % % % % % % % % % % % % % % % % % % % % % % % % % % % % 
\chapter{Introduction}
\label{ch:introduction}
% % % % % % % % % % % % % % % % % % % % % % % % % % % % % % % % 
% =========================================================== %
% % % % % % % % % % % % % % % % % % % % % % % % % % % % % % % % 
\begin{itemize}
 \item General wordy introduction
 \item Inflation very successful
 \item Non-gaussianity as new tool
 \item Outline of intro chapter
\end{itemize}

% % % % % % % % % % % % % % % % % % % % % % % % % % % % % % % % 
% =========================================================== %
% % % % % % % % % % % % % % % % % % % % % % % % % % % % % % % % 
\section{The Friedmann-Robertson-Walker Universe}
\label{sec:frw-intro}
% % % % % % % % % % % % % % % % % % % % % % % % % % % % % % % % 
% =========================================================== %
% % % % % % % % % % % % % % % % % % % % % % % % % % % % % % % % 
The central idea of Friedmann-Robertson-Walker (FRW) Universe is the
cosmological principle. According to this postulate there is no privileged
place in the universe and no privileged direction in which one can look. This
is formalised by assuming that the universe is homogeneous and isotropic, at
least on large scales. 
\addtodo{Cite SDSS, 2dF, etc and then Bianchi models, voids etc.}
Surveys of the observable universe indicate that this is
indeed the case at the largest scales accessible. Historically homogeneity and
isotropy were assumed for simplicity and there are many
other cosmologies that do not take this approach.


By imposing both conditions on a generic metric, the line element of the
FRW universe is obtained:
\begin{equation}
 \label{eq:frwmetric-intro}
\d s^2 = -\d t^2 + a^2(t)
  \left( 
    \frac{\d r^2}{1-Kr^2} + r^2\left(\d\theta^2 + \sin^2(\theta) \d\phi^2\right)
  \right)\,,
\end{equation}
where $K=+1, 0$ or $-1$ depending on whether the universe is closed, flat
or open respectively. We will consider only the flat universe and take $K=0$
from now on. 


The spatial part of the FRW metric is multiplied by the term $a(t)$ known as the
scale factor. This characterises the size of space-like hypersurfaces at
different coordinate times. In an expanding universe $a$ grows with $t$ and
$\dot{a}>0$ where an overdot represents differentiation with respect to $t$.
The definition of the Hubble parameter $H$ captures this expansion (or
contraction):
% Definition of Hubble parameter
\begin{equation}
 \label{eq:Hdefn-intro}
  H = \frac{\dot{a}}{a} \,.
\end{equation}
% 
Instead of using coordinate time we can bring the scale factor $a$ outside the
whole metric and instead use conformal time $\eta$ defined by
% Conformal time
\begin{equation}
\label{eq:etatime-intro}
 \eta = \int\frac{\d t}{a}\,.
\end{equation}
A prime ($'$) signifies differentiation with respect to $\eta$.

The Einstein equations can be derived by the variational principle from the
action $S\equiv S_\mathrm{EH} + S_\mathrm{M}$. This is a combination of the
Einstein-Hilbert and matter actions which are defined as 
% 
\begin{eqnarray}
\label{eq:EHeqn-intro}
 S_\mathrm{EH} &=& \frac{1}{16\pi G}\int \d^4 x \sqrt{|g|} \left(R +
2\Lambda\right) \,,\\
\label{eq:matteraction-intro}
 S_\mathrm{M} &=& \int \d^4 x \sqrt{|g|}
\mathcal{L}_\mathrm{M} \,.
\end{eqnarray}
% 
Here $g$ is the determinant of the metric $g_{\mu\nu}$
and $\mathcal{L}_\mathrm{M}$ is the sum of the Lagrangian densities
for all the matter fields. 


We can now write down the Einstein equations for a general matter Lagrangian:
% 
\begin{equation}
\label{eq:einstein-intro}
 R_{\mu\nu} - \frac{1}{2}R g_{\mu\nu} = 8\pi G T_{\mu\nu} + \Lambda
g_{\mu\nu}\,,
\end{equation}
% 
where $T_{\mu\nu}$ is the stress energy tensor obtained by the variation of the
matter Lagrangian. 
In the definition above we have included a
cosmological constant term, $\Lambda$, for completeness. From now on we will
disregard the negligible contribution of such a term in the early universe.
We concentrate now on the case of a universe filled with a perfect
fluid. Suppose $u^\mu$ is the 4-velocity of this fluid with $u^\mu
u_\mu=-1$. The stress-energy tensor of the fluid is
% 
\begin{equation}
 \label{eq:fluidstress-intro}
  T^\mu_\nu = (E + P)u^\mu u_\nu + P\delta^\mu_\nu \,,
\end{equation}
% 
where $E$ is the matter energy density and $P$ is the isotropic pressure.
The trace of $T$ is given by
% 
\begin{equation}
 \label{eq:Ttrace-intro}
  T^\mu_\mu = -E + 3P\,.
\end{equation}
% 

We now use the Einstein equations and the stress-energy tensor of the
perfect fluid to derive the equations of motion of the fluid.
From the metric in \eq{eq:frwmetric-intro} the $00$ and $ij$ components of the
Ricci tensor can be found:
% 
\begin{eqnarray}
\label{eq:Ricci00-intro}
 R_{00} &=& -3 \frac{\ddot{a}}{a} \,,\\
\label{eq:Ricciij-intro}
 R_{ij} &=& \delta_{ij} \left[ 2\dot{a}^2 +
  a \ddot{a} \right] \,.
\end{eqnarray}
% 
The Friedmann equations can then be determined from the Einstein equations
\eqref{eq:einstein-intro}.
The $00$ equation gives
% 
\begin{equation}
 \label{eq:Friedmann1-intro}
 H^2 = \left(\frac{\dot{a}}{a}\right)^2 = \frac{8\pi G}{3} E \,,
\end{equation}
% 
while the trace of the Einstein equations gives
% 
\begin{equation}
 \label{eq:Friedmann2-intro}
 \frac{\ddot{a}}{a}  = -\frac{1}{2} \frac{8\pi G}{3}(E + 3P)\,.
\end{equation}
% 
By combining these two equations we can determine a continuity equation for the
energy density:
\begin{equation}
 \label{eq:continuity-intro}
 \frac{\d E}{\d t} + 3H(E+P) = 0 \,.
\end{equation}




% % % % % % % % % % % % % % % % % % % % % % % % % % % % % % % % 
% =========================================================== %
% % % % % % % % % % % % % % % % % % % % % % % % % % % % % % % % 
\subsection{Conventions}
\label{sec:conventions}
% % % % % % % % % % % % % % % % % % % % % % % % % % % % % % % % 
% =========================================================== %
% % % % % % % % % % % % % % % % % % % % % % % % % % % % % % % % 
Units are chosen such that $\Mpl \equiv (8\pi G )^{-1/2}= 2 
.4 \times 10^{18}\, {\rm GeV}$ 
defines the reduced Planck mass and $c=\hbar =1$. 

An overdot ($\dot{}$) is used for differentiation with respect to coordinate
time $t$ and a prime ($'$) for differentiation with respect to conformal time
$\eta$. A subscripted comma denotes partial differentiation by the symbol it
precedes, \eg $f_{,\varphi} = \dfrac{\d f}{\d \varphi}$.

% % % % % % % % % % % % % % % % % % % % % % % % % % % % % % % % 
% =========================================================== %
% % % % % % % % % % % % % % % % % % % % % % % % % % % % % % % % 
\section{Inflation}
\label{sec:inflation-intro}
% % % % % % % % % % % % % % % % % % % % % % % % % % % % % % % % 
% =========================================================== %
% % % % % % % % % % % % % % % % % % % % % % % % % % % % % % % % 
\begin{itemize}
 \item Problems with standard big band
 \item Period of exponential expansion
\end{itemize}
\begin{equation}
 \label{eq:addot-intro}
 \ddot{a} > 0
\end{equation}

For a perfect fluid as described in Section~\ref{sec:frw-intro} this requires
% 
\begin{equation}
 E + 3P <0\,,
\end{equation}
% 
which violates the Strong Energy Condition [cite*****].

Most straightforward way to achieve this is fill universe with a single scalar
field $\varphi$. The canonical action for this field is
\begin{equation}
 \mathcal{L}_\mathrm{M} \equiv P(\varphi, X) = X -V(\varphi) \,,
\end{equation}
where $-\frac{1}{2}g_{\mu\nu}\partial^\mu\varphi \partial^\nu\varphi$ denotes
the kinetic energy of $\varphi$ and $P(\varphi, X)$ is called the ``kinetic
function''. In Section~\ref{sec:noncanoninfl} we will consider other
choices for $P$. 

With the canonical choice of $P=X$ and for a homogeneous field
$\varphi=\varphi(t)$ we have the following relations for the matter
energy-density and isotropic pressure:
\begin{eqnarray}
 E &=& \frac{1}{2}\dot{\varphi}^2 + V(\varphi) \,,\\
 P &=& \frac{1}{2}\dot{\varphi}^2 - V(\varphi) \,.
\end{eqnarray}
Under these conditions the kinetic function $P(\varphi, X)$ can be identified as
the isotropic pressure.

The dynamics of the field are governed by the potential $V(\varphi)$. 

To solve the horizon problem the universe must expand by a factor of
approximately $e^{60}$ during inflation. 

Define efoldings

Bg eqns of motion for $\varphi$


% % % % % % % % % % % % % % % % % % % % % % % % % % % % % % % % 
% =========================================================== %
% % % % % % % % % % % % % % % % % % % % % % % % % % % % % % % % 
\subsection{Perturbations}
\label{sec:perts-intro}
% % % % % % % % % % % % % % % % % % % % % % % % % % % % % % % % 
% =========================================================== %
% % % % % % % % % % % % % % % % % % % % % % % % % % % % % % % % 
Split






% % % % % % % % % % % % % % % % % % % % % % % % % % % % % % % % 
% =========================================================== %
% % % % % % % % % % % % % % % % % % % % % % % % % % % % % % % % 
\subsection{Canonical Slow roll inflation}
\label{sec:slowroll-intro}
% % % % % % % % % % % % % % % % % % % % % % % % % % % % % % % % 
% =========================================================== %
% % % % % % % % % % % % % % % % % % % % % % % % % % % % % % % % 
\begin{itemize}
 \item Canonical action
 \item Scalar fields $\rho$ and $P$
 \item Friedmann eqns for $\varphi$
 \item Slow roll params
 \item Perturbation theory, general and first order
 \item $\delta\rho/\rho$ connection to $\delta \varphi$
 \item Initial conditions, Bunch-Davies vacuum
 \item  Power spectra, spectral index
\end{itemize}


% % % % % % % % % % % % % % % % % % % % % % % % % % % % % % % % 
% =========================================================== %
% % % % % % % % % % % % % % % % % % % % % % % % % % % % % % % % 
\subsection{Non-Canonical Inflation} 
\label{sec:noncanoninfl}
% % % % % % % % % % % % % % % % % % % % % % % % % % % % % % % % 
% =========================================================== %
% % % % % % % % % % % % % % % % % % % % % % % % % % % % % % % % 


The low-energy, world-volume dynamics of a 
${\rm D3}$-brane in a warped background is determined 
by an effective action of the form 
% 
\begin{equation}
\label{eq:DBIaction-dbiintro2}
S=\int  d^4x \sqrt{|g|} \left[ \frac{\Mpl^2}{2} R 
+ P (\varphi , X) \right] \,,
\end{equation}
% 
where $R$ is the Ricci curvature scalar, 
$X \equiv -\frac{1}{2}g^{\mu \nu}\nabla_\mu \varphi \nabla_\nu \varphi$
denotes the kinetic energy of the inflaton field $\varphi$, and the function  
$P (\varphi , X)$ is referred to as the `kinetic function'.  


We assume that the four-dimensional universe is   
spatially flat and isotropic and sourced by an  
homogeneous inflaton field, $\varphi =\varphi (t)$, with energy 
density $E = 2X\PX - P$, where a subscripted comma denotes partial
differentiation. 
We further assume that the inflaton dynamics  
generates a quasi-exponential expansion of the universe 
where $\epsilon_H \equiv -\dot{H}/H^2 \ll1$,
as described in Section~\ref{sec:frw-intro}. 


It proves convenient to define two parameters in terms of the 
kinetic  function $P$ and its derivatives \cite{lidser1,lidser3}: 
% 
\begin{eqnarray}
\label{eq:defcs-dbiintro}
 c_s^2 \equiv \frac{\PX}{\PX + 2X P_{,XX}} \,,
\\
\label{eq:deflambda-dbiintro}
\Lambda \equiv  \frac{X^2 P_{,XX} +
\frac{2}{3}X^3 P_{,XXX}}{X P_{,X} +
2X^2 P_{,XX}}\,.
\end{eqnarray}
% 
The first parameter, $c_s$, determines the sound speed of fluctuations 
in the inflaton field. This can be significantly less than unity, 
in contrast to slow-roll inflation driven by a canonical 
field such that $P_{,X} =1$.
The amplitudes of the scalar and tensor perturbations 
generated during inflation change in this case and are given by \cite{gm}
% 
\begin{eqnarray} 
\label{eq:Ps-dbiintro}
 \Ps = \frac{H^4}{8\pi^2 X}\frac{1}{c_s \PX} \,,
\\
\label{eq:Pt-dbiintro}
\Pt = \frac{2}{\pi^2} \frac{H^2}{\Mpl^2} \,,
\end{eqnarray}
% 
respectively, and the ratio of these amplitudes 
is defined as \cite{gm} 
% 
\begin{equation}
\label{eq:rdefn-dbiintro}
r\equiv \frac{\Pt}{\Ps} = 16c_s \epsilon_H \,.
\end{equation}
%   
The WMAP5 normalization of the CMB power spectrum 
implies that $\Ps= 2.5\times10^{-9}$ and 
the experimental upper bound on the tensor-scalar 
ratio is $r <0.55$ \cite{Komatsu:2008hk}.
\addtodo{Change to r<0.20 for no-running or r<0.54 for running (WMAP5+BAO+SN).
Does this change any later calculations?}

Deviations from Gaussian statistics in the curvature perturbation, ${\cal{R}}$,
are parametrized in terms of the non-linearity parameter, 
$\fnl$, as defined in Section~\ref{sec:fnl-intro} by 
${\cal{R}} = {\cal{R}}_G + \frac{3}{5} \fnl  (
{\cal{R}}_G^2 -\langle {\cal{R}}_G^2 \rangle )$. Here the 
quadratic component represents a convolution and 
${\cal{R}}_G$ denotes the Gaussian contribution \cite{maldacena}\footnotemark.
\footnotetext{We use the WMAP sign convention for $\fnl$ throughout. 
This is the opposite of the Maldacena convention:
$\fnl^\mathrm{WMAP}=-\fnl^\mathrm{Maldacena}$.}
 In the limit  
where the three momenta have equal magnitude (corresponding to the equilateral  
triangle limit), the leading-order contribution to the non-linearity 
parameter is given by \cite{chenetal,lidser3}
% 
\begin{equation} 
\label{eq:fnldefn-dbiintro}
 \fnl = -\frac{35}{108}\left(\frac{1}{c_s^2} -1 \right) +
\frac{5}{81}\left( \frac{1}{c_s^2} -1 -2\Lambda \right) \,.
\end{equation}
%  
One should note that the sign convention is that employed
by the WMAP data set.
Data from WMAP3 imposes the bound $|\fnl| < 300$ on this parameter
\cite{spergel}. The corresponding bounds for other triangle configurations 
may be much tighter than this and this may be particularly relevant if 
non-Gaussian signatures have indeed been detected in the 
CMB \cite{Yadav:2007yy,crim}. The more recent WMAP5 data set
\cite{Komatsu:2008hk} improves on this bound somewhat, and
also indicates that it is distinctly asymmetric. At the $95 \%$ confidence
level, the bound on the 
equilateral triangle becomes $-151 < \fnl < 253$.


% % % % % % % % % % % % % % % % % % % % % % % % % % % % % % % % 
% =========================================================== %
% % % % % % % % % % % % % % % % % % % % % % % % % % % % % % % % 
\section{Non-gaussianity}
\label{sec:fnl-intro}
% % % % % % % % % % % % % % % % % % % % % % % % % % % % % % % % 
% =========================================================== %
% % % % % % % % % % % % % % % % % % % % % % % % % % % % % % % % 
\begin{itemize}
 \item Deviation from uncoupled standard case
 \item Defn of $\fnl$, higher order terms
\end{itemize}


% % % % % % % % % % % % % % % % % % % % % % % % % % % % % % % % 
% =========================================================== %
% % % % % % % % % % % % % % % % % % % % % % % % % % % % % % % % 
\section{Current observations}
\label{sec:obs-intro}
% % % % % % % % % % % % % % % % % % % % % % % % % % % % % % % % 
% =========================================================== %
% % % % % % % % % % % % % % % % % % % % % % % % % % % % % % % % 
\begin{itemize}
 \item WMAP results
 \item Other ground based observations
 \item Current status of inflation in general and specific inflationary models
\end{itemize}


% % % % % % % % % % % % % % % % % % % % % % % % % % % % % % % % 
% =========================================================== %
% % % % % % % % % % % % % % % % % % % % % % % % % % % % % % % % 
\section{Outline and goals}
\label{sec:goals-intro}
% % % % % % % % % % % % % % % % % % % % % % % % % % % % % % % % 
% =========================================================== %
% % % % % % % % % % % % % % % % % % % % % % % % % % % % % % % % 
\begin{itemize}
 \item Main aims
 \item Chapter by chapter
 \item Description of two separate parts
\end{itemize}

