% % % % % % % % % % % % % % % % % % % % % % % % % % % % 
% mainintro.tex - Ian Huston
% $Id: mainintro.tex,v 1.20 2009/08/23 20:02:25 ith Exp $
% % % % % % % % % % % % % % % % % % % % % % % % % % % % 
% Redefine CVSRevision for this section
\renewcommand{\CVSrevision}{\version$Id: mainintro.tex,v 1.20 2009/08/23 20:02:25 ith Exp $}

% % % % % % % % % % % % % % % % % % % % % % % % % % % % % % % % 
% =========================================================== %
% % % % % % % % % % % % % % % % % % % % % % % % % % % % % % % % 
\chapter{Introduction}
\label{ch:introduction}
% % % % % % % % % % % % % % % % % % % % % % % % % % % % % % % % 
% =========================================================== %
% % % % % % % % % % % % % % % % % % % % % % % % % % % % % % % % 
\begin{itemize}
 \item General wordy introduction
 \item Inflation very successful
 \item Non-gaussianity as new tool
 \item Outline of intro chapter
\end{itemize}

% Hot big bang
Evidence for the expansion of the universe, $H>0$, has been in place for many
years and forms the basis of the Hot Big Bang scenario [cite reviews]. 

% % % % % % % % % % % % % % % % % % % % % % % % % % % % % % % % 
% =========================================================== %
% % % % % % % % % % % % % % % % % % % % % % % % % % % % % % % % 
\section{The Friedmann-Robertson-Walker Universe}
\label{sec:frw-intro}
% % % % % % % % % % % % % % % % % % % % % % % % % % % % % % % % 
% =========================================================== %
% % % % % % % % % % % % % % % % % % % % % % % % % % % % % % % % 
The central idea of the Friedmann-Robertson-Walker (FRW) Universe is the
cosmological principle\footnotemark. 
\footnotetext{Lema\^{i}tre is sometimes also included in this group to give
FLRW.}
According to this postulate there is no privileged
place in the universe and no privileged direction in which one can look. These
assertions
are formalised by assuming that the universe is homogeneous and isotropic at
every point. This clearly conflicts with our experience of the small part of
the universe around us but is regarded to hold as larger and larger scales are
considered.
\addtodo{Cite SDSS, 2dF, etc and then Bianchi models, voids etc.}
Surveys of the observable universe indicate that this is
indeed the case up to the largest scales accessible. 
Historically homogeneity and isotropy were assumed primarily for simplicity and
there are many other approaches can be taken. Violating these assumptions can
be done for example by specifying a preferred direction or supposing that the
universe is formed by a series of voids connected by filaments. Although many
of these approaches have been disregarded some are still
allowed by the observational evidence [cite].

In this section we will outline the dynamics of the standard Big Bang scenario.
By assuming homogeneity and isotropy as discussed the equations of motion of a
fluid-filled universe can be derived. What follows in this chapter is a
standard exposition of well-known physics and has been the subject of
numerous reviews including [cites].


By imposing both homogeneity and isotropy on a general 4-dimensional metric, the
line element of the FRW universe is obtained:
% 
\begin{equation}
 \label{eq:frwmetric-intro}
\d s^2 = -\d t^2 + a^2(t)
  \left( 
    \frac{\d r^2}{1-Kr^2} + r^2\left(\d\theta^2 + \sin^2(\theta) \d\phi^2\right)
  \right)\,,
\end{equation}
% 
where $K=+1, 0$ or $-1$ depending on whether the universe is closed, flat
or open respectively. The time-like coordinate in the metric is $t$, known as
coordinate time, and differentiation with respect to $t$ is denoted by an
overdot $(\dot{})$. 


The spatial part of the FRW metric is multiplied by the term $a(t)$ known as the
scale factor. This characterises the size of space-like hypersurfaces at
different coordinate times. In an expanding universe $a$ grows with $t$ and
$\dot{a}>0$ where an overdot represents differentiation with respect to $t$.
The definition of the Hubble parameter $H$ captures this expansion (or
contraction):
% Definition of Hubble parameter
\begin{equation}
 \label{eq:Hdefn-intro}
  H = \frac{\dot{a}}{a} \,.
\end{equation}
%


The Einstein equations can be derived by the variational principle from the
action $S\equiv S_\mathrm{EH} + S_\mathrm{M}$. This is a combination of the
Einstein-Hilbert and matter actions which are defined as 
% 
\begin{eqnarray}
\label{eq:EHeqn-intro}
 S_\mathrm{EH} &=& \frac{1}{16\pi G}\int \d^4 x \sqrt{|g|} \left(R +
2\Lambda\right) \,,\\
\label{eq:matteraction-intro}
 S_\mathrm{M} &=& \int \d^4 x \sqrt{|g|}
\mathcal{L}_\mathrm{M} \,.
\end{eqnarray}
% 
Here $g$ is the determinant of the metric $g_{\mu\nu}$
and $\mathcal{L}_\mathrm{M}$ is the sum of the Lagrangian densities
for all the matter fields.
Changing either the matter or gravity part of the action will affect the
resultant physics. In this work we focus our attention only on the matter
Lagrangian and will use the standard Einstein-Hilbert action throughout.
We can now write down the Einstein equations for a general matter Lagrangian:
% 
\begin{equation}
\label{eq:einstein-intro}
 R_{\mu\nu} - \frac{1}{2}R g_{\mu\nu} = 8\pi G T_{\mu\nu} + \Lambda
g_{\mu\nu}\,,
\end{equation}
% 
where $T_{\mu\nu}$ is the stress energy tensor obtained by the variation of the
matter Lagrangian. 
In the definitions above we have included a
cosmological constant term, $\Lambda$, for completeness. From now on we will
disregard the negligible contribution of such a term in the early universe.


We concentrate now on the case of a universe filled with a perfect
fluid. Suppose $u^\mu$ is the 4-velocity of this fluid with $u^\mu
u_\mu=-1$. The stress-energy tensor of the fluid is
% 
\begin{equation}
 \label{eq:fluidstress-intro}
  T^\mu_\nu = (E + P)u^\mu u_\nu + P\delta^\mu_\nu \,,
\end{equation}
% 
where $E$ is the matter energy density and $P$ is the isotropic pressure.
The trace of $T$ is given by
% 
\begin{equation}
 \label{eq:Ttrace-intro}
  T^\mu_\mu = -E + 3P\,.
\end{equation}
% 
The Einstein equations and the stress-energy tensor of the
perfect fluid can now be used to derive the equations of motion of the fluid.
From the metric in \eq{eq:frwmetric-intro} the $00$ and $ij$ components of the
Ricci tensor can be found:
% 
\begin{eqnarray}
\label{eq:Ricci00-intro}
 R_{00} &=& -3 \frac{\ddot{a}}{a} \,,\\
\label{eq:Ricciij-intro}
 R_{ij} &=& \delta_{ij} \left[ 2\dot{a}^2 +
  a \ddot{a} + 2\frac{K}{a^2} \right] \,.
\end{eqnarray}
% 
The Friedmann equations are then determined from the Einstein equations
\eqref{eq:einstein-intro}.
The $00$ equation gives
% 
\begin{equation}
 \label{eq:Friedmann1-intro}
 H^2 = \left(\frac{\dot{a}}{a}\right)^2 = \frac{8\pi G}{3} E - \frac{K}{a^2}\,,
\end{equation}
% 
while the trace of the Einstein equations gives
% 
\begin{equation}
 \label{eq:Friedmann2-intro}
 \frac{\ddot{a}}{a}  = -\frac{1}{2} \frac{8\pi G}{3}(E + 3P)\,.
\end{equation}
% 
By combining these two equations we can determine a continuity equation for the
energy density:
\begin{equation}
 \label{eq:continuity-intro}
 \frac{\d E}{\d t} + 3H(E+P) = 0 \,.
\end{equation}


These three equations \eqref{eq:Friedmann1-intro}, \eqref{eq:Friedmann2-intro}
and \eqref{eq:continuity-intro} will determine the evolution of the
universe-filling perfect fluid. Two important solutions are the radiation and
matter filled universes. In the standard big bang scenario the universe is
filled with radiation (to a good approximation) until matter becomes dominant
at later times. The different dominant components affect the rate of expansion
of the universe straightforwardly. Relativistic radiation has $P=E/3$ and
integrating the continuity equation \eqref{eq:continuity-intro} gives
$E_r\propto a^{-4}$. Matter conversely is taken to be dust-like with zero
pressure and so $E_m\propto a^{-3}$. The dependence of $a$ on $t$ can then be
found from \eq{eq:Friedmann1-intro}, giving $a\propto t^{1/2}$ and $a\propto
t^{2/3}$ for the radiation and matter eras respectively.


% Horizons
Instead of using coordinate time as above we could instead bring
the scale factor $a$ outside the
whole metric and instead use conformal time $\eta$ defined by
% Conformal time
\begin{equation}
\label{eq:etatime-intro}
 \eta = \int\frac{\d t}{a}\,.
\end{equation}
A prime ($'$) signifies differentiation with respect to $\eta$. 
Because all the coordinates in the line element are now scaled by $a(t)$ we
have defined a coordinate grid which does not change as the universe expands.
These ``comoving'' coordinates allow distances to be compared at different
eras with ease. A comoving distance $x$ can be translated into a physical
distance $d$ by
% 
\begin{equation}
 \label{eq:comovingdefn-intro}
 d = ax \,.
\end{equation}
The physical distance changes as the universe expands but the comoving distance
will remain fixed.

One particularly important distance is the maximum distance light could have
propagated from some earlier time $t_i$ to a later time $t$. From
\eq{eq:comovingdefn-intro} this is simply the conformal time integrated from
$t_i$ and is called the comoving or particle horizon.
If the initial time is restricted to
being at some finite time in the past, as in the Big Bang scenario, then the
particle horizon will be finite. Two points which are further apart than
this finite distance could never have been in causal contact. This
is the origin of one of the major problems with the standard Big Bang scenario
which will be discussed in the next section.
The comoving horizon can be rewritten as
% 
\begin{equation}
 \eta = \int_{a_i}^a \frac{\d a'}{a'} \frac{1}{a' H(a')} \,,
\end{equation}
which shows that it is also the logarithmic integral of the comoving Hubble
radius $1/aH$. This distance is how far particles can travel in one
``e-folding'', the time for $a$ to expand by one exponential factor. Particles
that are separated by more than the Hubble radius cannot be in causal
contact now. Particles separated by more than the
comoving horizon, however, could never have been in causal contact.






% % % % % % % % % % % % % % % % % % % % % % % % % % % % % % % % 
% =========================================================== %
% % % % % % % % % % % % % % % % % % % % % % % % % % % % % % % % 
\subsection{Conventions}
\label{sec:conventions}
% % % % % % % % % % % % % % % % % % % % % % % % % % % % % % % % 
% =========================================================== %
% % % % % % % % % % % % % % % % % % % % % % % % % % % % % % % % 
Throughout this work units are chosen such that $\Mpl \equiv (8\pi G )^{-1/2}=
2 
.4 \times 10^{18}\, {\rm GeV}$ 
defines the reduced Planck mass and $c=\hbar =1$. 

An overdot ($\dot{}$) is used for differentiation with respect to coordinate
time $t$ and a prime ($'$) for differentiation with respect to conformal time
$\eta$. A subscripted comma denotes partial differentiation by the symbol it
precedes, \eg $f_{,\varphi} = \dfrac{\d f}{\d \varphi}$.

% % % % % % % % % % % % % % % % % % % % % % % % % % % % % % % % 
% =========================================================== %
% % % % % % % % % % % % % % % % % % % % % % % % % % % % % % % % 
\section{Inflation}
\label{sec:inflation-intro}
% % % % % % % % % % % % % % % % % % % % % % % % % % % % % % % % 
% =========================================================== %
% % % % % % % % % % % % % % % % % % % % % % % % % % % % % % % % 
In this section we introduce the inflationary scenario and briefly describe how
it solves three major problems with the standard Big Bang picture. 

\subsection{Problems with the Big Bang scenario}
\label{sec:problems-intro}
Although remarkably successful in describing the evolution of the universe from
very early in its history, the standard Big Bang scenario suffers from a number
of serious problems. 

\subsubsection{Flatness Problem} 
\label{sec:flatprob}
The Friedmann equation
\eq{eq:Friedmann1-intro} can be re-written as
% 
\begin{equation}
\label{eq:omegadefn-intro}
\Omega(t) - 1 = \frac{K}{(aH)^2} \,,
\end{equation}
% 
where $\Omega(t)=E(t)/E_\mathrm{crit}$ and the critical density
$E_\mathrm{crit}= 3H^2/8\pi G$. 
If $K=0$ then $\Omega=1$ and the density remains constant. However if
$K\neq0$ the density will change with time. The flat universe is an
unstable fixed point in the parameter space.
Current observations confirm $\Omega=1$ within about 2\% \cite{Komatsu:2008hk}.
During the radiation and matter dominated eras $aH$ is decreasing with time,
meaning that $\Omega$ diverges away from 1. If it is measured now as being very
close to 1 then in the past it must have been even closer. The requirement of
extreme proximity to $\Omega=1$ as an initial condition is known as the
flatness problem. 


\subsubsection{Horizon Problem} \label{sec:horizprob}
As defined in \eq{eq:particlehorizon-intro} the particle horizon, also known
as the comoving horizon, defines the maximum separation between two points that
have been in causual contact sometime in the past. During the radiation and
matter eras this comoving horizon increases monotonically and so length-scales
which are now entering the horizon would have been far outside the horizon in
the past. 
The CMB as observed by the WMAP satellites is extremely smooth on scales that
would have been far outside the horizon at the time of last scattering
\cite{Komatsu:2008hk}. These regions of space have very similar energies and
yet they could not have been in causal contact since the Big Bang. 


\subsubsection{Cosmological Defects} \label{sec:cosdefects}
\addtodo{Citations}
The hot Big Bang achieved temperatures at which a Grand Unified Theory
(GUT) would predict a higher order symmetry gauge group. During cooling,
spontaneous symmetry breaking could leave relics that might survive to
the current day. Magnetic monopoles, cosmic strings and domain walls are some of
the unwanted features predicted by models. However none of these have been
observed and some mechanism is needed to explain their absence.

\subsection{Introduction to Inflation}
\label{sec:inflintro-intro}

Inflation is a period of accelerated expansion in the size of the universe
which takes place before the start of the standard Big Bang. During this
expansion the comoving Hubble radius $(aH)^{-1}$ decreases and the isotropic
pressure of the universe is negative:
\begin{equation}
\label{eq:infldefn-intro}
 \frac{\d }{\d t}\left( \frac{1}{aH}\right) <0 \Rightarrow \ddot{a}>0
  \Rightarrow E + 3P < 0 \,.
\end{equation}
The decrease in the Hubble radius immediately solves the horizon problem.
Comoving scales that entered the horizon recently, such as those we observe in
the CMB, would previously have been within the horizon.
We require that inflation lasted long enough that all the scales we
measure were previously inside the horizon. 
Figure~\ref{fig:comovingscales-intro} shows how comoving scales could
previously have been inside the horizon if the period of inflation was
sufficently long.
% 
\begin{figure}
 \missingfigure{Add figure of comoving scales and changing Hubble radius.}
 \caption{Comoving scales that have recently entered the horizon would
previously have been inside the horizon if inflation lasted long enough.}
 \label{fig:comovingscales-intro}
\end{figure}
% 
Now consider the time derivative of $|\Omega -1|$ as defined in
\eq{eq:omegadefn-intro}:
% 
\begin{equation}
 \label{eq:omegaderiv-intro}
 \frac{\d}{\d t}\left(|\Omega -1 |\right) = 2\frac{\d}{\d t}
\left(\frac{1}{aH}\right)\,.
\end{equation}
% 
If the universe is not flat to begin with a period of inflation will push it
towards $\Omega=1$, solving the flatness problem. 
% Keep?
To achieve the level of
flatness required to meet current observations takes less inflation than the
solution of the horizon problem requires.
% 

To solve the horizon and flatness problems the period of inflation must be
sufficiently long. Typically 50-70 e-foldings is considered standard. Inflation
can last longer than this but only the last 50-70 e-foldings will be important
for our observable universe. With the scale factor of the universe expanding
approximately $e^{60}\simeq 10^{26}$ times the observable universe would
originate in a Planck length sized initial patch [cite?].
Due to the volume increasing by a factor of $e^{180}$ the number density
of monopoles or other relics that existed before inflation would be
diluted so much that observing the relics would be increasingly unlikely.


Most straightforward way to achieve this is fill universe with a single scalar
field $\varphi$. The canonical action for this field is
\begin{equation}
 \mathcal{L}_\mathrm{M} \equiv P(\varphi, X) = X -V(\varphi) \,,
\end{equation}
where $-\frac{1}{2}g_{\mu\nu}\partial^\mu\varphi \partial^\nu\varphi$ denotes
the kinetic energy of $\varphi$ and $P(\varphi, X)$ is called the ``kinetic
function''. In Section~\ref{sec:noncanoninfl} we will consider other
choices for $P$. 

With the canonical choice of $P=X$ and for a homogeneous field
$\varphi=\varphi(t)$ we have the following relations for the matter
energy-density and isotropic pressure:
\begin{eqnarray}
 E &=& \frac{1}{2}\dot{\varphi}^2 + V(\varphi) \,,\\
 P &=& \frac{1}{2}\dot{\varphi}^2 - V(\varphi) \,.
\end{eqnarray}
Under these conditions the kinetic function $P(\varphi, X)$ can be identified as
the isotropic pressure.

The dynamics of the field are governed by the potential $V(\varphi)$. 

To solve the horizon problem the universe must expand by a factor of
approximately $e^{60}$ during inflation. 

Define efoldings

Bg eqns of motion for $\varphi$


% % % % % % % % % % % % % % % % % % % % % % % % % % % % % % % % 
% =========================================================== %
% % % % % % % % % % % % % % % % % % % % % % % % % % % % % % % % 
\subsection{Perturbations}
\label{sec:perts-intro}
% % % % % % % % % % % % % % % % % % % % % % % % % % % % % % % % 
% =========================================================== %
% % % % % % % % % % % % % % % % % % % % % % % % % % % % % % % % 
Split






% % % % % % % % % % % % % % % % % % % % % % % % % % % % % % % % 
% =========================================================== %
% % % % % % % % % % % % % % % % % % % % % % % % % % % % % % % % 
\subsection{Canonical Slow roll inflation}
\label{sec:slowroll-intro}
% % % % % % % % % % % % % % % % % % % % % % % % % % % % % % % % 
% =========================================================== %
% % % % % % % % % % % % % % % % % % % % % % % % % % % % % % % % 
\begin{itemize}
 \item Canonical action
 \item Scalar fields $\rho$ and $P$
 \item Friedmann eqns for $\varphi$
 \item Slow roll params
 \item Perturbation theory, general and first order
 \item $\delta\rho/\rho$ connection to $\delta \varphi$
 \item Initial conditions, Bunch-Davies vacuum
 \item  Power spectra, spectral index
\end{itemize}


% % % % % % % % % % % % % % % % % % % % % % % % % % % % % % % % 
% =========================================================== %
% % % % % % % % % % % % % % % % % % % % % % % % % % % % % % % % 
\subsection{Non-Canonical Inflation} 
\label{sec:noncanoninfl}
% % % % % % % % % % % % % % % % % % % % % % % % % % % % % % % % 
% =========================================================== %
% % % % % % % % % % % % % % % % % % % % % % % % % % % % % % % % 


The low-energy, world-volume dynamics of a 
${\rm D3}$-brane in a warped background is determined 
by an effective action of the form 
% 
\begin{equation}
\label{eq:DBIaction-dbiintro2}
S=\int  d^4x \sqrt{|g|} \left[ \frac{\Mpl^2}{2} R 
+ P (\varphi , X) \right] \,,
\end{equation}
% 
where $R$ is the Ricci curvature scalar, 
$X \equiv -\frac{1}{2}g^{\mu \nu}\nabla_\mu \varphi \nabla_\nu \varphi$
denotes the kinetic energy of the inflaton field $\varphi$, and the function  
$P (\varphi , X)$ is referred to as the `kinetic function'.  


We assume that the four-dimensional universe is   
spatially flat and isotropic and sourced by an  
homogeneous inflaton field, $\varphi =\varphi (t)$, with energy 
density $E = 2X\PX - P$, where a subscripted comma denotes partial
differentiation. 
We further assume that the inflaton dynamics  
generates a quasi-exponential expansion of the universe 
where $\epsilon_H \equiv -\dot{H}/H^2 \ll1$,
as described in Section~\ref{sec:frw-intro}. 


It proves convenient to define two parameters in terms of the 
kinetic  function $P$ and its derivatives \cite{lidser1,lidser3}: 
% 
\begin{eqnarray}
\label{eq:defcs-dbiintro}
 c_s^2 \equiv \frac{\PX}{\PX + 2X P_{,XX}} \,,
\\
\label{eq:deflambda-dbiintro}
\Lambda \equiv  \frac{X^2 P_{,XX} +
\frac{2}{3}X^3 P_{,XXX}}{X P_{,X} +
2X^2 P_{,XX}}\,.
\end{eqnarray}
% 
The first parameter, $c_s$, determines the sound speed of fluctuations 
in the inflaton field. This can be significantly less than unity, 
in contrast to slow-roll inflation driven by a canonical 
field such that $P_{,X} =1$.
The amplitudes of the scalar and tensor perturbations 
generated during inflation change in this case and are given by \cite{gm}
% 
\begin{eqnarray} 
\label{eq:Ps-dbiintro}
 \Ps = \frac{H^4}{8\pi^2 X}\frac{1}{c_s \PX} \,,
\\
\label{eq:Pt-dbiintro}
\Pt = \frac{2}{\pi^2} \frac{H^2}{\Mpl^2} \,,
\end{eqnarray}
% 
respectively, and the ratio of these amplitudes 
is defined as \cite{gm} 
% 
\begin{equation}
\label{eq:rdefn-dbiintro}
r\equiv \frac{\Pt}{\Ps} = 16c_s \epsilon_H \,.
\end{equation}
%   
The WMAP5 normalization of the CMB power spectrum 
implies that $\Ps= 2.5\times10^{-9}$ and 
the experimental upper bound on the tensor-scalar 
ratio is $r <0.55$ \cite{Komatsu:2008hk}.
\addtodo{Change to r<0.20 for no-running or r<0.54 for running (WMAP5+BAO+SN).
Does this change any later calculations?}

Deviations from Gaussian statistics in the curvature perturbation, ${\cal{R}}$,
are parametrized in terms of the non-linearity parameter, 
$\fnl$, as defined in Section~\ref{sec:fnl-intro} by 
${\cal{R}} = {\cal{R}}_G + \frac{3}{5} \fnl  (
{\cal{R}}_G^2 -\langle {\cal{R}}_G^2 \rangle )$. Here the 
quadratic component represents a convolution and 
${\cal{R}}_G$ denotes the Gaussian contribution \cite{maldacena}\footnotemark.
\footnotetext{We use the WMAP sign convention for $\fnl$ throughout. 
This is the opposite of the Maldacena convention:
$\fnl^\mathrm{WMAP}=-\fnl^\mathrm{Maldacena}$.}
 In the limit  
where the three momenta have equal magnitude (corresponding to the equilateral  
triangle limit), the leading-order contribution to the non-linearity 
parameter is given by \cite{chenetal,lidser3}
% 
\begin{equation} 
\label{eq:fnldefn-dbiintro}
 \fnl = -\frac{35}{108}\left(\frac{1}{c_s^2} -1 \right) +
\frac{5}{81}\left( \frac{1}{c_s^2} -1 -2\Lambda \right) \,.
\end{equation}
%  
One should note that the sign convention is that employed
by the WMAP data set.
Data from WMAP3 imposes the bound $|\fnl| < 300$ on this parameter
\cite{spergel}. The corresponding bounds for other triangle configurations 
may be much tighter than this and this may be particularly relevant if 
non-Gaussian signatures have indeed been detected in the 
CMB \cite{Yadav:2007yy,crim}. The more recent WMAP5 data set
\cite{Komatsu:2008hk} improves on this bound somewhat, and
also indicates that it is distinctly asymmetric. At the $95 \%$ confidence
level, the bound on the 
equilateral triangle becomes $-151 < \fnl < 253$.


% % % % % % % % % % % % % % % % % % % % % % % % % % % % % % % % 
% =========================================================== %
% % % % % % % % % % % % % % % % % % % % % % % % % % % % % % % % 
\section{Non-gaussianity}
\label{sec:fnl-intro}
% % % % % % % % % % % % % % % % % % % % % % % % % % % % % % % % 
% =========================================================== %
% % % % % % % % % % % % % % % % % % % % % % % % % % % % % % % % 
\begin{itemize}
 \item Deviation from uncoupled standard case
 \item Defn of $\fnl$, higher order terms
\end{itemize}


% % % % % % % % % % % % % % % % % % % % % % % % % % % % % % % % 
% =========================================================== %
% % % % % % % % % % % % % % % % % % % % % % % % % % % % % % % % 
\section{Current observations}
\label{sec:obs-intro}
% % % % % % % % % % % % % % % % % % % % % % % % % % % % % % % % 
% =========================================================== %
% % % % % % % % % % % % % % % % % % % % % % % % % % % % % % % % 
\begin{itemize}
 \item WMAP results
 \item Other ground based observations
 \item Current status of inflation in general and specific inflationary models
\end{itemize}


% % % % % % % % % % % % % % % % % % % % % % % % % % % % % % % % 
% =========================================================== %
% % % % % % % % % % % % % % % % % % % % % % % % % % % % % % % % 
\section{Outline and goals}
\label{sec:goals-intro}
% % % % % % % % % % % % % % % % % % % % % % % % % % % % % % % % 
% =========================================================== %
% % % % % % % % % % % % % % % % % % % % % % % % % % % % % % % % 
\begin{itemize}
 \item Main aims
 \item Chapter by chapter
 \item Description of two separate parts
\end{itemize}

