% % % % % % % % % % % % % % % % % % % % % % % % % % % % 
% shortintro.tex - Ian Huston
% $Id: shortintro.tex,v 1.8 2009/11/16 16:58:58 ith Exp $
% % % % % % % % % % % % % % % % % % % % % % % % % % % % 
% Redefine CVSRevision for this section
\renewcommand{\CVSrevision}%
{\version$Id: shortintro.tex,v 1.8 2009/11/16 16:58:58 ith Exp $}

% % % % % % % % % % % % % % % % % % % % % % % % % % % % % % % % 
% =========================================================== %
% % % % % % % % % % % % % % % % % % % % % % % % % % % % % % % % 
\chapter{Introduction}
\label{ch:shortintro}
% % % % % % % % % % % % % % % % % % % % % % % % % % % % % % % % 
% =========================================================== %
% % % % % % % % % % % % % % % % % % % % % % % % % % % % % % % % 
In the past cosmology was a speculative science. The scarcity of observational
data meant that many conflicting theories for the evolution of the universe were
entertained, with nothing but personal opinion to differentiate between them.
The explosion in the quantity and quality of observational data in recent years
has led to a much more competitive market-place for ideas about the physical
beginning of the universe.

The Big Bang scenario has emerged as a cohesive framework for the evolution of
the universe from very early times. The observation of the Cosmic
Microwave Background (CMB) provided much supporting evidence for this scenario
\cite{book:kolbturner}. This
relic radiation emitted 300,000 years after the Big Bang continues to
be our primary source of information about the early universe.

The inflationary scenario is an attempt to solve problems with the standard Big
Bang picture and provide an origin for the fluctuations in energy that seeded
the growth of structure in the universe
\cite{Starobinsky:1980te,Guth:1980zm,Albrecht:1982wi,Linde:1981mu,
Starobinsky:1982ee}. These fluctuations link the classical scales of
relativistic gravity with the quantum scales of Planck level physics. There are
many possible realisations of inflation and there has been an explosion in the
number of theoretical models which agree with current observational limits
(for reviews see for example \cite{book:liddle, Alabidi:2008ej, Baumann2009}).

Ground and space-based observations have significantly challenged theoretical
cosmological models with a wealth of new data. The Wilkinson Microwave
Anisotropy Probe (WMAP) mission \cite{Komatsu:2008hk}, in conjunction with supernova
surveys and other evidence, have shown that the fluctuations in
the temperature of the CMB are $10^{5}$ times smaller than the background value
and that the magnitude of the fluctuations is roughly independent of the
angular scales at which they are measured. This is in agreement with the
predictions of inflationary models and has led to other scenarios being ruled
out. An upper bound has been placed
on the amplitude of gravitational wave perturbations and bounds have also been
placed on the deviation of the fluctuations from a purely random Gaussian
distribution.

Constraining the parameter space of inflationary models is an important step
towards limiting the number of observationally viable models, and ultimately
towards identifying one such model as the best candidate to descibe the
physics of the early
universe. 
% In this work theoretical and numerical methods are used to constrain
% particular classes of inflationary models.

The goal of this thesis is to constrain inflationary models in two very different
ways: by deriving theoretical limits on their parameter spaces, and by demonstrating
a numerical calculation which will allow the investigation of higher order
perturbations. Both these methods have the potential to limit the
parameter space of the models investigated and possibly to rule them out.


In Chapter~\ref{ch:introduction} the foundations are laid for these investigations.
The geometry and physics of the Friedmann-Robertson-Walker universe are presented and
inflationary cosmology is introduced to alleviate problems with the standard Big
Bang scenario. Slow roll conditions are then defined to ensure an adequate duration
of inflation. Despite its elegance, this homogeneous cosmology does not provide an
adequate description of our universe. To understand the inhomogeneities that
are present in reality, first order
cosmological perturbation theory is employed. Models with non-canonical
actions can also be considered. The relationships between observable quantities and
the model parameters are altered in this case, meaning these models could be
distinguished from those with canonical actions. The departure of primordial
perturbations from a Gaussian random distribution could also reveal significant
information about the underlying physics at work.


In Part~\ref{part:dbi} of this thesis, theoretical bounds are placed on a class of
non-canonical inflationary models. These models illustrate the dynamics of
extended objects called branes in superstring theory and are considered to be some of
the most promising candidates for achieving inflation using string theory.

Chapter~\ref{ch:dbi-intro} outlines the Dirac-Born-Infeld (DBI) scenario in
terms of the string theoretic background and how it applies in four-dimensions
as a realisation of inflation. The six extra dimensions required by string theory
play an integral role in this scenario. These are compactified into a complex
manifold whose geometry allows extended regions called throats to exist. DBI
inflation consists of a brane moving in one of the throats. The inflaton field is
the radial distance of the brane from the tip of the throat. Translating the
higher-dimensional motion into four dimensions introduces a non-canonical term into
the
effective action. The real nature of the action then enforces an upper bound on the
kinetic energy of the inflaton, allowing a sufficiently long period of inflation. The
total inflaton field variation is directly linked to the
amplitude of tensor modes which can be produced.


In Chapter~\ref{ch:dbi} the repercussions of this relationship between the change in
the field value and the tensor mode amplitude are explored further. In the DBI
scenario,
Baumann \& McAllister \cite{bmpaper} placed a conservative
upper bound on the total production of tensor modes during inflation, by assuming the
brane does not move further than the length of the throat. By considering only the
period of observable inflation, which takes place over a much smaller region of the
throat, we have derived a new bound which is considerably stronger. In the
generic
case, the ratio of the amplitudes of the tensor and scalar perturbations must be less
than
$10^{-7}$. This is below  even the most optimistic forecasts for
the sensitivity of future observational experiments. 

If attention is limited to brane motion down the throat, another complementary bound
on the tensor modes can be derived, which depends on the non-Gaussianity of the
scalar modes produced
during inflation. The DBI scenario is inherently non-Gaussian in nature but, even
assuming the largest levels allowed by observations, the tensor-scalar ratio must
exceed $0.005$. These two bounds are clearly incompatible in the generic
case and only a very fine-tuned selection of model parameters allows the standard DBI
scenario to survive. By taking a more phenomenological approach and allowing the
other parameters to vary,
conditions are found under which the bounds can be relaxed.


A more general class of models which evade the upper bound are identified in
Chapter~\ref{ch:multibrane}. The DBI scenario is characterised by a simple
algebraic relation, in which the sound speed of fluctuations is inversely
proportional to the contribution to the non-Gaussianity. By allowing the
proportionality constant to vary, a new family of actions is derived for which the
bound on the tensor-scalar ratio can be relaxed. 

Instead of considering a single brane moving in the throat, a more natural scenario
might involve multiple branes. These could be created from the energy released by a
brane/anti-brane annihilation and could move up the throat away from the tip.
In \Rref{thomasward}, Thomas \& Ward described the case when these branes are
coincident. When a large number of branes coincide the resultant action is similar
to the single brane action and is restricted by the bounds on the tensor-scalar
ratio. For a small, finite number of branes, however, the action is non-Abelian in
nature and is one of the family of ``bound-relaxing'' actions described above.
Nevertheless, this model is still constrained by observations and, if a detectable
tensor signal is required, only two or three coincident branes
are allowed. This limit on the number of branes is strongly dependent on the
non-Gaussianity and a tightening of the observational bounds could rule out the
possibility of an observable tensor signal from this model.


In Part~\ref{part:numerical}, the focus of the thesis moves from analytical to
numerical
techniques. Second order cosmological perturbations are numerically calculated for
single field canonical inflationary models.

In Chapter~\ref{ch:perts}, the system of equations for the numerical calculation is
developed. In order to understand non-linear perturbative effects, it is necessary to
examine models using perturbation theory beyond first order. The gauge
transformation for second order perturbations is outlined and the effect on
scalar quantities is considered in the uniform curvature gauge. In
\Rref{Malik:2006ir} the Klein-Gordon equation for second order perturbations was
written in terms of the field perturbations alone. This forms the basis of the
numerical calculation once it is transformed into Fourier space. As the original
equation involves terms quadratic in the first order perturbations, the Fourier
transformed equation contains convolutions of these perturbations. As a first step
towards demonstrating the calculation for the full equation, the slow roll version of
the source term is considered in the second order equation. The second order
perturbations can be linked to observable quantities including the curvature
perturbation and the non-Gaussianity parameter.



The Klein-Gordon equations are the central governing equations of the 
calculation described in Chapter~\ref{ch:numericalsystem}. They must first be
rewritten in a form more suitable for numerical work. This involves changing the
time coordinate to the number of elapsed e-foldings and writing the convolution
terms in spherical polar coordinates. Four different single field slow roll
potentials will be investigated. The parameters for these models are set by
comparing
the calculated power spectrum of first order scalar perturbations with the latest
WMAP data. The initial conditions for the background field and perturbations must
also be specified. The second order perturbations are intially set to zero, to
highlight the creation of second order effects. As this is a novel procedure a
thorough description of the implementation of the calculation is given. Where
an analytic solution for one of the convolution terms is possible, this is compared
with the calculated value. Numerical parameters are set by minimising the relative
error of this test calculation.


The results of the numerical calculation are presented in Chapter~\ref{ch:results}.
Three different ranges of the discretised momenta are considered and general
results presented for the quadratic potential. 
% Comment out results?
As expected for a single field slow
roll model, the second order perturbations are highly suppressed compared to the
first order ones. 
The source term of the second order perturbation equation is similar in form to the
power spectrum of first order perturbations. It descreases rapidly until
horizon
crossing after which a more steady amplitude is maintained.
% 
% 
The results for all four potentials are also compared. Differences are apparent in
the behaviour of the models after horizon crossing.
This calculation represents only the first step towards a full numerical integration
of the second order Klein-Gordon equation. The next stages towards this goal are
outlined and the second order equation for single field models without the slow roll
assumption is written in the correct form for numerical use.

% Add section about conclusions?






% % % % % % % % % % % % % % % % % % % % % % % % % % % % % % % % 
% =========================================================== %
% % % % % % % % % % % % % % % % % % % % % % % % % % % % % % % % 
\subsection*{Conventions}
\label{sec:conventions}
% % % % % % % % % % % % % % % % % % % % % % % % % % % % % % % % 
% =========================================================== %
% % % % % % % % % % % % % % % % % % % % % % % % % % % % % % % % 
Throughout this thesis units are chosen such that $\Mpl \equiv (8\pi G )^{-1/2}=
2.4 \times 10^{18}\, {\rm GeV}$ defines the reduced Planck mass and $c=\hbar =1$. 

An overdot ($\dot{~}$) is used for differentiation with respect to proper
time $t$ and a prime ($'$) for differentiation with respect to conformal time
$\eta$. The dagger symbol ($^\dagger$) denotes differentiation with respect to the
number of e-foldings $\N$.
% 
A subscripted comma denotes partial differentiation by the symbol it
precedes, \eg $f_{,\varphi} = \dfrac{\partial f}{\partial \varphi}$.


