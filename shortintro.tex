% % % % % % % % % % % % % % % % % % % % % % % % % % % % 
% shortintro.tex - Ian Huston
% $Id: shortintro.tex,v 1.1 2009/10/15 17:23:17 ith Exp $
% % % % % % % % % % % % % % % % % % % % % % % % % % % % 
% Redefine CVSRevision for this section
\renewcommand{\CVSrevision}%
{\version$Id: shortintro.tex,v 1.1 2009/10/15 17:23:17 ith Exp $}

% % % % % % % % % % % % % % % % % % % % % % % % % % % % % % % % 
% =========================================================== %
% % % % % % % % % % % % % % % % % % % % % % % % % % % % % % % % 
\chapter{Introduction}
\label{ch:shortintro}
% % % % % % % % % % % % % % % % % % % % % % % % % % % % % % % % 
% =========================================================== %
% % % % % % % % % % % % % % % % % % % % % % % % % % % % % % % % 
In the past cosmology was a speculative science. The scarcity of observational
data meant that many conflicting theories of the evolution of the universe were
entertained with nothing but personal opinion to differentiate between them.
The explosion in the quantity and quality of observational data in recent years
has led to a much more competitive market place for ideas about the physical
beginning of everything.

The Big Bang scenario has emerged as a cohesive framework for the evolution of
the universe from very early times. The observation of the Cosmic
Microwave Background (CMB) provided much supporting evidence for this scenario
\cite{book:kolbturner}. This
relic radiation emitted 300,000 years after the Big Bang continues to
be our primary source of information about the early universe.

The inflationary scenario is an attempt to solve problems with the standard Big
Bang picture and provide an origin for the fluctuations in energy that seeded
the growth of structure in the universe
\cite{Starobinsky:1980te,Guth:1980zm,Albrecht:1982wi,Linde:1981mu,
Starobinsky:1982ee}. These fluctuations link the classical scales of
relativistic gravity with the quantum scales of Planck level physics. There are
many possible realisations of inflation and there has been an explosion in the
number of theoretical models which agree with current observational limits
(for reviews see for example \cite{book:liddle, Alabidi:2008ej, Baumann2009}).

Ground and spaced based observations have significantly challenged theoretical
cosmological models with a wealth of new data. The Wilkinson Microwave
Anisotropy Probe mission (WMAP) \cite{Komatsu:2008hk} in conjunction with supernova
surveys and other evidence have shown that the fluctuations in
temperature of the CMB are $10^{-5}$ times smaller than the background value
and that the magnitudes of the fluctuations are roughly independent of the
angular scales at which they are measured. This is in agreement with the
predictions of inflationary models and has led to other scenarios being ruled
out. An upper bound has also been placed
on the amplitude of gravitational wave perturbations and bounds have been
placed on the deviation of the fluctuations from a purely random Gaussian
distribution.

Constraining the parameter space of inflationary models is an important step
towards limiting the number of observationally viable models, and ultimately
singling out one of these as the description of the physics of the early
universe. 
% In this work theoretical and numerical methods are used to constrain
% particular classes of inflationary models.

The goal of this work is to constrain inflationary models in two very different
ways: by deriving theoretical limits on their parameter space and demonstrating
a numerical calculation which will allow the investigation of higher order
perturbations. Both these methods have the potential to seriously limit the
models investigated and possibly rule these scenarios out.

In Part~\ref{part:dbi} of this work theoretical bounds are placed on a class of
non-canonical inflationary models. These models illustrate the dynamics of
extended objects called branes in superstring theory and are considered some of
the most promising candidates for achieving inflation using string theory.
Chapter~\ref{ch:dbi-intro} outlines the Dirac-Born-Infeld (DBI) scenario in
terms of the string theoretic background and how it applies in four-dimensions
as a realisation of inflation. In Chapter~\ref{ch:dbi} two bounds are derived
for the standard DBI model which are incompatible for most of the known
parameter space. A class of models which evade this bound are found in
Chapter~\ref{ch:multibrane}. One particular model in this class is a
multi-coincident brane scenario with a small finite number of branes.


Part~\ref{part:numerical} applies numerical techniques to investigate the
production of second order cosmological perturbations. In
Chapter~\ref{ch:perts} the Klein-Gordon equation of the second order
scalar field perturbation is described purely in terms of the field itself.
This equation is the basis of the numerical calculation in
Chapter~\ref{ch:numericalsystem} which is tested for two standard inflationary
potentials. 


