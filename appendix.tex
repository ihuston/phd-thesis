% % % % % % % % % % % % % % % % % % % % % % 
% appendix.tex - Ian Huston
% 
% % % % % % % % % % % % % % % % % % % % % % 


% % % % % % % % % % % % % % % % % % % % % % % % % % % % % % % % 
% =========================================================== %
% % % % % % % % % % % % % % % % % % % % % % % % % % % % % % % %
\chapter{Appendix}
\label{ch:appendix}
% % % % % % % % % % % % % % % % % % % % % % % % % % % % % % % % 
% =========================================================== %
% % % % % % % % % % % % % % % % % % % % % % % % % % % % % % % %

The following materials supplement the calculations and discussions in the main thesis.

% % % % % % % % % % % % % % % % % % % % % % % % % % % % % % % % 
% =========================================================== %
% % % % % % % % % % % % % % % % % % % % % % % % % % % % % % % % 
\section{Analytic Solution of Generalised Sound Speed Relation}
\label{sec:apx-multi}
% % % % % % % % % % % % % % % % % % % % % % % % % % % % % % % % 
% =========================================================== %
% % % % % % % % % % % % % % % % % % % % % % % % % % % % % % % % 
\eq{eq:defalpha} can be analytically solved in full 
generality without imposing the limits (\ref{eq:Plimits}) on the 
derivatives of the kinetic function. This allows us to determine the 
most general class of models where the non-linearity parameter 
satisfies the condition $\fnleq \propto 1/\cs^2$ at leading order. 

In general \eq{eq:defalpha} takes the form 
% 
\begin{equation}
\label{eq:genPXeqn-multi}
(2-\alpha ) P_{,X}P_{,XX} + 4XP^2_{,XX} = \frac{2\alpha }{3}
X P_{,X}P_{,XXX}
\end{equation}
% 
and this reduces to 
% 
\begin{equation}
\label{eq:genreduce-multi}
\alpha \Upsilon_{,X} = (6-\alpha ) \Upsilon^2 + \frac{3(2-\alpha )}{2}
\frac{\Upsilon}{X} \, ,
\end{equation}
% 
where $\Upsilon \equiv P_{,XX}/P_{,X}$. 
\eq{eq:genreduce-multi} can be transformed into the 
linear equation
% 
\begin{equation}
\label{eq:lineargen-multi}
U_{,X}+ \frac{3(2-\alpha )}{2\alpha} \frac{U}{X} = \frac{\alpha -6}{\alpha}
\end{equation}
% 
after the change of variables $U \equiv 1/\Upsilon$
and the general solution to \eq{eq:lineargen-multi} is given by
%  
\begin{equation}
\label{eq:gensolnlinear-multi}
\frac{P_{,XX}}{P_{,X}} = \frac{1}{X\left[ f_2(\varphi) X^{(\alpha -6)/2\alpha}
-2 \right] } \, .
\end{equation}
% 
Integrating a second time implies that
% 
\begin{equation}
\label{eq:secondint-multi}
P_{,X} = -f_1 (\varphi ) \left( 1- f_2(\varphi ) X^{-s} \right)^{1/(2s)}  \, ,
\end{equation}
% 
where $s \equiv (\alpha -6 )/(2 \alpha)$ and we have redefined 
the arbitrary integration functions $f_i(\varphi )$.  
Finally \eq{eq:secondint-multi} can be formally integrated 
in terms of a hypergeometric function
%  
\begin{equation}
 \label{eq:thirdint-multi}
 P= -f_1X \,{_2}F_1 \left( -\frac{1}{s}, -\frac{1}{2s}; 1-\frac{1}{s}, f_2X^{-s}
\right)  \, ,
\end{equation}
%  
which represents the most general solution for this class of models. 
Note that we have set the
remaining constant of integration to zero to ensure 
that the kinetic function vanishes in the limit of
zero velocity. In fact this expression admits many 
different classes of solution, arising as limits
of the expansion of the hypergeometric function.


% The special case of $\alpha =2$ $(s=-1)$ implies (after a 
% further redefinition of the functions $f_i (\varphi ))$ that 
% % 
% \begin{equation}
% \label{eq:DBIsoln-multi}
% P = -f_1 \sqrt{1-f_2 X} -f_3
% \end{equation}
% % 
% and this corresponds to the standard DBI action (\ref{eq:DBIkinetic}) 
% \cite{chenetal,lidser2}.          


% % % % % % % % % % % % % % % % % % % % % % % % % % % % % % % % 
% =========================================================== %
% % % % % % % % % % % % % % % % % % % % % % % % % % % % % % % % 
\section{Generalised BM bound for Finite \texorpdfstring{$n$}{n} Models}
\label{sec:apx-genbmbound}
% % % % % % % % % % % % % % % % % % % % % % % % % % % % % % % % 
% =========================================================== %
% % % % % % % % % % % % % % % % % % % % % % % % % % % % % % % % 


For completeness we should also consider the 
BM bound \eqref{eq:genBMbound} for the finite $n$ multi-coincident brane models. This is
given by 
% 
\begin{equation}
\label{eq:BMAdS-multi}
r_* < -\frac{42}{N \Neff^2}\sqrt{1 +(n-1)^2Y}\fnleq \,,
\end{equation}
%  
and in the case of an $AdS_5 \times X_5$ throat simplifies to
%  
\begin{equation}
\label{eq:bmadsbound}
r_* < -\frac{5}{\Neff^2} 
\frac{\fnleq}{(n-1)\sqrt{N}} \,.
\end{equation}
%  
Comparing the limits in Eqs.~\eqref{eq:AdSupper-multi} and
\eqref{eq:bmadsbound} 
implies that the bound \eqref{eq:LHbound} is stronger than the corresponding BM
bound \eqref{eq:genBMbound} if 
% 
\begin{equation}
\label{eq:LHstrongerads}
n > 1 -5.5 \times 10^{-14} N^{3/2} \Neff^2 \fnleq \,,
\end{equation}
% 
and this condition is always satisfied if 
% 
\begin{equation}
\label{eq:allNbound-multi}
-5.5 \times 10^{-14} N^{3/2} \Neff^2 \fnleq  <1  \, .
\end{equation}
% 
Moreover, the bound \eqref{eq:allNbound-multi} will itself be satisfied for 
all values of $\fnleq$ and $N$ if it is satisfied when the limits 
$\fnleq =-151$ and $N=75852$ are imposed. Hence, we conclude that the bound
\eqref{eq:LHbound} 
is stronger for $\Neff < 75$. 
In general, it is difficult to quantify 
the magnitude of $\Neff$ without 
imposing further restrictions on the parameters of the models 
and, in particular, on the functional form of the inflaton potential. 
However, if the ratio $\varepsilon_H/P_{,X}$ remains approximately 
constant during the final stages of inflation, one would anticipate that 
$\Neff \lesssim 60$. Nevertheless, if $N \ll 75852$, the bound 
\eqref{eq:LHstrongerads} will only be violated for $n \le 3$ if 
$\Neff \gg 60$.


% % % % % % % % % % % % % % % % % % % % % % % % % % % % % % % % 
% =========================================================== %
% % % % % % % % % % % % % % % % % % % % % % % % % % % % % % % % 
\section{Analytic tests for \texorpdfstring{$\B,\wt{\C}$ and $\wt{\D}$}{B, C and D} terms}
\label{sec:apx-codetests}
% % % % % % % % % % % % % % % % % % % % % % % % % % % % % % % % 
% =========================================================== %
% % % % % % % % % % % % % % % % % % % % % % % % % % % % % % % % 

% 
% 
% Bterm
Analytic solutions can also be found for the $\B$, $\wt{\C}$ and $\wt{\D}$ terms.
The $\B$ term integral, $I_\B$, is given by
% 
\begin{align}
 \label{eq:bintegral-num}
I_\B &= 2\pi \int_{\kmin}^{\kmax} \d q\, q^2\dvp1(\qvi)\B(\kvi,\qvi) \nonumber\\
% 
     &= 2\pi\alpha^2 \int_{\kmin}^{\kmax} \d q\, q^{\frac{3}{2}}
\int_{0}^{\pi} \d\theta\, (k^2 + q^2 -2k q \cos{\theta})^{-1/4}
\cos{\theta}\sin{\theta}\,,
\end{align}
% 
and has the following analytic solution when $\dvp1(q) = \alpha/\sqrt{q}$:
% 
\begin{align}
\label{eq:intb-soln-num}
 I_\B = -\frac{\pi\alpha^2}{168 k^2}\Bigg\{ 
        &-63 k^4 \Bigg[ \log\Biggl(\frac{\sqrt{k}}{\sqrt{k+\kmin} + \sqrt{\kmin}}
                            \Biggr)
         + \log\Biggl( \frac{\sqrt{k+\kmax} +\sqrt{\kmax}}{\sqrt{\kmax-k} +
                      \sqrt{\kmax}}\Biggr) \nonumber \\
        &-\frac{\pi}{2} + \arctan\left( \frac{\sqrt{\kmin}}{\sqrt{k-\kmin}}\right)
        \Bigg] \nonumber\\
% 
        &+\sqrt{\kmax}\Bigg[ \left(-65k^3 + 8k\kmax^2 \right)\left(\sqrt{k+\kmax} +
          \sqrt{\kmax-k}\right) \nonumber \\
        &\qquad +\left(22k^2\kmax -16\kmax^3\right) \left(\sqrt{k+\kmax} -
         \sqrt{\kmax -k} \right) \Bigg] \nonumber \\
        &+\sqrt{\kmin}\Bigg[ \left(65k^3 - 8k\kmin^2 \right)\left(\sqrt{k+\kmin} -
          \sqrt{k-\kmin}\right) \nonumber \\
        &\qquad +\left(-22k^2\kmin +16\kmin^3\right) \left(\sqrt{k+\kmin} +
         \sqrt{k-\kmin} \right) \Bigg] \Bigg\} \,.
\end{align}
% 
If, in addition to $\dvp1(q) = \alpha/\sqrt{q}$, we also take
% 
\begin{equation}
 \dN{\dvp1}(q) = -\frac{\alpha}{\sqrt{q}} -i\frac{\alpha\sqrt{q}}{\beta}\,
\end{equation}
% 
then the $\wt{\C}$ and $\wt{\D}$ terms can be integrated analytically.
The integral of the $\wt{\C}$ term is 
% 
\begin{align}
 \label{eq:cint-num}
I_{\wt{\C}} &= \int\d^3 q\, \dvp1(\qvi) \dN{\dvp1}(\kvi-\qvi) 
    = 2\pi \int \d q\, q^2 \dvp1(\qvi) \wt{\C}(\kvi, \qvi) \nonumber\\
% 
 &= -2\pi\alpha^2 \int_{\kmin}^{\kmax} \d q\, q^{\frac{3}{2}} \int_0^\pi 
     \left( \left(k^2 + q^2 -2kq \cos\theta\right)^{-\frac{1}{4}} \right.\nonumber \\
            &\qquad \qquad \left.+\frac{i}{\beta}\left(k^2 + q^2 -2kq
\cos\theta\right)^{\frac{1}{4}}
        \right) \sin\theta \d \theta\,,
\end{align}
% 
and the analytic solution is given by
% 
\begin{align}
\label{eq:cint-soln-num}
I_{\wt{\C}} = -I_\A -i\frac{\pi\alpha^2}{240 \beta k}\Bigg\{ 
        &15 k^4 \Bigg[ \log\Biggl(\frac{\sqrt{k+\kmin} + \sqrt{\kmin}}{\sqrt{k}}
                            \Biggr)
         + \log\Biggl( \frac{\sqrt{\kmax-k} + \sqrt{\kmax}}{\sqrt{k+\kmax}
                        +\sqrt{\kmax}}\Biggr) \nonumber \\
        &-\frac{\pi}{2} + \arctan\left( \frac{\sqrt{\kmin}}{\sqrt{k-\kmin}}\right)
        \Bigg] \nonumber\\
% 
        &+\sqrt{\kmax}\Bigg[ \left(15k^3 + 136k\kmax^2 \right)\left(\sqrt{k+\kmax} +
          \sqrt{\kmax-k}\right) \nonumber \\
        &\qquad +\left(118k^2\kmax -48\kmax^3\right) \left(\sqrt{k+\kmax} -
         \sqrt{\kmax -k} \right) \Bigg] \nonumber \\
% 
        &-\sqrt{\kmin}\Bigg[ \left(15k^3 + 136k\kmin^2 \right)\left(\sqrt{k+\kmin} +
          \sqrt{k-\kmin}\right) \nonumber \\
        &\qquad +\left(118k^2\kmin +48\kmin^3\right) \left(\sqrt{k+\kmin} -
         \sqrt{k-\kmin} \right) \Bigg] \Bigg\} \,.
\end{align}
 % 
The integral of the $\wt{\D}$ term is 
% 
\begin{align}
 \label{eq:dint-num}
I_{\wt{\D}} &= 2\pi \int \d q\, q^2 \dvp1(\qvi) \wt{\D}(\kvi, \qvi) \\
% 
 &= -2\pi\alpha^2 \int_{\kmin}^{\kmax} \d q\, q^{\frac{3}{2}} \int_0^\pi 
     \left( \left(k^2 + q^2 -2kq \cos\theta\right)^{-\frac{1}{4}} \right.\nonumber\\
% 
        &\qquad\qquad\left.+\frac{i}{\beta}\left(k^2 + q^2 -2kq
\cos\theta\right)^{\frac{1}{4}}
        \right) \cos\theta\sin\theta \d \theta\,,
\end{align}
% 
and the analytic solution is
% 
\begin{align}
\label{eq:dint-soln-num}
I_{\wt{\D}} = -I_\B &-i\frac{\pi\alpha^2}{900\beta k^2}\Bigg\{ \nonumber \\
        &135 k^5 \Bigg[ \log\Biggl(\frac{\sqrt{\kmax-k} + \sqrt{\kmax}}{\sqrt{k}}
                            \Biggr)
         + \log\Biggl( \frac{\sqrt{k+\kmax} +\sqrt{\kmax}}{\sqrt{k+\kmin} +
                          \sqrt{\kmin}}\Biggr) \nonumber \\
        &\qquad -\frac{\pi}{2} + \arctan\left(
\frac{\sqrt{\kmin}}{\sqrt{k-\kmin}}\right)
        \Bigg] \nonumber\\
% 
        &-\sqrt{\kmax}\Bigg[ \left(-185k^4 + 168k^2\kmax^2-32\kmax^4
            \right)\left(\sqrt{k+\kmax} - \sqrt{\kmax-k}\right) \nonumber \\
        &\qquad +\left(70k^3\kmax +16k\kmax^3\right) \left(\sqrt{k+\kmax} +
         \sqrt{\kmax -k} \right) \Bigg] \nonumber \\
% 
        &+\sqrt{\kmin}\Bigg[ \left(-185k^4 + 168k^2\kmin^2 -32\kmax^4
            \right)\left(\sqrt{k+\kmin} - \sqrt{k-\kmin}\right) \nonumber \\
        &\qquad +\left(70k^3\kmin +16k\kmin^3\right) \left(\sqrt{k+\kmin} +
         \sqrt{k-\kmin} \right) \Bigg] \Bigg\} \,.
\end{align}
% 
